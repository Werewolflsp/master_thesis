\ctitle{基于时间自动机的\\中断实时性分析与验证}
% 根据自己的情况选,不用这样复杂
\makeatletter
\ifthu@bachelor\relax\else
  \ifthu@doctor
    \cdegree{工学博士}
  \else
    \ifthu@master
      \cdegree{工学硕士}
    \fi
  \fi
\fi
\makeatother


\cdepartment[软件学院]{软件学院}
\cmajor{软件工程}
\cauthor{刘盛鹏} 
\csupervisor{顾~~~~~~~~~~明 教~授}
% 如果没有副指导老师或者联合指导老师,把下面两行相应的删除即可。
%\cassosupervisor{贺飞教授}
% 日期自动生成,如果你要自己写就改这个cdate
\cdate{二〇一五年四月}

% 博士后部分
% \cfirstdiscipline{计算机科学与技术}
% \cseconddiscipline{系统结构}
% \postdoctordate{2009年7月——2011年7月}

\etitle{Timing Analysis of Interrupt\\by Timed Automata} 
% 这块比较复杂,需要分情况讨论:
% 1. 学术型硕士
%    \edegree:必须为Master of Arts或Master of Science(注意大小写)
%              “哲学、文学、历史学、法学、教育学、艺术学门类,公共管理学科
%               填写Master of Arts,其它填写Master of Science”
%    \emajor:“获得一级学科授权的学科填写一级学科名称,其它填写二级学科名称”
% 2. 专业型硕士
%    \edegree:“填写专业学位英文名称全称”
%    \emajor:“工程硕士填写工程领域,其它专业学位不填写此项”
% 3. 学术型博士
%    \edegree:Doctor of Philosophy(注意大小写)
%    \emajor:“获得一级学科授权的学科填写一级学科名称,其它填写二级学科名称”
% 4. 专业型博士
%    \edegree:“填写专业学位英文名称全称”
%    \emajor:不填写此项
\edegree{Master of Science} 
\emajor{Software Engineering} 
\eauthor{Liu Shengpeng} 
\esupervisor{Professor Gu Min} 
%\eassosupervisor{Professor He Fei} 
% 这个日期也会自动生成,你要改么?
\edate{April, 2015}

% 定义中英文摘要和关键字
\begin{cabstract}
  随着现在嵌入式程序在各行各业的大量应用,针对嵌入式程序的验证需求也越来越多。
  中断,作为嵌入式程序的重要组成部分,对嵌入式程序的实时性起着至关重要的作用。
  然而,中断本身的验证难度却比普通的程序大得多。目前学术界对中断的研究仍停留
  在传统桌面领域,对中断在嵌入式程序中的复杂行为没有一个全面而精确的描述。这
  使得对嵌入式程序以及其中的中断的验证缺乏理论依据。同时,在嵌入式程序中断的
  实时性验证方面,目前应用得最广泛的时间自动机理论还不够完善,无法直接应用到
  真实的系统中。
  
  就此问题,本文从中断的行为模式和实现原理角度,经过深入调研,掌握了市面上主
  流硬件平台和操作系统的中断的行为和具体实现。调研显示,中断作为一个硬件与软
  件结合的并发机制,其行为模式十分多样化。相应的,中断背后的实现机制差异也很
  大。最重要的是,中断的行为不仅仅取决于硬件平台。部分嵌入式操作系统利用底层
  硬件机制,对中断封装和修改,从而改变了中断的行为模式。
  
  经过总结,本文将已知的所有中断归纳为三个类型。本文详细描述了三个中断类型的
  行为模式,深入分析了他们背后的实现原理,涉及到了硬件平台和部分嵌入式操作系
  统。经过适度抽象和合理的假设,本文针对三类中断分别给出了形式化的定义。
  
  同时,为了研究嵌入式程序中的中断的实时性,本文在现有的时间自动机理论上做了
  扩展,将秒表自动机扩展到能够描述现有中断行为的混合自动机。本文以扩展后的混
  合自动机和CCS并行组合为理论基础,构造了上述三类中断的自动机模型,并给出了
  其组成的自动机网络的形式化定义。
  
  至此,本文提出了一个面向嵌入式程序里中断实时性质研究的方法,该方法被应用在
  针对某航空控制系统的中断实时性验证项目上。借由模型检测工具\uppaal ,本文证
  明了该系统的中断实时性不满足业务需求,并对其中违反实时性约束的中断设置给出
  了一条可重现的运行反例。
  
\end{cabstract}

\ckeywords{中断,时间自动机,实时性分析,建模,验证}

\begin{eabstract} 
   As embedded programs being adopted heavily almost everywhere, demand
   for the verification of them have increased greatly during recent years.
   Interrupt, a very important component of all embedded programs, is vital
   to the real-time properties of them. However, it is much more difficult
   to verify an interrupt program than a common one. Acdamic researches
   on interrupts still remain in the tradtional field, focusing on the desktop
   environment. There has been none of full and precise description of the
   complex behavior of interrupts in a embedded program, which leads to
   lack of theoretical support for verification of the embedded programs
   along with interrrupts in it. Meanwhile, the theory of timed automata,
   which is mostly widely applied in verification of real-time properties
   of embedded programs, is not developed enough to be really applicable to
   real systems. 
   
   To address this problem, a wide survey was conducted on the common hardware
   platform and embedded operationg systems. It revealed the fact 
   that the behavior mode of interrupts, which is a parallele mechanism combining
   both hardware and software, is rather diversed. Correspondingly, the 
   implementation behind varies greatly. Most importantly, the behavior mode
   of interrupts depends more than the hardware on which it is running. Some
   embedded operating systems manipulated the interrupt mechanism of the hardware
   laying below, encapsulated and modified it, thus changed the behavior mode
   ultimately. From the perspective of an application program, it is a totally
   new type of interrupts.
    
   The thesis summerized most common interrupts into three types. We decribed
   the behavior mode of the three types in detail, analyzed their implementation
   involving both hardware platform and embedded operationg system on it. With
   moderate abstraction and reasonable assumption, the thesis gives formal
   definition of three types.
   
   To study real-time properties of interrupts in embedded programs, the thesis
   extended current theory of timed automata, extending the stopwatch automaton
   to a new hybrid automton which is exactly suited to model the interrupt
   behavior. Combining the extended hybrid automta and CCS parallele composition,
   the thesis presented models of the three types of interrupts, and it gives
   a formal definition of the network of these automata. 
   
   Thus, the thesis presented a complete scheme for studying the real-time 
   properties of interrupts in embedded programs. The scheme has been applied 
   to a project of verification on real-time properties of interrupts in an 
   air control system. With the help of model checking tool \uppaal, we 
   verified the real-times properties, roved there was a flaw in the interrupt 
   setting and produced a counter example.
\end{eabstract}

\ekeywords{Interrupt, Timed Automata, Real-time Analysis, Modeling, Verification}
