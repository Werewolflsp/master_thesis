
\ctitle{基于时间自动机的\\中断实时性分析与验证}
% 根据自己的情况选,不用这样复杂
\makeatletter
\ifthu@bachelor\relax\else
  \ifthu@doctor
    \cdegree{工学博士}
  \else
    \ifthu@master
      \cdegree{工学硕士}
    \fi
  \fi
\fi
\makeatother


\cdepartment[软件学院]{软件学院}
\cmajor{软件工程}
\cauthor{刘盛鹏} 
\csupervisor{顾明教授}
% 如果没有副指导老师或者联合指导老师,把下面两行相应的删除即可。
%\cassosupervisor{贺飞教授}
% 日期自动生成,如果你要自己写就改这个cdate
%\cdate{\CJKdigits{\the\year}年\CJKnumber{\the\month}月}

% 博士后部分
% \cfirstdiscipline{计算机科学与技术}
% \cseconddiscipline{系统结构}
% \postdoctordate{2009年7月——2011年7月}

\etitle{Timing Analysis on Interrupt\\Based on Timed Automata} 
% 这块比较复杂,需要分情况讨论:
% 1. 学术型硕士
%    \edegree:必须为Master of Arts或Master of Science(注意大小写)
%              “哲学、文学、历史学、法学、教育学、艺术学门类,公共管理学科
%               填写Master of Arts,其它填写Master of Science”
%    \emajor:“获得一级学科授权的学科填写一级学科名称,其它填写二级学科名称”
% 2. 专业型硕士
%    \edegree:“填写专业学位英文名称全称”
%    \emajor:“工程硕士填写工程领域,其它专业学位不填写此项”
% 3. 学术型博士
%    \edegree:Doctor of Philosophy(注意大小写)
%    \emajor:“获得一级学科授权的学科填写一级学科名称,其它填写二级学科名称”
% 4. 专业型博士
%    \edegree:“填写专业学位英文名称全称”
%    \emajor:不填写此项
\edegree{Master of Engineering} 
\emajor{Software Engineering} 
\eauthor{Liu Shengpeng} 
\esupervisor{Professor Gu Min} 
%\eassosupervisor{Professor He Fei} 
% 这个日期也会自动生成,你要改么?
% \edate{December, 2005}

% 定义中英文摘要和关键字
\begin{cabstract}
   本文总结了三种常见的中断类型,分析了他们的软硬件实现。从实时性角度,本文对这
   些中断构造了相应的时间自动机模型。结合实际项目,本文应用该建模方法对某嵌入式
   平台上的中断驱动程序中的中断在\uppaal 中进行建模,并分析和验证了其实时性性质,
   对其中违反实时性约束的中断设置给出了反例。
\end{cabstract}

\ckeywords{中断,时间自动机,实时性,建模,验证}

\begin{eabstract} 
   The thesis summerized three most common types of interrupts and analyzed
   their implementation from the perspectives of both hardware and software.
   Focusing on the timing property, this thesis constructed models of the 
   timed automata of the thses interrupts .It gives all the specifics on 
   abstraction and modelling. The thesis applied the same approach on a set 
   of interrupt-driven software on some embedded platform from an actual 
   project. After analysis and verification on its timing properties, the 
   thesis proved there was a flaw in the interrupt setting and produced a 
   counter example.
\end{eabstract}

\ekeywords{Interrupt, Timed Automata, Timing, Model, Verification}
