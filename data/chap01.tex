%%% Local Variables:
%%% mode: latex
%%% TeX-master: t
%%% End:

\chapter{引言}
%本章介绍中断的概念和现在学术界对中断的研究。就是文献综述。

\section{什么是中断}
中断(Interrupt)是指处理器接收到特殊信号,提示某个事件发生,应当采取对应措施的情况。发出这样的信号的行为被称作中断请求(Interrupt Request,IRQ)。
这个信号可以是来自外围硬件的异步信号,也可以是由运行在处理器上的软件发出同步信号。前者被称为硬件中断(Hardware Interrupt),后者则被称为软件中断(Software Interrupt,常简称为软中断)。

处理器在接收到中断信号以后,通常会保存当前的执行状态,该执行状态被称为上下文(Context)。上下文的内容视各个硬件平台的不同而有所区别,但是大多包含程序计数器
和程序状态字以及部分通用计算寄存器。之后,处理器会根据中断向量表的指示跳转到指定代码片段,即当前中断的处理程序。在执行完中断处理程序之后,处理器会重新加
载之前保存的上下文,继续执行之前被打断的程序。进入中断和退出中断都包含了将保存当前上下文,载入新的上下文的操作。此类操作被称作为上下文切换(Context Switch)。\footnote{严格意义上的上下文切换并不要求保存一个完好的上下文,载入另一个完好的上下文,实际情况中经常发生载入的上下文只有部分内容有意义的情况。
或者对原有上下文并未做出妥善保存即另行载入,相当于舍弃了当前上下文。这在处理器响应中断时尤其多见。}

人们引入中断是为了提高计算机系统的性能。如果没有中断,处理器在接受外部硬件通信时只能采取轮询方式。例如,处理器向某硬件发出某指令并需求其回复时,只能采取
繁忙等待(Busy-waiting)模式,这样就会浪费许多处理器周期。即使在软件层面运用多线程技术进行优化,此类轮询操作依然是性能的损失。在引入中断之后,处理器就可以
专注于当前任务,并且可以在需要时候与外部硬件通信,从而大幅提高运行效率。\footnote{处理器检查中断的原理其实也类似于一个轮询操作。每个周期,处理器会检查特定
的寄存器的某些位来判断是否有中断需要处理。只不过这个处理相比于外部硬件通信快速得多。}后来被用于CPU外部与内部紧急事件的处理、机器故障的处理、时间控制等多个方面,并产生通过软件方式进入中断处理(软中断)的概念。在处理中断时几乎必然会触发两次切换上下文的操作,该操作会涉及内存访问,因此其耗时在中断处理程序本身比较短的时候也不可忽略。过于频繁的中断触发也会在一定程度降低系统的性能。

中断系统在硬件实现上可以是一个包含控制线路的独立系统,也可以被集成进存储器子系统中。对于前者,在IBM个人机上,广泛使用可编程中断控制器(Programmable Interrupt Controller,PIC)来负责中断响应和处理。PIC被连接在若干中断请求设备和处理器的中断引脚之间,从而实现对处理器中断请求线路(多为一针或两针)的复用。作为另一种中断实现的形式,即存储器子系统实现方式,可以将中断端口映射到存储器的地址空间,这样对特定存储器地址的访问实际上是中断请求。一般桌面领域的PC机中,中断控制器一般是集成在处理器内部,\footnote{例如Intel推出的8086A系列处理器中,集成了Intel 8259系列中断控制器。}而在嵌入式领域,很多厂商推出的微控制器实现了外置的中断控制器。\footnote{例如STMicroelectronics推出的STM32F4系列微控制器中,在ARM的CPU外提供了中断控制器}

\section{操作系统中的中断}
在实际应用中,中断的行为并不一直忠实地遵从其硬件实现,操作系统可以在一定程度上改变中断的行为。在嵌入式操作系统中,出于实时性的考虑,一个中断的运行时间不应该过长,但是许多中断需要实现的逻辑功能却比较复杂。因此一个通用的解决方案就是在操作系统中将中断分成两段,前一段在中断触发时立即执行,后一段则延后执行。后一段代码的执行优先级较低,因此有可能被其他低优先级中断甚至是普通代码抢先执行。这样,在编写中断代码的时候,将需要立即执行不容延迟同时运行时间较短的代码放在第一段,其他的对延时不敏感,或者任务繁重,或者需要和其他线程进行同步交互的代码放在第二段,就能在满足实时性要求的前提下完成该中断。eCos等许多操作系统尽管实现方式各异,采用的都是分段中断的模式。\cite{ecos}

\section{对中断的研究}


\subsection{中断与线程}