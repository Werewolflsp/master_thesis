%%% Local Variables:
%%% mode: latex
%%% TeX-master: t
%%% End:

\chapter{引言}
\label{cha:intro}
%引出问题。

\section{选题背景}
\label{sec:background}

计算机领域发展到现在,硬件和软件技术都取得了长足的进步。在过去二三十年里,
随着电子产品的普及,嵌入式程序从一个专业性极强的术语逐渐走向大众。越来越
多的开发人员和学术界的研究者们也将目光投向了这一个新兴的软件类别。然而,
与嵌入式程序开发的火热现状相比,针对嵌入式程序的验证相关的研究还是处于刚
刚起步的阶段。

嵌入式程序自诞生之初,就与硬件紧密结合。最初的嵌入式程序多应用在大型项目
上,例如火箭、载人航天飞机、深海探测器、人造卫星等高精尖科技以及导弹,战
斗机等军事工业和一些高端大型制造业。随着技术进步,电子化的设备越来越多,
逐步取代原有的机械部件,嵌入式程序有逐步进入了汽车等民用制造业。更不用说
近十年来,智能手机普及并引领的智能硬件风潮。嵌入式程序正在逐渐并终将完全
占据所有人的日常生活。

既然应用如此广泛,嵌入式程序的可靠性就显得格外重要。然而,事与愿违的是,
嵌入式程序的安全可靠保障研究和实践比它自身的发展要慢得多。早在形式化验证
技术问世之前,嵌入式程序的测试就比传统桌面程序复杂得多。因为硬件的行为受
环境因素影响太大,在大部分情况下,很难准确预测与之紧密结合的嵌入式程序的
行为。测试的用例范围相对于实际应用场景明显偏小。即使在后来,形式化技术
被用于嵌入式程序的安全保障,其效果依然不够令人满意。一个最典型的例子就是
Ariane 5火箭坠毁事件\cite{www.around.com}。嵌入式程序一旦发生错误,其
后果可能是灾难性的。这也是当前人们认为嵌入式程序的测试和验证工作如此重要
的原因。

人们现在针对嵌入式程序的可靠性保障的研究和应用技术都来源于传统的桌面软件
领域。对桌面软件的安全可靠性研究有很大的借鉴意义。这些研究定义了关于软件
的很多性质,比如最重要,正确性\cite{Harel198061}。然而这些研究很容易忽
视硬件的作用。在传统的桌面软件中,软件和硬件的隔离做得比较好,本身软件直
接和硬件打交道的场合就不多。而且IBM-PC机在事实上统一了桌面硬件标准。除了
少数计算机厂商,绝大部分计算机的硬件都是采用IBM-PC机的架构,这也导致了桌
面应用领域的硬件十分规范,研究起来相对容易。人们对桌面的硬件的重视相对就
远不如软件。

然而,虽然嵌入式程序也是由类似的代码构成,它在运行中却与硬件紧密相连,离
开硬件谈嵌入式软件的安全可靠是不够全面的。而且,最重要的,嵌入式场景中,
硬件的组成、行为完全不同,几乎没有统一标准。事实上,由于这些硬件被用于完
全不同的场合,本身功能就千差万别,没有统一标准也是合理的。另一方面,嵌入
式领域关心的可靠性问题并不局限于传统桌面领域的程序结果的正确可靠。在嵌入
式领域,与软件直接交互的是底层的硬件,而与硬件交互的除了那些代码,更重要
的它身处的物理世界。有很多物理量在嵌入式环境中是无法被忽略的。最典型的一
个就是时间。嵌入式程序通常对实时性要求非常高,程序相应的时间会直接影响硬
件行为,从而影响外部的物理世界。

所以,当我们将目光从传统软件验证的诸如数据竞争,数组越界这类问题转移到与
硬件紧密相关的嵌入式程序里的中断实时性上以后,桌面领域的许多研究手段和成
果都不能简单移植过来。

中断(Interrupt)是指处理器接收到特殊信号,提示某个事件发生,应当采取
对应措施的情况。发出这样的信号的行为被称作中断请求(Interrupt Request,
IRQ)。这个信号可以是来自外围硬件的异步信号,也可以是由运行在处理器上的
软件发出同步信号。前者被称为硬件中断(Hardware Interrupt),后者则被
称为软件中断(Software Interrupt,常简称为软中断)。

处理器在接收到中断信号以后,通常会保存当前的执行状态,该执行状态被称为上
下文(Context)。上下文的内容视各个硬件平台的不同而有所区别,但是大多包
含程序计数器(Program Counter,PC)和程序状态字(Program Status Word, 
PSW)以及部分通用计算寄存器。之后,处理器会根据中断向量表的指示跳转到指定
代码片段,即当前中断的处理程序。在执行完中断处理程序之后,处理器会重新加
载之前保存的上下文,继续执行之前被打断的程序。进入中断和退出中断都包含了
将保存当前上下文,载入新的上下文的操作。此类操作被称作为上下文切换(Context 
Switch)。\footnote{严格意义上的上下文切换并不要求保存一个完好的上下文,
载入另一个完好的上下文,实际情况中经常发生载入的上下文只有部分内容有意义
的情况。或者对原有上下文并未做出妥善保存即另行载入,相当于舍弃了当前上下
文。这在处理器响应中断时尤其多见。}

中断本身从计算机发展的早起就存在了。传统桌面领域对中断的应用也早就驾轻就
熟。然而,传统桌面领域由于其应用特点,中断只是应用程序的一个很小的组成部
分。只有少数相对底层的程序才会涉及中断。而且,这些程序一般对时间等外部世
界的物理量并不敏感,所以在这方面的研究一直比较滞后。

但是,进入嵌入式程序之后,中断成为了一个非常重要的组件。基本上所有的嵌入
式程序都严重依赖中断,以至于许多程序将其主要功能都写在中断里,我们将其称
之为中断驱动程序(Interrupt-driven Program)。而且,由于嵌入式应用场景
本身对实时性要求很高,那么嵌入式程序中的中断的实时性就显得尤为重要。

本文的工作就是建立在此基础上,希望能在嵌入式程序里的中断实时性研究方面做
出一点贡献。

\section{研究现状}
\label{sec:study}

目前,学术界对中断的实时性分析和验证的研究并不充分。但是已经比一二十年前
进步了许多。过去众多研究依然是立足于代码,在做中断的分析的时候更多的是从
数据的角度来看待中断的性质,诸如数据竞争的判定等。从时间性质上来说,从代
码出发会导致执行路径的爆炸式增长,从而会使分析规模急剧扩大以致自动分析几
乎是不可能的任务。

大部分研究存在两个主要问题。其一,对中断的定义过于狭窄。事实上有许多的中
断不能以简单的抢占\pozhehao 执行\pozhehao 退出来看待。许多关于嵌入式领
域中断的研究工作还局限于传统桌面领域的中断理解,然而事实并非如此。其二,
对于嵌入式程序的实时性,现在比较完善的思路是利用近些年出现的时间自动机理
论,借助在传统桌面程序研究领域已经发展得十分成熟的模型检测技术,进行研究。
然而,该方法的基础,即现有时间自动机理论并不能很准确地描述出所有中断的行
为。这两个问题也是本文着力解决的。

\section{研究内容}
\label{sec:subject}

要研究嵌入式程序里中断的实时性,首先需要对嵌入式场景下的中断有一个全面且
深入的认识。由于现有的学术文献对中断的描述都源于桌面环境下的浅显理解,本
文首先用了大量的时间阅读各类嵌入式硬件平台的制造商提供的硬件规格说明(specification)。
在这些规格说明中,硬件制造商会详细描述该平台的中断行为及其背后的硬件机制。
这里包含了很多细节,诸如硬件如何标示中断的触发,硬件如何与CPU通信,CPU
如何进入中断处理,又如何从中断中退出等。这些细节在稍后的研究中将用到。

另一方面,嵌入式场景下的中断行为并不仅仅由其运行的硬件所决定。本文在调研
后发现,部分嵌入式操作系统会为了特殊的应用场合的需求,将底层硬件提供的中
断机制屏蔽,并利用它重新封装成一个新的中断机制提供给上层的应用程序。在应
用程序看来,这是一套全新的中断机制。为了对这些操作系统这样做的原理和新
的中断机制有更加深入的了解,本文阅读了这些操作系统的内核代码以及针对部分
硬件平台的硬件抽象层(HAL)代码,掌握了此类中断的所有细节。

经过总结和归纳,本文将嵌入式领域中的常见中断划分为三个类型,并对其行为模
式和实现原理进行了详尽的阐述和分析。

在充分把握中断的行为模式和实现原理之后,本文为使用时间自动机理论研究其实
时性,还做了以下两项工作。其一,用形式化的语言给出中断的严格定义。自然语
言的描述往往不够精确,且对不同背景的人群容易造成不一样的理解。形式化的定
义不仅可以杜绝此类歧义,而且也为将中断问题纳入自动机理论框架做好了准备。其二,
本文扩展了现有的时间自动机理论,在秒表自动机\cite{Abdeddaim:2002:PJS:646486.694487}
的基础上,扩展成为能够描述现有中
断行为的混合自动机。并且,在该自动机中,本文结合了CCS(calculus of communicating systems,通信系统演算)并行组合\cite{Milner:1989:CC:534666}理论。本文
用该自动机建模描述的不仅仅是单独的中断,而是由多个中断相互作用的系统。因
此,本文给出了扩展的混合自动机与CCS并行组合结合的自动机网络。

在这些基础工作准备好之后,本文给出了上文提到的三类中断的时间自动机网络模
型。这项工作还被应用于某航空控制系统的中断实时性验证。本文从该时间自
动机网络模型出发,借助模型检测工具\uppaal ,验证了该系统的中断实时性质存
在漏洞,并对其中违反实时性约束的中断设置给出了可在真实系统中还原的运行反
例。

\section{论文结构}
\label{sec:structure}
本文第~\ref{cha:intro} 章简要介绍研究内容以及相关背景。在
第~\ref{cha:related_work} 章,本文将介绍一些有关中断分析验证以及时间自
动机的相关工作。在第~\ref{cha:intr} 章,本文将描述对中断行为的分析,并
给出时间自动机理论下的中断模型。在第~\ref{cha:case} 章,本文将描述模型
应用到某航空控制系统进行验证的实验过程以及实验结果。在第
\ref{cha:sum} 章,本文将做全文工作的总结和未来工作的展望。


