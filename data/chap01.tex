%%% Local Variables:
%%% mode: latex
%%% TeX-master: t
%%% End:

\chapter{引言}
\label{cha:intro}
%引出问题。

\section{中断}
\label{sec:intr}
中断(Interrupt)是指处理器接收到特殊信号,提示某个事件发生,应当采取
对应措施的情况。发出这样的信号的行为被称作中断请求(Interrupt Request,
IRQ)。这个信号可以是来自外围硬件的异步信号,也可以是由运行在处理器上的
软件发出同步信号。前者被称为硬件中断(Hardware Interrupt),后者则被
称为软件中断(Software Interrupt,常简称为软中断)。

处理器在接收到中断信号以后,通常会保存当前的执行状态,该执行状态被称为上
下文(Context)。上下文的内容视各个硬件平台的不同而有所区别,但是大多包
含程序计数器和程序状态字以及部分通用计算寄存器。之后,处理器会根据中断向
量表的指示跳转到指定代码片段,即当前中断的处理程序。在执行完中断处理程序
之后,处理器会重新加载之前保存的上下文,继续执行之前被打断的程序。进入中
断和退出中断都包含了将保存当前上下文,载入新的上下文的操作。此类操作被称
作为上下文切换(Context Switch)。\footnote{严格意义上的上下文切换
并不要求保存一个完好的上下文,载入另一个完好的上下文,实际情况中经常发生
载入的上下文只有部分内容有意义的情况。或者对原有上下文并未做出妥善保存即
另行载入,相当于舍弃了当前上下文。这在处理器响应中断时尤其多见。}

人们引入中断是为了提高计算机系统的性能。如果没有中断,处理器在接受外部硬
件通信时只能采取轮询方式。例如,处理器向某硬件发出某指令并需求其回复时,
只能采取繁忙等待(Busy-waiting)模式,这样就会浪费许多处理器周期。即使
在软件层面运用多线程技术进行优化,此类轮询操作依然是性能的损失。在引入中
断之后,处理器就可以专注于当前任务,并且可以在需要时候与外部硬件通信,从
而大幅提高运行效率。\footnote{处理器检查中断的原理其实也类似于一个轮询
操作。每个周期,处理器会检查特定的寄存器的某些位来判断是否有中断需要处理。
只不过这个处理相比于外部硬件通信快速得多。}后来被用于CPU外部与内部紧急事
件的处理、机器故障的处理、时间控制等多个方面,并产生通过软件方式进入中断
处理(软中断)的概念。在处理中断时几乎必然会触发两次切换上下文的操作,该
操作会涉及内存访问,因此其耗时在中断处理程序本身比较短的时候也不可忽略。
过于频繁的中断触发也会在一定程度降低系统的性能。

中断系统在硬件实现上可以是一个包含控制线路的独立系统,也可以被集成进存储
器子系统中。对于前者,在IBM个人机上,广泛使用可编程中断控制器(Programmable 
Interrupt Controller,PIC)来负责中断响应和处理。PIC被连接在若干中
断请求设备和处理器的中断引脚之间,从而实现对处理器中断请求线路(多为一针
或两针)的复用。作为另一种中断实现的形式,即存储器子系统实现方式,可以将
中断端口映射到存储器的地址空间,这样对特定存储器地址的访问实际上是中断请
求。一般桌面领域的PC机中,中断控制器一般是集成在处理器内部,\footnote{
例如Intel推出的8086A系列处理器中,集成了Intel 8259系列中断控制器。}
而在嵌入式领域,很多厂商推出的微控制器实现了外置的中断控制器。\footnote{
例如STMicroelectronics推出的STM32F4系列微控制器中,在ARM的CPU外提
供了中断控制器。}

\subsection{操作系统中的中断}
\label{subsec:intr_OS}
在实际应用中,中断的行为并不一直忠实地遵从其硬件实现,操作系统可以在一定
程度上改变中断的行为。在嵌入式操作系统中,出于实时性的考虑,一个中断的运
行时间不应该过长,但是许多中断需要实现的逻辑功能却比较复杂。因此一个通用
的解决方案就是在操作系统中将中断分成两段,前一段在中断触发时立即执行,后
一段则延后执行。后一段代码的执行优先级较低,因此有可能被其他低优先级中断
甚至是普通代码抢先执行。这样,在编写中断代码的时候,将需要立即执行不容延
迟同时运行时间较短的代码放在第一段,其他的对延时不敏感,或者任务繁重,或
者需要和其他线程进行同步交互的代码放在第二段,就能在满足实时性要求的前提
下完成该中断。eCos等许多操作系统尽管实现方式各异,采用的都是分段中断的模
式。\cite{ecos}图~\ref{fig:intr_exec}和图~\ref{fig:ecos_intr_exec}分
别展示了上述两种中断的执行流。

\begin{figure}[H]
	\centering
	\subcaptionbox{普通中断的执行流\label{fig:intr_exec}}
	{%LaTeX with PSTricks extensions
%%Creator: 0.91_64bit
%%Please note this file requires PSTricks extensions
\psset{xunit=.5pt,yunit=.5pt,runit=.5pt}
\begin{pspicture}(294.47600898,389.9795322)
{
\newrgbcolor{curcolor}{0 0 0}
\pscustom[linewidth=0.83516997,linecolor=curcolor]
{
\newpath
\moveto(58.110382,311.5881822)
\lineto(58.110382,31.5881922)
}
}
{
\newrgbcolor{curcolor}{0 0 0}
\pscustom[linewidth=0.83516997,linecolor=curcolor]
{
\newpath
\moveto(118.110382,311.5881922)
\lineto(118.110382,31.5881922)
}
}
{
\newrgbcolor{curcolor}{0 0 0}
\pscustom[linewidth=0.5905543,linecolor=curcolor]
{
\newpath
\moveto(197.988082,241.5881922)
\lineto(197.988082,101.5881922)
}
}
{
\newrgbcolor{curcolor}{0 0 0}
\pscustom[linewidth=0.5905543,linecolor=curcolor]
{
\newpath
\moveto(257.988082,241.5881922)
\lineto(257.988082,101.5881922)
}
}
{
\newrgbcolor{curcolor}{0 0 0}
\pscustom[linewidth=1,linecolor=curcolor]
{
\newpath
\moveto(87.703652,347.3024722)
\lineto(87.703652,188.7310422)
}
}
{
\newrgbcolor{curcolor}{0 0 0}
\pscustom[linestyle=none,fillstyle=solid,fillcolor=curcolor]
{
\newpath
\moveto(87.703652,198.7310422)
\lineto(83.703652,202.7310422)
\lineto(87.703652,188.7310422)
\lineto(91.703652,202.7310422)
\lineto(87.703652,198.7310422)
\closepath
}
}
{
\newrgbcolor{curcolor}{0 0 0}
\pscustom[linewidth=1,linecolor=curcolor]
{
\newpath
\moveto(87.703652,198.7310422)
\lineto(83.703652,202.7310422)
\lineto(87.703652,188.7310422)
\lineto(91.703652,202.7310422)
\lineto(87.703652,198.7310422)
\closepath
}
}
{
\newrgbcolor{curcolor}{0 0 0}
\pscustom[linewidth=1.03499508,linecolor=curcolor]
{
\newpath
\moveto(230.549942,260.3553422)
\lineto(230.549942,90.4912722)
}
}
{
\newrgbcolor{curcolor}{0 0 0}
\pscustom[linestyle=none,fillstyle=solid,fillcolor=curcolor]
{
\newpath
\moveto(230.549942,100.84122299)
\lineto(226.40996168,104.9812033)
\lineto(230.549942,90.4912722)
\lineto(234.68992232,104.9812033)
\lineto(230.549942,100.84122299)
\closepath
}
}
{
\newrgbcolor{curcolor}{0 0 0}
\pscustom[linewidth=1.03499508,linecolor=curcolor]
{
\newpath
\moveto(230.549942,100.84122299)
\lineto(226.40996168,104.9812033)
\lineto(230.549942,90.4912722)
\lineto(234.68992232,104.9812033)
\lineto(230.549942,100.84122299)
\closepath
}
}
{
\newrgbcolor{curcolor}{0 0 0}
\pscustom[linewidth=1.03499508,linecolor=curcolor]
{
\newpath
\moveto(87.879012,171.7477922)
\lineto(87.879012,1.8837222)
}
}
{
\newrgbcolor{curcolor}{0 0 0}
\pscustom[linestyle=none,fillstyle=solid,fillcolor=curcolor]
{
\newpath
\moveto(87.879012,12.23367299)
\lineto(83.73903168,16.3736533)
\lineto(87.879012,1.8837222)
\lineto(92.01899232,16.3736533)
\lineto(87.879012,12.23367299)
\closepath
}
}
{
\newrgbcolor{curcolor}{0 0 0}
\pscustom[linewidth=1.03499508,linecolor=curcolor]
{
\newpath
\moveto(87.879012,12.23367299)
\lineto(83.73903168,16.3736533)
\lineto(87.879012,1.8837222)
\lineto(92.01899232,16.3736533)
\lineto(87.879012,12.23367299)
\closepath
}
}
{
\newrgbcolor{curcolor}{0 0 0}
\pscustom[linewidth=1,linecolor=curcolor]
{
\newpath
\moveto(90.549942,184.4453322)
\lineto(229.121372,264.4453322)
}
}
{
\newrgbcolor{curcolor}{0 0 0}
\pscustom[linestyle=none,fillstyle=solid,fillcolor=curcolor]
{
\newpath
\moveto(217.64458474,262.94302193)
\lineto(230.26267805,265.12456983)
\lineto(222.06390164,255.28813233)
\curveto(222.4221528,258.50783971)(220.62767763,261.59399949)(217.64458474,262.94302193)
\closepath
}
}
{
\newrgbcolor{curcolor}{0 0 0}
\pscustom[linewidth=0.6875,linecolor=curcolor]
{
\newpath
\moveto(217.64458474,262.94302193)
\lineto(230.26267805,265.12456983)
\lineto(222.06390164,255.28813233)
\curveto(222.4221528,258.50783971)(220.62767763,261.59399949)(217.64458474,262.94302193)
\closepath
}
}
{
\newrgbcolor{curcolor}{0 0 0}
\pscustom[linewidth=1,linecolor=curcolor]
{
\newpath
\moveto(229.121372,88.7310422)
\lineto(87.692802,183.0167622)
}
}
{
\newrgbcolor{curcolor}{0 0 0}
\pscustom[linestyle=none,fillstyle=solid,fillcolor=curcolor]
{
\newpath
\moveto(94.12652583,173.39487155)
\lineto(86.57805307,183.73875811)
\lineto(99.02951461,180.74935301)
\curveto(95.98457696,179.64336288)(94.01354121,176.66686471)(94.12652583,173.39487155)
\closepath
}
}
{
\newrgbcolor{curcolor}{0 0 0}
\pscustom[linewidth=0.6875,linecolor=curcolor]
{
\newpath
\moveto(94.12652583,173.39487155)
\lineto(86.57805307,183.73875811)
\lineto(99.02951461,180.74935301)
\curveto(95.98457696,179.64336288)(94.01354121,176.66686471)(94.12652583,173.39487155)
\closepath
}
}
{
\newrgbcolor{curcolor}{0 0 0}
\pscustom[linestyle=none,fillstyle=solid,fillcolor=curcolor]
{
\newpath
\moveto(6.93359412,375.8193764)
\lineto(6.93359412,372.79203265)
\lineto(16.40625037,372.79203265)
\lineto(16.40625037,375.8193764)
\lineto(6.93359412,375.8193764)
\closepath
\moveto(6.93359412,372.20609515)
\lineto(6.93359412,369.08109515)
\lineto(16.40625037,369.08109515)
\lineto(16.40625037,372.20609515)
\lineto(6.93359412,372.20609515)
\closepath
\moveto(17.77343787,375.4287514)
\curveto(17.77343787,371.3271889)(17.80598996,368.7881264)(17.87109412,367.8115639)
\lineto(16.40625037,367.12797015)
\lineto(16.40625037,368.49515765)
\lineto(6.93359412,368.49515765)
\lineto(6.93359412,367.71390765)
\lineto(5.46875037,366.93265765)
\curveto(5.53385454,368.16963682)(5.56640662,369.82979307)(5.56640662,371.9131264)
\curveto(5.56640662,373.99645973)(5.53385454,375.78682432)(5.46875037,377.28422015)
\lineto(6.93359412,376.4053139)
\lineto(16.21093787,376.4053139)
\lineto(17.18750037,377.3818764)
\lineto(18.65234412,376.0146889)
\lineto(17.77343787,375.4287514)
\closepath
\moveto(9.86328162,385.29203265)
\lineto(9.86328162,379.04203265)
\lineto(13.18359412,379.04203265)
\lineto(13.18359412,385.29203265)
\lineto(9.86328162,385.29203265)
\closepath
\moveto(3.90625037,384.12015765)
\lineto(4.19921912,384.31547015)
\curveto(5.24088579,383.59932432)(6.05468787,382.88317848)(6.64062537,382.16703265)
\curveto(7.22656287,381.51599098)(7.22656287,380.86494932)(6.64062537,380.21390765)
\curveto(6.05468787,379.56286598)(5.66406287,379.69307432)(5.46875037,380.60453265)
\curveto(5.33854204,381.58109515)(4.81770871,382.75297015)(3.90625037,384.12015765)
\closepath
\moveto(6.64062537,389.6865639)
\lineto(6.83593787,389.97953265)
\curveto(8.33333371,389.19828265)(9.37500037,388.5146889)(9.96093787,387.9287514)
\curveto(10.54687537,387.40791807)(10.48177121,386.72432432)(9.76562537,385.87797015)
\lineto(13.08593787,385.87797015)
\curveto(14.32291704,387.76599098)(15.03906287,389.13317848)(15.23437537,389.97953265)
\lineto(17.18750037,388.7100014)
\curveto(16.66666704,388.57979307)(16.14583371,388.28682432)(15.62500037,387.83109515)
\lineto(13.67187537,385.87797015)
\lineto(18.65234412,385.87797015)
\lineto(20.01953162,387.24515765)
\lineto(21.97265662,385.29203265)
\lineto(14.55078162,385.29203265)
\lineto(14.55078162,379.04203265)
\lineto(19.53125037,379.04203265)
\lineto(20.99609412,380.5068764)
\lineto(23.04687537,378.45609515)
\lineto(3.22265662,378.45609515)
\curveto(2.37630246,378.45609515)(1.59505246,378.3584389)(0.87890662,378.1631264)
\lineto(0.00000037,379.04203265)
\lineto(8.49609412,379.04203265)
\lineto(8.49609412,385.29203265)
\lineto(4.19921912,385.29203265)
\lineto(2.73437537,385.09672015)
\lineto(1.95312537,385.87797015)
\lineto(8.88671912,385.87797015)
\curveto(8.75651079,386.91963682)(8.00781287,388.18916807)(6.64062537,389.6865639)
\closepath
\moveto(18.06640662,384.70609515)
\lineto(20.11718787,383.4365639)
\curveto(19.53125037,383.30635557)(18.97786496,382.88317848)(18.45703162,382.16703265)
\curveto(17.93619829,381.45088682)(17.12239621,380.53942848)(16.01562537,379.43265765)
\lineto(15.62500037,379.5303139)
\curveto(16.73177121,381.15791807)(17.54557329,382.88317848)(18.06640662,384.70609515)
\closepath
}
}
{
\newrgbcolor{curcolor}{0 0 0}
\pscustom[linestyle=none,fillstyle=solid,fillcolor=curcolor]
{
\newpath
\moveto(34.37500037,383.4365639)
\lineto(34.37500037,380.5068764)
\lineto(38.67187537,380.5068764)
\lineto(38.67187537,383.4365639)
\lineto(34.37500037,383.4365639)
\closepath
\moveto(39.94140662,383.4365639)
\lineto(39.94140662,380.5068764)
\lineto(44.53125037,380.5068764)
\lineto(44.53125037,383.4365639)
\lineto(39.94140662,383.4365639)
\closepath
\moveto(34.37500037,379.9209389)
\lineto(34.37500037,376.9912514)
\lineto(38.67187537,376.9912514)
\lineto(38.67187537,379.9209389)
\lineto(34.37500037,379.9209389)
\closepath
\moveto(39.94140662,379.9209389)
\lineto(39.94140662,376.9912514)
\lineto(44.53125037,376.9912514)
\lineto(44.53125037,379.9209389)
\lineto(39.94140662,379.9209389)
\closepath
\moveto(41.40625037,372.6943764)
\curveto(42.44791704,372.62927223)(43.22916704,372.59672015)(43.75000037,372.59672015)
\curveto(44.27083371,372.59672015)(44.53125037,372.85713682)(44.53125037,373.37797015)
\lineto(44.53125037,376.4053139)
\lineto(39.94140662,376.4053139)
\lineto(39.94140662,372.01078265)
\lineto(38.67187537,371.42484515)
\lineto(38.67187537,376.4053139)
\lineto(34.37500037,376.4053139)
\lineto(34.37500037,371.7178139)
\lineto(33.00781287,371.1318764)
\curveto(33.07291704,372.8896889)(33.10546912,375.26599098)(33.10546912,378.26078265)
\curveto(33.10546912,381.25557432)(33.07291704,383.50166807)(33.00781287,384.9990639)
\lineto(34.66796912,384.0225014)
\lineto(38.86718787,384.0225014)
\curveto(38.73697954,385.06416807)(37.98828162,385.9756264)(36.62109412,386.7568764)
\lineto(36.81640662,387.04984515)
\curveto(37.92317746,386.85453265)(38.99739621,386.43135557)(40.03906287,385.7803139)
\lineto(42.96875037,388.1240639)
\lineto(36.42578162,388.1240639)
\curveto(35.51432329,388.1240639)(34.70052121,388.02640765)(33.98437537,387.83109515)
\lineto(33.10546912,388.7100014)
\lineto(43.06640662,388.7100014)
\lineto(43.94531287,389.58890765)
\lineto(45.70312537,387.7334389)
\curveto(44.66145871,387.79854307)(42.87109412,387.04984515)(40.33203162,385.48734515)
\curveto(40.59244829,384.90140765)(40.46223996,384.4131264)(39.94140662,384.0225014)
\lineto(44.33593787,384.0225014)
\lineto(45.01953162,384.9990639)
\lineto(46.48437537,383.92484515)
\lineto(45.80078162,383.2412514)
\lineto(45.80078162,372.8896889)
\curveto(45.86588579,371.78291807)(45.24739621,371.00166807)(43.94531287,370.5459389)
\curveto(43.88020871,371.45739723)(43.03385454,372.04333473)(41.40625037,372.3037514)
\lineto(41.40625037,372.6943764)
\closepath
\moveto(26.95312537,388.5146889)
\lineto(27.14843787,388.7100014)
\curveto(28.05989621,388.25427223)(28.77604204,387.79854307)(29.29687537,387.3428139)
\curveto(29.88281287,386.88708473)(30.17578162,386.43135557)(30.17578162,385.9756264)
\curveto(30.17578162,385.51989723)(29.94791704,385.06416807)(29.49218787,384.6084389)
\curveto(29.10156287,384.15270973)(28.80859412,384.34802223)(28.61328162,385.1943764)
\curveto(28.41796912,386.10583473)(27.86458371,387.21260557)(26.95312537,388.5146889)
\closepath
\moveto(30.27343787,372.59672015)
\curveto(30.92447954,371.81547015)(31.77083371,371.1318764)(32.81250037,370.5459389)
\curveto(33.91927121,370.02510557)(35.87239621,369.69958473)(38.67187537,369.5693764)
\curveto(41.47135454,369.50427223)(44.62890662,369.66703265)(48.14453162,370.05765765)
\lineto(48.14453162,369.66703265)
\curveto(46.84244829,369.34151182)(46.22395871,368.75557432)(46.28906287,367.90922015)
\curveto(43.22916704,367.90922015)(40.72265662,367.97432432)(38.76953162,368.10453265)
\curveto(36.88151079,368.23474098)(35.31901079,368.52770973)(34.08203162,368.9834389)
\curveto(32.91015662,369.3740639)(31.93359412,369.9600014)(31.15234412,370.7412514)
\curveto(30.43619829,371.5225014)(29.91536496,371.88057432)(29.58984412,371.81547015)
\curveto(29.26432329,371.81547015)(28.74348996,371.42484515)(28.02734412,370.64359515)
\curveto(27.37630246,369.92744932)(26.88802121,369.24385557)(26.56250037,368.5928139)
\lineto(25.09765662,369.86234515)
\curveto(26.07421912,370.83890765)(27.37630246,371.68526182)(29.00390662,372.40140765)
\lineto(29.00390662,380.60453265)
\lineto(28.32031287,380.60453265)
\curveto(27.47395871,380.60453265)(26.69270871,380.5068764)(25.97656287,380.3115639)
\lineto(25.09765662,381.19047015)
\lineto(28.90625037,381.19047015)
\lineto(29.49218787,382.16703265)
\lineto(31.05468787,380.99515765)
\lineto(30.27343787,380.3115639)
\lineto(30.27343787,372.59672015)
\closepath
}
}
{
\newrgbcolor{curcolor}{0 0 0}
\pscustom[linestyle=none,fillstyle=solid,fillcolor=curcolor]
{
\newpath
\moveto(61.03515662,380.21390765)
\curveto(61.10026079,381.71130348)(61.13281287,383.17614723)(61.13281287,384.6084389)
\curveto(61.13281287,386.10583473)(61.10026079,387.57067848)(61.03515662,389.00297015)
\lineto(62.50000037,388.1240639)
\lineto(69.62890662,388.1240639)
\lineto(70.50781287,389.00297015)
\lineto(71.87500037,387.63578265)
\lineto(70.99609412,387.04984515)
\curveto(70.99609412,384.05505348)(71.02864621,382.10192848)(71.09375037,381.19047015)
\lineto(69.72656287,380.7021889)
\lineto(69.72656287,381.8740639)
\lineto(62.40234412,381.8740639)
\lineto(62.40234412,380.79984515)
\lineto(61.03515662,380.21390765)
\closepath
\moveto(62.40234412,387.5381264)
\lineto(62.40234412,382.4600014)
\lineto(69.72656287,382.4600014)
\lineto(69.72656287,387.5381264)
\lineto(62.40234412,387.5381264)
\closepath
\moveto(59.76562537,378.9443764)
\lineto(69.92187537,378.9443764)
\lineto(70.99609412,380.01859515)
\lineto(72.55859412,378.3584389)
\lineto(66.60156287,378.3584389)
\lineto(66.60156287,374.15922015)
\lineto(69.43359412,374.15922015)
\lineto(70.50781287,375.2334389)
\lineto(72.07031287,373.57328265)
\lineto(66.60156287,373.57328265)
\lineto(66.60156287,369.08109515)
\lineto(70.41015662,369.08109515)
\lineto(71.58203162,370.25297015)
\lineto(73.24218787,368.49515765)
\lineto(61.52343787,368.49515765)
\curveto(60.67708371,368.49515765)(59.89583371,368.3975014)(59.17968787,368.2021889)
\lineto(58.30078162,369.08109515)
\lineto(65.33203162,369.08109515)
\lineto(65.33203162,373.57328265)
\lineto(63.57421912,373.57328265)
\curveto(62.72786496,373.57328265)(61.94661496,373.4756264)(61.23046912,373.2803139)
\lineto(60.35156287,374.15922015)
\lineto(65.33203162,374.15922015)
\lineto(65.33203162,378.3584389)
\lineto(62.98828162,378.3584389)
\curveto(62.14192746,378.3584389)(61.36067746,378.26078265)(60.64453162,378.06547015)
\lineto(59.76562537,378.9443764)
\closepath
\moveto(54.88281287,387.1475014)
\curveto(54.03645871,387.01729307)(52.66927121,386.85453265)(50.78125037,386.65922015)
\lineto(50.78125037,387.04984515)
\curveto(52.01822954,387.24515765)(53.45052121,387.60323057)(55.07812537,388.1240639)
\curveto(56.77083371,388.64489723)(57.87760454,389.13317848)(58.39843787,389.58890765)
\lineto(59.66796912,387.83109515)
\curveto(59.08203162,387.83109515)(57.91015662,387.66833473)(56.15234412,387.3428139)
\lineto(56.15234412,382.4600014)
\lineto(57.61718787,382.4600014)
\lineto(58.69140662,383.53422015)
\lineto(60.25390662,381.8740639)
\lineto(56.15234412,381.8740639)
\lineto(56.15234412,380.01859515)
\curveto(57.51953162,379.43265765)(58.46354204,378.84672015)(58.98437537,378.26078265)
\curveto(59.57031287,377.73994932)(59.66796912,377.12145973)(59.27734412,376.4053139)
\curveto(58.88671912,375.75427223)(58.49609412,375.91703265)(58.10546912,376.89359515)
\curveto(57.71484412,377.87015765)(57.06380246,378.78161598)(56.15234412,379.62797015)
\curveto(56.15234412,372.85713682)(56.18489621,368.88578265)(56.25000037,367.71390765)
\lineto(54.78515662,367.0303139)
\curveto(54.85026079,369.50427223)(54.88281287,373.2803139)(54.88281287,378.3584389)
\curveto(53.71093787,375.36364723)(52.18098996,372.85713682)(50.29296912,370.83890765)
\lineto(50.09765662,371.1318764)
\curveto(51.39973996,373.15010557)(52.40885454,375.10323057)(53.12500037,376.9912514)
\curveto(53.84114621,378.87927223)(54.32942746,380.5068764)(54.58984412,381.8740639)
\lineto(53.22265662,381.8740639)
\curveto(52.37630246,381.8740639)(51.59505246,381.77640765)(50.87890662,381.58109515)
\lineto(50.00000037,382.4600014)
\lineto(54.88281287,382.4600014)
\lineto(54.88281287,387.1475014)
\closepath
}
}
{
\newrgbcolor{curcolor}{0 0 0}
\pscustom[linestyle=none,fillstyle=solid,fillcolor=curcolor]
{
\newpath
\moveto(78.02734412,387.3428139)
\lineto(79.49218787,386.46390765)
\lineto(93.84765662,386.46390765)
\lineto(95.31250037,387.9287514)
\lineto(97.36328162,385.87797015)
\lineto(79.49218787,385.87797015)
\curveto(79.49218787,383.33890765)(79.45963579,381.06026182)(79.39453162,379.04203265)
\curveto(79.32942746,377.08890765)(79.00390662,375.13578265)(78.41796912,373.18265765)
\curveto(77.89713579,371.29463682)(76.85546912,369.30895973)(75.29296912,367.2256264)
\lineto(75.00000037,367.4209389)
\curveto(76.30208371,369.69958473)(77.11588579,371.78291807)(77.44140662,373.6709389)
\curveto(77.83203162,375.6240639)(78.02734412,377.70739723)(78.02734412,379.9209389)
\curveto(78.09244829,382.19958473)(78.09244829,384.67354307)(78.02734412,387.3428139)
\closepath
\moveto(85.83984412,389.8818764)
\curveto(87.01171912,389.4912514)(87.82552121,389.1006264)(88.28125037,388.7100014)
\curveto(88.73697954,388.38448057)(88.89973996,387.99385557)(88.76953162,387.5381264)
\curveto(88.70442746,387.1475014)(88.50911496,386.85453265)(88.18359412,386.65922015)
\curveto(87.92317746,386.46390765)(87.63020871,386.72432432)(87.30468787,387.44047015)
\curveto(87.04427121,388.15661598)(86.49088579,388.87276182)(85.64453162,389.58890765)
\lineto(85.83984412,389.8818764)
\closepath
\moveto(84.57031287,369.47172015)
\curveto(85.67708371,369.34151182)(86.52343787,369.27640765)(87.10937537,369.27640765)
\curveto(87.69531287,369.27640765)(88.02083371,369.63448057)(88.08593787,370.3506264)
\lineto(88.08593787,377.47953265)
\lineto(83.20312537,377.47953265)
\curveto(82.35677121,377.47953265)(81.57552121,377.3818764)(80.85937537,377.1865639)
\lineto(79.98046912,378.06547015)
\lineto(88.28125037,378.06547015)
\curveto(88.28125037,378.58630348)(87.98828162,379.1396889)(87.40234412,379.7256264)
\curveto(86.81640662,380.37666807)(86.19791704,380.96260557)(85.54687537,381.4834389)
\lineto(85.74218787,381.77640765)
\curveto(86.65364621,381.51599098)(87.69531287,381.06026182)(88.86718787,380.40922015)
\lineto(92.67578162,383.33890765)
\lineto(84.76562537,383.33890765)
\curveto(83.91927121,383.33890765)(83.13802121,383.2412514)(82.42187537,383.0459389)
\lineto(81.54296912,383.92484515)
\lineto(92.77343787,383.92484515)
\lineto(93.84765662,384.90140765)
\lineto(95.50781287,382.8506264)
\curveto(94.79166704,382.8506264)(93.94531287,382.62276182)(92.96875037,382.16703265)
\curveto(92.05729204,381.77640765)(90.82031287,381.0928139)(89.25781287,380.1162514)
\curveto(89.58333371,379.7256264)(89.71354204,379.36755348)(89.64843787,379.04203265)
\curveto(89.58333371,378.71651182)(89.38802121,378.39099098)(89.06250037,378.06547015)
\lineto(94.92187537,378.06547015)
\lineto(95.80078162,379.1396889)
\lineto(97.65625037,377.28422015)
\curveto(96.54947954,377.28422015)(95.34505246,376.43786598)(94.04296912,374.74515765)
\lineto(93.65234412,374.94047015)
\lineto(95.01953162,377.47953265)
\lineto(89.35546912,377.47953265)
\lineto(89.35546912,369.9600014)
\curveto(89.35546912,368.65791807)(88.70442746,367.71390765)(87.40234412,367.12797015)
\curveto(87.20703162,367.97432432)(86.26302121,368.5928139)(84.57031287,368.9834389)
\lineto(84.57031287,369.47172015)
\closepath
}
}
{
\newrgbcolor{curcolor}{0 0 0}
\pscustom[linestyle=none,fillstyle=solid,fillcolor=curcolor]
{
\newpath
\moveto(112.59765662,383.14359515)
\curveto(112.53255246,386.39880348)(112.46744829,388.57979307)(112.40234412,389.6865639)
\lineto(114.84375037,388.61234515)
\lineto(113.96484412,387.9287514)
\lineto(113.96484412,383.2412514)
\lineto(117.18750037,383.53422015)
\lineto(117.87109412,384.6084389)
\lineto(119.43359412,383.4365639)
\lineto(118.65234412,382.75297015)
\curveto(118.32682329,380.73474098)(118.26171912,378.3584389)(118.45703162,375.6240639)
\curveto(118.65234412,372.8896889)(119.49869829,370.96911598)(120.99609412,369.86234515)
\lineto(122.36328162,374.2568764)
\lineto(122.75390662,374.15922015)
\curveto(122.55859412,373.44307432)(122.39583371,372.4990639)(122.26562537,371.3271889)
\curveto(122.20052121,370.1553139)(122.33072954,369.1787514)(122.65625037,368.3975014)
\curveto(123.04687537,367.6162514)(122.68880246,367.45349098)(121.58203162,367.90922015)
\curveto(120.54036496,368.29984515)(119.59635454,369.11364723)(118.75000037,370.3506264)
\curveto(117.90364621,371.65270973)(117.38281287,373.21520973)(117.18750037,375.0381264)
\curveto(116.99218787,376.86104307)(117.02473996,379.49776182)(117.28515662,382.94828265)
\lineto(113.96484412,382.6553139)
\curveto(113.89973996,380.3115639)(113.70442746,378.29333473)(113.37890662,376.6006264)
\curveto(114.87630246,375.88448057)(115.78776079,375.26599098)(116.11328162,374.74515765)
\curveto(116.43880246,374.28942848)(116.47135454,373.73604307)(116.21093787,373.0850014)
\curveto(116.01562537,372.4990639)(115.65755246,372.56416807)(115.13671912,373.2803139)
\curveto(114.68098996,373.99645973)(113.99739621,374.74515765)(113.08593787,375.52640765)
\curveto(111.71875037,371.75036598)(109.44010454,368.88578265)(106.25000037,366.93265765)
\lineto(106.05468787,367.2256264)
\curveto(108.85416704,369.43916807)(110.80729204,372.43395973)(111.91406287,376.2100014)
\curveto(110.87239621,376.92614723)(109.96093787,377.47953265)(109.17968787,377.87015765)
\lineto(109.27734412,378.26078265)
\curveto(110.12369829,377.93526182)(111.06770871,377.5771889)(112.10937537,377.1865639)
\curveto(112.36979204,378.42354307)(112.53255246,380.21390765)(112.59765662,382.55765765)
\lineto(110.25390662,382.36234515)
\lineto(109.47265662,381.97172015)
\lineto(108.39843787,382.8506264)
\lineto(112.59765662,383.14359515)
\closepath
\moveto(100.09765662,376.89359515)
\curveto(100.81380246,377.02380348)(102.40885454,377.5771889)(104.88281287,378.5537514)
\lineto(104.88281287,383.4365639)
\lineto(103.32031287,383.4365639)
\curveto(102.47395871,383.4365639)(101.69270871,383.33890765)(100.97656287,383.14359515)
\lineto(100.09765662,384.0225014)
\lineto(104.88281287,384.0225014)
\curveto(104.88281287,386.30114723)(104.85026079,388.25427223)(104.78515662,389.8818764)
\lineto(107.12890662,388.80765765)
\lineto(106.25000037,388.22172015)
\lineto(106.25000037,384.0225014)
\lineto(107.51953162,384.0225014)
\lineto(108.59375037,385.09672015)
\lineto(110.25390662,383.4365639)
\lineto(106.25000037,383.4365639)
\lineto(106.25000037,379.04203265)
\lineto(109.76562537,380.5068764)
\lineto(109.96093787,380.1162514)
\lineto(106.25000037,378.06547015)
\lineto(106.25000037,370.25297015)
\curveto(106.38020871,368.69047015)(105.66406287,367.6162514)(104.10156287,367.0303139)
\curveto(104.16666704,367.8115639)(103.19010454,368.52770973)(101.17187537,369.1787514)
\lineto(101.17187537,369.5693764)
\curveto(102.73437537,369.3740639)(103.74348996,369.30895973)(104.19921912,369.3740639)
\curveto(104.65494829,369.3740639)(104.88281287,369.73213682)(104.88281287,370.44828265)
\lineto(104.88281287,377.3818764)
\curveto(103.84114621,376.86104307)(102.79947954,376.30765765)(101.75781287,375.72172015)
\lineto(101.26953162,375.13578265)
\lineto(100.09765662,376.89359515)
\closepath
}
}
{
\newrgbcolor{curcolor}{0 0 0}
\pscustom[linestyle=none,fillstyle=solid,fillcolor=curcolor]
{
\newpath
\moveto(135.15625038,387.24515765)
\lineto(143.45703163,387.24515765)
\lineto(144.82421913,388.5146889)
\lineto(146.48437538,386.65922015)
\lineto(138.37890663,386.65922015)
\curveto(137.53255246,386.65922015)(136.75130246,386.5615639)(136.03515663,386.3662514)
\lineto(135.15625038,387.24515765)
\closepath
\moveto(136.03515663,369.86234515)
\curveto(137.79296913,369.73213682)(138.96484413,369.66703265)(139.55078163,369.66703265)
\curveto(140.13671913,369.66703265)(140.42968788,370.05765765)(140.42968788,370.83890765)
\lineto(140.42968788,380.79984515)
\lineto(136.71875038,380.79984515)
\curveto(135.87239621,380.79984515)(135.09114621,380.7021889)(134.37500038,380.5068764)
\lineto(133.49609413,381.38578265)
\lineto(144.72656288,381.38578265)
\lineto(146.19140663,382.75297015)
\lineto(147.85156288,380.79984515)
\lineto(141.79687538,380.79984515)
\lineto(141.79687538,370.3506264)
\curveto(141.86197954,368.5928139)(141.04817746,367.55114723)(139.35546913,367.2256264)
\curveto(139.35546913,368.13708473)(138.24869829,368.85323057)(136.03515663,369.3740639)
\lineto(136.03515663,369.86234515)
\closepath
\moveto(131.25000038,389.6865639)
\lineto(133.00781288,388.3193764)
\curveto(132.55208371,388.18916807)(131.77083371,387.47302223)(130.66406288,386.1709389)
\curveto(129.55729204,384.93395973)(128.15755246,383.69698057)(126.46484412,382.4600014)
\lineto(126.26953162,382.75297015)
\curveto(127.37630246,383.79463682)(128.38541704,384.96651182)(129.29687538,386.26859515)
\curveto(130.27343788,387.63578265)(130.92447954,388.77510557)(131.25000038,389.6865639)
\closepath
\moveto(131.44531288,378.5537514)
\lineto(131.44531288,367.90922015)
\lineto(129.98046913,367.0303139)
\curveto(130.04557329,368.46260557)(130.07812538,372.43395973)(130.07812538,378.9443764)
\curveto(128.58072954,377.3818764)(127.11588579,376.11234515)(125.68359412,375.13578265)
\lineto(125.48828162,375.4287514)
\curveto(126.85546913,376.7959389)(128.15755246,378.32588682)(129.39453163,380.01859515)
\curveto(130.69661496,381.71130348)(131.67317746,383.33890765)(132.32421913,384.90140765)
\lineto(133.98437538,383.4365639)
\lineto(133.10546913,382.94828265)
\curveto(132.25911496,381.71130348)(131.54296913,380.73474098)(130.95703163,380.01859515)
\lineto(132.03125038,379.1396889)
\lineto(131.44531288,378.5537514)
\closepath
}
}
{
\newrgbcolor{curcolor}{0 0 0}
\pscustom[linestyle=none,fillstyle=solid,fillcolor=curcolor]
{
\newpath
\moveto(162.79296913,367.6162514)
\curveto(162.85807329,368.5928139)(162.89062538,369.53682432)(162.89062538,370.44828265)
\lineto(162.89062538,376.69828265)
\curveto(162.89062538,377.47953265)(162.85807329,378.22823057)(162.79296913,378.9443764)
\lineto(165.03906288,377.9678139)
\lineto(164.16015663,377.28422015)
\lineto(164.16015663,370.9365639)
\curveto(164.16015663,370.02510557)(164.19270871,369.11364723)(164.25781288,368.2021889)
\lineto(162.79296913,367.6162514)
\closepath
\moveto(158.98437538,379.3350014)
\lineto(161.13281288,378.26078265)
\lineto(160.35156288,377.67484515)
\curveto(160.28645871,374.87536598)(159.92838579,372.75948057)(159.27734413,371.3271889)
\curveto(158.69140663,369.9600014)(157.19401079,368.65791807)(154.78515663,367.4209389)
\lineto(154.58984413,367.71390765)
\curveto(156.41276079,368.95088682)(157.61718788,370.28552223)(158.20312538,371.7178139)
\curveto(158.78906288,373.15010557)(159.04947954,375.68916807)(158.98437538,379.3350014)
\closepath
\moveto(166.99218788,370.05765765)
\lineto(166.99218788,376.50297015)
\curveto(166.99218788,377.54463682)(166.95963579,378.42354307)(166.89453163,379.1396889)
\lineto(169.04296913,378.1631264)
\lineto(168.26171913,377.47953265)
\lineto(168.26171913,370.44828265)
\curveto(168.19661496,369.73213682)(168.48958371,369.3740639)(169.14062538,369.3740639)
\lineto(170.50781288,369.3740639)
\curveto(171.02864621,369.3740639)(171.32161496,369.69958473)(171.38671913,370.3506264)
\curveto(171.45182329,371.00166807)(171.51692746,372.01078265)(171.58203163,373.37797015)
\lineto(171.97265663,373.37797015)
\curveto(171.97265663,372.07588682)(172.03776079,371.16442848)(172.16796913,370.64359515)
\curveto(172.29817746,370.12276182)(172.59114621,369.79724098)(173.04687538,369.66703265)
\curveto(172.65625038,368.75557432)(171.94010454,368.29984515)(170.89843788,368.29984515)
\lineto(168.45703163,368.29984515)
\curveto(167.48046913,368.29984515)(166.99218788,368.88578265)(166.99218788,370.05765765)
\closepath
\moveto(168.35937538,381.2881264)
\curveto(167.77343788,382.32979307)(167.18750038,383.20869932)(166.60156288,383.92484515)
\lineto(166.79687538,384.12015765)
\curveto(168.29427121,383.14359515)(169.33593788,382.32979307)(169.92187538,381.6787514)
\curveto(170.50781288,381.02770973)(170.57291704,380.37666807)(170.11718788,379.7256264)
\curveto(169.66145871,379.07458473)(169.33593788,378.97692848)(169.14062538,379.43265765)
\curveto(169.01041704,379.88838682)(168.81510454,380.37666807)(168.55468788,380.8975014)
\curveto(166.01562538,380.7021889)(163.99739621,380.5068764)(162.50000038,380.3115639)
\curveto(161.06770871,380.1162514)(160.05859413,379.82328265)(159.47265663,379.43265765)
\lineto(158.49609413,381.19047015)
\curveto(159.08203163,381.25557432)(159.79817746,381.74385557)(160.64453163,382.6553139)
\curveto(161.49088579,383.56677223)(162.30468788,384.70609515)(163.08593788,386.07328265)
\lineto(160.74218788,386.07328265)
\curveto(159.89583371,386.07328265)(159.11458371,385.9756264)(158.39843788,385.7803139)
\lineto(157.51953163,386.65922015)
\lineto(163.67187538,386.65922015)
\curveto(163.54166704,387.70088682)(162.95572954,388.7100014)(161.91406288,389.6865639)
\lineto(162.01171913,389.97953265)
\curveto(163.63932329,389.32849098)(164.58333371,388.74255348)(164.84375038,388.22172015)
\curveto(165.10416704,387.76599098)(164.97395871,387.24515765)(164.45312538,386.65922015)
\lineto(169.14062538,386.65922015)
\lineto(170.50781288,387.9287514)
\lineto(172.16796913,386.07328265)
\lineto(163.57421913,386.07328265)
\lineto(165.03906288,384.9990639)
\lineto(163.86718788,384.8037514)
\curveto(162.17447954,383.0459389)(160.93750038,381.84151182)(160.15625038,381.19047015)
\curveto(161.52343788,381.19047015)(164.25781288,381.22302223)(168.35937538,381.2881264)
\closepath
\moveto(158.00781288,384.51078265)
\curveto(155.79427121,378.19567848)(154.58984413,374.61494932)(154.39453163,373.76859515)
\curveto(154.26432329,372.98734515)(154.16666704,371.97823057)(154.10156288,370.7412514)
\lineto(154.10156288,368.3975014)
\curveto(154.10156288,367.94177223)(153.71093788,367.87666807)(152.92968788,368.2021889)
\curveto(152.14843788,368.52770973)(151.95312538,369.30895973)(152.34375038,370.5459389)
\curveto(152.79947954,371.84802223)(152.83203163,372.66182432)(152.44140663,372.98734515)
\curveto(152.05078163,373.37797015)(151.36718788,373.70349098)(150.39062538,373.96390765)
\lineto(150.39062538,374.35453265)
\curveto(151.56250038,374.28942848)(152.27864621,374.2568764)(152.53906288,374.2568764)
\curveto(152.79947954,374.32198057)(153.12500038,374.61494932)(153.51562538,375.13578265)
\curveto(153.90625038,375.65661598)(155.27343788,378.84672015)(157.61718788,384.70609515)
\lineto(158.00781288,384.51078265)
\closepath
\moveto(150.39062538,383.14359515)
\curveto(152.60416704,382.10192848)(153.71093788,381.25557432)(153.71093788,380.60453265)
\curveto(153.71093788,380.01859515)(153.54817746,379.5303139)(153.22265663,379.1396889)
\curveto(152.89713579,378.7490639)(152.57161496,378.9443764)(152.24609413,379.7256264)
\curveto(151.98567746,380.57198057)(151.30208371,381.64619932)(150.19531288,382.94828265)
\lineto(150.39062538,383.14359515)
\closepath
\moveto(152.73437538,388.7100014)
\curveto(154.55729204,387.79854307)(155.59895871,387.08239723)(155.85937538,386.5615639)
\curveto(156.11979204,386.04073057)(156.02213579,385.48734515)(155.56640663,384.90140765)
\curveto(155.17578163,384.31547015)(154.81770871,384.47823057)(154.49218788,385.3896889)
\curveto(154.16666704,386.30114723)(153.51562538,387.31026182)(152.53906288,388.41703265)
\lineto(152.73437538,388.7100014)
\closepath
}
}
{
\newrgbcolor{curcolor}{0 0 0}
\pscustom[linestyle=none,fillstyle=solid,fillcolor=curcolor]
{
\newpath
\moveto(183.53850513,300.7747333)
\lineto(183.53850513,294.3294208)
\lineto(190.56975513,294.3294208)
\lineto(190.56975513,300.7747333)
\lineto(183.53850513,300.7747333)
\closepath
\moveto(182.07366138,301.3606708)
\curveto(182.07366138,303.8997333)(182.04110929,305.82030622)(181.97600513,307.12238955)
\lineto(184.31975513,306.0481708)
\lineto(183.53850513,305.2669208)
\lineto(183.53850513,301.3606708)
\lineto(190.37444263,301.3606708)
\lineto(191.15569263,302.5325458)
\lineto(192.81584888,301.26301455)
\lineto(191.93694263,300.48176455)
\lineto(191.93694263,294.81770205)
\curveto(191.93694263,294.10155622)(191.96949471,293.3528583)(192.03459888,292.5716083)
\lineto(190.56975513,291.9856708)
\lineto(190.56975513,293.7434833)
\lineto(183.53850513,293.7434833)
\curveto(183.53850513,288.79556663)(183.57105721,285.89843122)(183.63616138,285.05207705)
\lineto(181.97600513,284.1731708)
\curveto(182.04110929,285.67056663)(182.07366138,288.8606708)(182.07366138,293.7434833)
\lineto(175.33538013,293.7434833)
\lineto(175.33538013,292.5716083)
\lineto(173.87053638,291.7903583)
\curveto(173.93564054,292.9622333)(173.96819263,294.65494163)(173.96819263,296.8684833)
\curveto(173.96819263,299.08202497)(173.93564054,300.87238955)(173.87053638,302.23957705)
\lineto(175.33538013,301.3606708)
\lineto(182.07366138,301.3606708)
\closepath
\moveto(175.33538013,300.7747333)
\lineto(175.33538013,294.3294208)
\lineto(182.07366138,294.3294208)
\lineto(182.07366138,300.7747333)
\lineto(175.33538013,300.7747333)
\closepath
}
}
{
\newrgbcolor{curcolor}{0 0 0}
\pscustom[linestyle=none,fillstyle=solid,fillcolor=curcolor]
{
\newpath
\moveto(200.62834888,303.80207705)
\curveto(201.60491138,303.08593122)(202.28850513,302.43488955)(202.67913013,301.84895205)
\curveto(203.13485929,301.32811872)(203.13485929,300.74218122)(202.67913013,300.09113955)
\curveto(202.28850513,299.44009788)(201.99553638,299.53775413)(201.80022388,300.3841083)
\curveto(201.60491138,301.29556663)(201.14918221,302.36978538)(200.43303638,303.60676455)
\lineto(200.62834888,303.80207705)
\closepath
\moveto(207.36663013,304.2903583)
\lineto(209.12444263,303.02082705)
\curveto(208.73381763,302.89061872)(208.37574471,302.59764997)(208.05022388,302.1419208)
\curveto(207.72470304,301.68619163)(207.04110929,300.70962913)(205.99944263,299.2122333)
\lineto(205.70647388,299.4075458)
\curveto(206.61793221,301.49087913)(207.17131763,303.1184833)(207.36663013,304.2903583)
\closepath
\moveto(205.12053638,296.2825458)
\curveto(205.12053638,292.9622333)(205.15308846,290.48827497)(205.21819263,288.8606708)
\lineto(203.75334888,288.2747333)
\curveto(203.81845304,290.09764997)(203.85100513,292.50650413)(203.85100513,295.5012958)
\curveto(203.13485929,293.67837913)(201.89788013,291.82291038)(200.14006763,289.93488955)
\lineto(199.84709888,290.13020205)
\curveto(201.47470304,292.40884788)(202.61402596,294.98046247)(203.26506763,297.8450458)
\lineto(201.40959888,297.8450458)
\lineto(200.53069263,297.6497333)
\lineto(199.74944263,298.4309833)
\lineto(203.85100513,298.4309833)
\curveto(203.85100513,301.81639997)(203.81845304,304.61587913)(203.75334888,306.8294208)
\lineto(205.90178638,305.75520205)
\lineto(205.12053638,305.0716083)
\lineto(205.12053638,298.4309833)
\lineto(207.07366138,298.4309833)
\lineto(208.05022388,299.4075458)
\lineto(209.61272388,297.8450458)
\lineto(205.12053638,297.8450458)
\lineto(205.12053638,296.77082705)
\curveto(206.81324471,295.72916038)(207.92001554,294.9153583)(208.44084888,294.3294208)
\curveto(209.02678638,293.7434833)(209.12444263,293.09244163)(208.73381763,292.3762958)
\curveto(208.34319263,291.66014997)(207.92001554,291.82291038)(207.46428638,292.86457705)
\curveto(207.07366138,293.97134788)(206.29241138,295.1106708)(205.12053638,296.2825458)
\closepath
\moveto(198.28459888,288.2747333)
\lineto(198.28459888,298.8216083)
\curveto(198.28459888,300.31900413)(198.25204679,302.76041038)(198.18694263,306.14582705)
\lineto(200.33538013,305.0716083)
\lineto(199.55413013,304.38801455)
\lineto(199.55413013,287.78645205)
\lineto(205.31584888,287.78645205)
\lineto(206.39006763,288.8606708)
\lineto(208.05022388,287.20051455)
\lineto(199.65178638,287.20051455)
\lineto(198.87053638,286.3216083)
\lineto(197.50334888,287.59113955)
\lineto(198.28459888,288.2747333)
\closepath
\moveto(214.98381763,284.5637958)
\curveto(215.04892179,288.2747333)(215.08147388,292.63671247)(215.08147388,297.6497333)
\lineto(211.46819263,297.6497333)
\curveto(211.59840096,293.5481708)(211.20777596,290.6184833)(210.29631763,288.8606708)
\curveto(209.44996346,287.1028583)(208.01767179,285.50780622)(205.99944263,284.07551455)
\lineto(205.70647388,284.3684833)
\curveto(207.07366138,285.60546247)(208.08277596,286.80988955)(208.73381763,287.98176455)
\curveto(209.38485929,289.15363955)(209.77548429,290.45572288)(209.90569263,291.88801455)
\curveto(210.10100513,293.32030622)(210.19866138,295.33853538)(210.19866138,297.94270205)
\curveto(210.19866138,300.61197288)(210.16610929,303.02082705)(210.10100513,305.16926455)
\lineto(211.37053638,304.38801455)
\curveto(212.73772388,304.64843122)(213.87704679,304.94139997)(214.78850513,305.2669208)
\curveto(215.76506763,305.59244163)(216.51376554,305.98306663)(217.03459888,306.4387958)
\lineto(218.59709888,304.77863955)
\curveto(218.01116138,304.77863955)(217.42522388,304.71353538)(216.83928638,304.58332705)
\curveto(216.25334888,304.51822288)(214.46298429,304.2903583)(211.46819263,303.8997333)
\lineto(211.46819263,298.2356708)
\lineto(216.54631763,298.2356708)
\lineto(217.71819263,299.4075458)
\lineto(219.47600513,297.6497333)
\lineto(216.35100513,297.6497333)
\lineto(216.35100513,289.2512958)
\curveto(216.35100513,288.53514997)(216.38355721,287.20051455)(216.44866138,285.24738955)
\lineto(214.98381763,284.5637958)
\closepath
}
}
{
\newrgbcolor{curcolor}{0 0 0}
\pscustom[linestyle=none,fillstyle=solid,fillcolor=curcolor]
{
\newpath
\moveto(234.02678638,300.28645205)
\curveto(233.96168221,303.54166038)(233.89657804,305.72264997)(233.83147388,306.8294208)
\lineto(236.27288013,305.75520205)
\lineto(235.39397388,305.0716083)
\lineto(235.39397388,300.3841083)
\lineto(238.61663013,300.67707705)
\lineto(239.30022388,301.7512958)
\lineto(240.86272388,300.5794208)
\lineto(240.08147388,299.89582705)
\curveto(239.75595304,297.87759788)(239.69084888,295.5012958)(239.88616138,292.7669208)
\curveto(240.08147388,290.0325458)(240.92782804,288.11197288)(242.42522388,287.00520205)
\lineto(243.79241138,291.3997333)
\lineto(244.18303638,291.30207705)
\curveto(243.98772388,290.58593122)(243.82496346,289.6419208)(243.69475513,288.4700458)
\curveto(243.62965096,287.2981708)(243.75985929,286.3216083)(244.08538013,285.5403583)
\curveto(244.47600513,284.7591083)(244.11793221,284.59634788)(243.01116138,285.05207705)
\curveto(241.96949471,285.44270205)(241.02548429,286.25650413)(240.17913013,287.4934833)
\curveto(239.33277596,288.79556663)(238.81194263,290.35806663)(238.61663013,292.1809833)
\curveto(238.42131763,294.00389997)(238.45386971,296.64061872)(238.71428638,300.09113955)
\lineto(235.39397388,299.7981708)
\curveto(235.32886971,297.4544208)(235.13355721,295.43619163)(234.80803638,293.7434833)
\curveto(236.30543221,293.02733747)(237.21689054,292.40884788)(237.54241138,291.88801455)
\curveto(237.86793221,291.43228538)(237.90048429,290.87889997)(237.64006763,290.2278583)
\curveto(237.44475513,289.6419208)(237.08668221,289.70702497)(236.56584888,290.4231708)
\curveto(236.11011971,291.13931663)(235.42652596,291.88801455)(234.51506763,292.66926455)
\curveto(233.14788013,288.89322288)(230.86923429,286.02863955)(227.67913013,284.07551455)
\lineto(227.48381763,284.3684833)
\curveto(230.28329679,286.58202497)(232.23642179,289.57681663)(233.34319263,293.3528583)
\curveto(232.30152596,294.06900413)(231.39006763,294.62238955)(230.60881763,295.01301455)
\lineto(230.70647388,295.40363955)
\curveto(231.55282804,295.07811872)(232.49683846,294.7200458)(233.53850513,294.3294208)
\curveto(233.79892179,295.56639997)(233.96168221,297.35676455)(234.02678638,299.70051455)
\lineto(231.68303638,299.50520205)
\lineto(230.90178638,299.11457705)
\lineto(229.82756763,299.9934833)
\lineto(234.02678638,300.28645205)
\closepath
\moveto(221.52678638,294.03645205)
\curveto(222.24293221,294.16666038)(223.83798429,294.7200458)(226.31194263,295.6966083)
\lineto(226.31194263,300.5794208)
\lineto(224.74944263,300.5794208)
\curveto(223.90308846,300.5794208)(223.12183846,300.48176455)(222.40569263,300.28645205)
\lineto(221.52678638,301.1653583)
\lineto(226.31194263,301.1653583)
\curveto(226.31194263,303.44400413)(226.27939054,305.39712913)(226.21428638,307.0247333)
\lineto(228.55803638,305.95051455)
\lineto(227.67913013,305.36457705)
\lineto(227.67913013,301.1653583)
\lineto(228.94866138,301.1653583)
\lineto(230.02288013,302.23957705)
\lineto(231.68303638,300.5794208)
\lineto(227.67913013,300.5794208)
\lineto(227.67913013,296.18488955)
\lineto(231.19475513,297.6497333)
\lineto(231.39006763,297.2591083)
\lineto(227.67913013,295.20832705)
\lineto(227.67913013,287.39582705)
\curveto(227.80933846,285.83332705)(227.09319263,284.7591083)(225.53069263,284.1731708)
\curveto(225.59579679,284.9544208)(224.61923429,285.67056663)(222.60100513,286.3216083)
\lineto(222.60100513,286.7122333)
\curveto(224.16350513,286.5169208)(225.17261971,286.45181663)(225.62834888,286.5169208)
\curveto(226.08407804,286.5169208)(226.31194263,286.87499372)(226.31194263,287.59113955)
\lineto(226.31194263,294.5247333)
\curveto(225.27027596,294.00389997)(224.22860929,293.45051455)(223.18694263,292.86457705)
\lineto(222.69866138,292.27863955)
\lineto(221.52678638,294.03645205)
\closepath
}
}
{
\newrgbcolor{curcolor}{0 0 0}
\pscustom[linestyle=none,fillstyle=solid,fillcolor=curcolor]
{
\newpath
\moveto(256.58538013,304.38801455)
\lineto(264.88616138,304.38801455)
\lineto(266.25334888,305.6575458)
\lineto(267.91350513,303.80207705)
\lineto(259.80803638,303.80207705)
\curveto(258.96168221,303.80207705)(258.18043221,303.7044208)(257.46428638,303.5091083)
\lineto(256.58538013,304.38801455)
\closepath
\moveto(257.46428638,287.00520205)
\curveto(259.22209888,286.87499372)(260.39397388,286.80988955)(260.97991138,286.80988955)
\curveto(261.56584888,286.80988955)(261.85881763,287.20051455)(261.85881763,287.98176455)
\lineto(261.85881763,297.94270205)
\lineto(258.14788013,297.94270205)
\curveto(257.30152596,297.94270205)(256.52027596,297.8450458)(255.80413013,297.6497333)
\lineto(254.92522388,298.52863955)
\lineto(266.15569263,298.52863955)
\lineto(267.62053638,299.89582705)
\lineto(269.28069263,297.94270205)
\lineto(263.22600513,297.94270205)
\lineto(263.22600513,287.4934833)
\curveto(263.29110929,285.7356708)(262.47730721,284.69400413)(260.78459888,284.3684833)
\curveto(260.78459888,285.27994163)(259.67782804,285.99608747)(257.46428638,286.5169208)
\lineto(257.46428638,287.00520205)
\closepath
\moveto(252.67913013,306.8294208)
\lineto(254.43694263,305.4622333)
\curveto(253.98121346,305.33202497)(253.19996346,304.61587913)(252.09319263,303.3137958)
\curveto(250.98642179,302.07681663)(249.58668221,300.83983747)(247.89397388,299.6028583)
\lineto(247.69866138,299.89582705)
\curveto(248.80543221,300.93749372)(249.81454679,302.10936872)(250.72600513,303.41145205)
\curveto(251.70256763,304.77863955)(252.35360929,305.91796247)(252.67913013,306.8294208)
\closepath
\moveto(252.87444263,295.6966083)
\lineto(252.87444263,285.05207705)
\lineto(251.40959888,284.1731708)
\curveto(251.47470304,285.60546247)(251.50725513,289.57681663)(251.50725513,296.0872333)
\curveto(250.00985929,294.5247333)(248.54501554,293.25520205)(247.11272388,292.27863955)
\lineto(246.91741138,292.5716083)
\curveto(248.28459888,293.9387958)(249.58668221,295.46874372)(250.82366138,297.16145205)
\curveto(252.12574471,298.85416038)(253.10230721,300.48176455)(253.75334888,302.04426455)
\lineto(255.41350513,300.5794208)
\lineto(254.53459888,300.09113955)
\curveto(253.68824471,298.85416038)(252.97209888,297.87759788)(252.38616138,297.16145205)
\lineto(253.46038013,296.2825458)
\lineto(252.87444263,295.6966083)
\closepath
}
}
{
\newrgbcolor{curcolor}{0 0 0}
\pscustom[linestyle=none,fillstyle=solid,fillcolor=curcolor]
{
\newpath
\moveto(284.22209888,284.7591083)
\curveto(284.28720304,285.7356708)(284.31975513,286.67968122)(284.31975513,287.59113955)
\lineto(284.31975513,293.84113955)
\curveto(284.31975513,294.62238955)(284.28720304,295.37108747)(284.22209888,296.0872333)
\lineto(286.46819263,295.1106708)
\lineto(285.58928638,294.42707705)
\lineto(285.58928638,288.0794208)
\curveto(285.58928638,287.16796247)(285.62183846,286.25650413)(285.68694263,285.3450458)
\lineto(284.22209888,284.7591083)
\closepath
\moveto(280.41350513,296.4778583)
\lineto(282.56194263,295.40363955)
\lineto(281.78069263,294.81770205)
\curveto(281.71558846,292.01822288)(281.35751554,289.90233747)(280.70647388,288.4700458)
\curveto(280.12053638,287.1028583)(278.62314054,285.80077497)(276.21428638,284.5637958)
\lineto(276.01897388,284.85676455)
\curveto(277.84189054,286.09374372)(279.04631763,287.42837913)(279.63225513,288.8606708)
\curveto(280.21819263,290.29296247)(280.47860929,292.83202497)(280.41350513,296.4778583)
\closepath
\moveto(288.42131763,287.20051455)
\lineto(288.42131763,293.64582705)
\curveto(288.42131763,294.68749372)(288.38876554,295.56639997)(288.32366138,296.2825458)
\lineto(290.47209888,295.3059833)
\lineto(289.69084888,294.62238955)
\lineto(289.69084888,287.59113955)
\curveto(289.62574471,286.87499372)(289.91871346,286.5169208)(290.56975513,286.5169208)
\lineto(291.93694263,286.5169208)
\curveto(292.45777596,286.5169208)(292.75074471,286.84244163)(292.81584888,287.4934833)
\curveto(292.88095304,288.14452497)(292.94605721,289.15363955)(293.01116138,290.52082705)
\lineto(293.40178638,290.52082705)
\curveto(293.40178638,289.21874372)(293.46689054,288.30728538)(293.59709888,287.78645205)
\curveto(293.72730721,287.26561872)(294.02027596,286.94009788)(294.47600513,286.80988955)
\curveto(294.08538013,285.89843122)(293.36923429,285.44270205)(292.32756763,285.44270205)
\lineto(289.88616138,285.44270205)
\curveto(288.90959888,285.44270205)(288.42131763,286.02863955)(288.42131763,287.20051455)
\closepath
\moveto(289.78850513,298.4309833)
\curveto(289.20256763,299.47264997)(288.61663013,300.35155622)(288.03069263,301.06770205)
\lineto(288.22600513,301.26301455)
\curveto(289.72340096,300.28645205)(290.76506763,299.47264997)(291.35100513,298.8216083)
\curveto(291.93694263,298.17056663)(292.00204679,297.51952497)(291.54631763,296.8684833)
\curveto(291.09058846,296.21744163)(290.76506763,296.11978538)(290.56975513,296.57551455)
\curveto(290.43954679,297.03124372)(290.24423429,297.51952497)(289.98381763,298.0403583)
\curveto(287.44475513,297.8450458)(285.42652596,297.6497333)(283.92913013,297.4544208)
\curveto(282.49683846,297.2591083)(281.48772388,296.96613955)(280.90178638,296.57551455)
\lineto(279.92522388,298.33332705)
\curveto(280.51116138,298.39843122)(281.22730721,298.88671247)(282.07366138,299.7981708)
\curveto(282.92001554,300.70962913)(283.73381763,301.84895205)(284.51506763,303.21613955)
\lineto(282.17131763,303.21613955)
\curveto(281.32496346,303.21613955)(280.54371346,303.1184833)(279.82756763,302.9231708)
\lineto(278.94866138,303.80207705)
\lineto(285.10100513,303.80207705)
\curveto(284.97079679,304.84374372)(284.38485929,305.8528583)(283.34319263,306.8294208)
\lineto(283.44084888,307.12238955)
\curveto(285.06845304,306.47134788)(286.01246346,305.88541038)(286.27288013,305.36457705)
\curveto(286.53329679,304.90884788)(286.40308846,304.38801455)(285.88225513,303.80207705)
\lineto(290.56975513,303.80207705)
\lineto(291.93694263,305.0716083)
\lineto(293.59709888,303.21613955)
\lineto(285.00334888,303.21613955)
\lineto(286.46819263,302.1419208)
\lineto(285.29631763,301.9466083)
\curveto(283.60360929,300.1887958)(282.36663013,298.98436872)(281.58538013,298.33332705)
\curveto(282.95256763,298.33332705)(285.68694263,298.36587913)(289.78850513,298.4309833)
\closepath
\moveto(279.43694263,301.65363955)
\curveto(277.22340096,295.33853538)(276.01897388,291.75780622)(275.82366138,290.91145205)
\curveto(275.69345304,290.13020205)(275.59579679,289.12108747)(275.53069263,287.8841083)
\lineto(275.53069263,285.5403583)
\curveto(275.53069263,285.08462913)(275.14006763,285.01952497)(274.35881763,285.3450458)
\curveto(273.57756763,285.67056663)(273.38225513,286.45181663)(273.77288013,287.6887958)
\curveto(274.22860929,288.99087913)(274.26116138,289.80468122)(273.87053638,290.13020205)
\curveto(273.47991138,290.52082705)(272.79631763,290.84634788)(271.81975513,291.10676455)
\lineto(271.81975513,291.49738955)
\curveto(272.99163013,291.43228538)(273.70777596,291.3997333)(273.96819263,291.3997333)
\curveto(274.22860929,291.46483747)(274.55413013,291.75780622)(274.94475513,292.27863955)
\curveto(275.33538013,292.79947288)(276.70256763,295.98957705)(279.04631763,301.84895205)
\lineto(279.43694263,301.65363955)
\closepath
\moveto(271.81975513,300.28645205)
\curveto(274.03329679,299.24478538)(275.14006763,298.39843122)(275.14006763,297.74738955)
\curveto(275.14006763,297.16145205)(274.97730721,296.6731708)(274.65178638,296.2825458)
\curveto(274.32626554,295.8919208)(274.00074471,296.0872333)(273.67522388,296.8684833)
\curveto(273.41480721,297.71483747)(272.73121346,298.78905622)(271.62444263,300.09113955)
\lineto(271.81975513,300.28645205)
\closepath
\moveto(274.16350513,305.8528583)
\curveto(275.98642179,304.94139997)(277.02808846,304.22525413)(277.28850513,303.7044208)
\curveto(277.54892179,303.18358747)(277.45126554,302.63020205)(276.99553638,302.04426455)
\curveto(276.60491138,301.45832705)(276.24683846,301.62108747)(275.92131763,302.5325458)
\curveto(275.59579679,303.44400413)(274.94475513,304.45311872)(273.96819263,305.55988955)
\lineto(274.16350513,305.8528583)
\closepath
}
}
\end{pspicture}
}
	\hspace{4em}%
	\subcaptionbox{分段中断的执行流\label{fig:ecos_intr_exec}}
	{%LaTeX with PSTricks extensions
%%Creator: 0.91_64bit
%%Please note this file requires PSTricks extensions
\psset{xunit=.5pt,yunit=.5pt,runit=.5pt}
\begin{pspicture}(744.09448819,1052.36220472)
{
\newrgbcolor{curcolor}{0 0 0}
\pscustom[linewidth=0.83516997,linecolor=curcolor]
{
\newpath
\moveto(100.41758,879.99999472)
\lineto(100.41758,600.00000472)
}
}
{
\newrgbcolor{curcolor}{0 0 0}
\pscustom[linewidth=0.83516997,linecolor=curcolor]
{
\newpath
\moveto(160.41758,880.00000472)
\lineto(160.41758,600.00000472)
}
}
{
\newrgbcolor{curcolor}{0 0 0}
\pscustom[linewidth=0.5905543,linecolor=curcolor]
{
\newpath
\moveto(240.29528,810.00000472)
\lineto(240.29528,670.00000472)
}
}
{
\newrgbcolor{curcolor}{0 0 0}
\pscustom[linewidth=0.5905543,linecolor=curcolor]
{
\newpath
\moveto(300.29528,810.00000472)
\lineto(300.29528,670.00000472)
}
}
{
\newrgbcolor{curcolor}{0 0 0}
\pscustom[linewidth=1,linecolor=curcolor]
{
\newpath
\moveto(130.01085,915.71428472)
\lineto(130.01085,757.14285472)
}
}
{
\newrgbcolor{curcolor}{0 0 0}
\pscustom[linestyle=none,fillstyle=solid,fillcolor=curcolor]
{
\newpath
\moveto(130.01085,767.14285472)
\lineto(126.01085,771.14285472)
\lineto(130.01085,757.14285472)
\lineto(134.01085,771.14285472)
\lineto(130.01085,767.14285472)
\closepath
}
}
{
\newrgbcolor{curcolor}{0 0 0}
\pscustom[linewidth=1,linecolor=curcolor]
{
\newpath
\moveto(130.01085,767.14285472)
\lineto(126.01085,771.14285472)
\lineto(130.01085,757.14285472)
\lineto(134.01085,771.14285472)
\lineto(130.01085,767.14285472)
\closepath
}
}
{
\newrgbcolor{curcolor}{0 0 0}
\pscustom[linewidth=1.03499508,linecolor=curcolor]
{
\newpath
\moveto(272.85714,828.76715472)
\lineto(272.85714,658.90308472)
}
}
{
\newrgbcolor{curcolor}{0 0 0}
\pscustom[linestyle=none,fillstyle=solid,fillcolor=curcolor]
{
\newpath
\moveto(272.85714,669.25303551)
\lineto(268.71715968,673.39301583)
\lineto(272.85714,658.90308472)
\lineto(276.99712032,673.39301583)
\lineto(272.85714,669.25303551)
\closepath
}
}
{
\newrgbcolor{curcolor}{0 0 0}
\pscustom[linewidth=1.03499508,linecolor=curcolor]
{
\newpath
\moveto(272.85714,669.25303551)
\lineto(268.71715968,673.39301583)
\lineto(272.85714,658.90308472)
\lineto(276.99712032,673.39301583)
\lineto(272.85714,669.25303551)
\closepath
}
}
{
\newrgbcolor{curcolor}{0 0 0}
\pscustom[linewidth=1.03499508,linecolor=curcolor]
{
\newpath
\moveto(130.18621,740.15960472)
\lineto(130.18621,570.29553472)
}
}
{
\newrgbcolor{curcolor}{0 0 0}
\pscustom[linestyle=none,fillstyle=solid,fillcolor=curcolor]
{
\newpath
\moveto(130.18621,580.64548551)
\lineto(126.04622968,584.78546583)
\lineto(130.18621,570.29553472)
\lineto(134.32619032,584.78546583)
\lineto(130.18621,580.64548551)
\closepath
}
}
{
\newrgbcolor{curcolor}{0 0 0}
\pscustom[linewidth=1.03499508,linecolor=curcolor]
{
\newpath
\moveto(130.18621,580.64548551)
\lineto(126.04622968,584.78546583)
\lineto(130.18621,570.29553472)
\lineto(134.32619032,584.78546583)
\lineto(130.18621,580.64548551)
\closepath
}
}
\end{pspicture}
}
	\caption{不同的中断执行流}
	\label{fig:two_intr_exec}
\end{figure}

\section{程序的正确性}
\label{sec:correctness}
人们通常关心程序的许多性质。我们常说一个程序是否正确,其实涵盖了很多方面。

程序正确性,从狭义上来说,是指一段代码实现的功能是否如预期。这个要求并没
有看上去那么简单。程序不仅在接受各种合法输入之后需要给出预期的输出,在接
受非法输入之后也应该能判断出输入非法,并作出相应的处理措施。

从广义上来说,程序正确性还包含了其在指定环境下运行的正确性。现在的程序很
少是完全孤立运行。大部分程序运行在操作系统中,需要与其他程序共享CPU,内
存,硬盘,网络等资源。那么,程序的正确性就包含了该程序在共享资源的条件下
依然能保持上述狭义的正确性质。对多线程程序的研究就是着眼于多个线程在共享
CPU的条件下能否保证其功能的正确,尤其是当线程间共享的资源不仅仅是CPU,
还有内存中的变量,共同的文件或网络链接等资源时,情况将变得更加复杂。即使
在单线程的运行环境中,程序还会受到中断的干扰\footnote{上述的多线程的环
境下,通常也是有中断存在的。多线程的时间片轮转模式就完全依赖时钟中断。只
不过由于多线程的运行环境已经十分复杂,许多研究就忽略了中断的参与以简化问
题。}除了时钟中断以外的其他中断相比多线程环境,行为的随机性更高,一旦出现
问题,想要完全重现问题场景更为困难,研究起来更加困难。

由于外部环境的参与,一些原本不属于正确性的性质也会对程序的正确运行产生影
响。举个例子,程序的运行时间在大多数情况,本与程序是否正确没有关系,从软
件工程的需求分析角度来说,运行时间,或者说效率只能算作程序的非功能需求。
一段程序运行时间长短似乎不会影响到程序运行的结果。然而,事实并非总是如此。

当程序的运行时间影响到共享资源的占用,而共享资源又对程序的功能正确性造成
影响的时候,程序的运行时间就会成为程序正确性的内涵之一。这在一些十分接近
硬件底层的程序中,体现得尤为明显。举个简单的例子,一个简单的Clinet-Server
(CS)架构,一个客户端和一个服务器。客户端给服务器发送一个请求,经过一段
时间服务器返回结果,客户端继续一段逻辑。在网络编程中,考虑到网络环境的不
稳定性,程序员通常会设置等待超时,即当一定时间后得不到网络另一端的响应,
就认为此次请求失效,接下来就进行例如重发等措施。在这个场景下,服务器接受
请求返回应答的程序的运行时间就会成为影响客户端程序正确性的因素。在极端情
况下,服务器端程序的运行时间过长,客户端一直无法在超时限制之前得到响应,
那么客户端的正常逻辑就永远无法进行下去。

在实时软件系统中,由于实时性要求非常高,大部分程序的运行时间都是其正确性
的关键内容。在大部分实时软件系统中,中断多且与整个软件系统的功能息息相关。
的正确性就显得尤为重要。中断本身的难以预测,中断之间的相互作用又
使得中断的时间性质验证相对困难。

\section{程序的验证技术}
\label{sec:verification}
编程技术发展至今,软件的验证技术已经日趋成熟。根据是否运行程序,我们可以
将验证技术分为动态验证技术和静态验证技术两大类。

\subsection{动态验证技术}
\label{subsec:dynamic}
动态验证即我们通常所谓的测试。根据测试范围的不同,我们可将测试分为三类。
\cite{SWEBOK}
\begin{itemize}
	\item 小范围测试:测试一个函数或一个类(单元测试)
	\item 大范围测试:测试一组类,例如
	\begin{enumerate}[(1)]
		\item 模块测试(测试一个模块)
		\item 集成测试(测试多个模块)
		\item 系统测试(测试整个系统)
	\end{enumerate}
	\item 验收测试:验证软件是否满足需求的正式测试
	\begin{enumerate}[(1)]
		\item 功能测试
		\item 非功能测试(性能测试,压力测试等)
	\end{enumerate}		
\end{itemize}
测试一直时软件工程中十分重要也是最常规的验证的手段。在大多数场合,软件测
试成本低,效果显著。在成熟的软件公司或软件开发团队中,都有专门从事测试的
部门或人员。

\subsection{静态验证技术}
\label{subsec:static}
静态验证指在不运行程序的前提下,对程序进行的验证。有些程序的测试成本太高,
或者测试用例无法覆盖正常运行时的很多情形,即当我们无法测试程序,或者测试
结果局限性太严重时,我们转向静态验证。静态验证的技术发展多年,已经有了很
多相对成熟的方案。例如:
\begin{itemize}
	\item 代码规范检查
	\item 反例检测
	\item 形式化验证
	\begin{enumerate}[(1)]
		\item 模型检测
		\item 定理证明
	\end{enumerate}	
\end{itemize}

在静态分析技术中,形式化验证利用形式化方法对程序性质进行验证,由于其在理
论上的可靠性受到学术界密切关注。常规的测试很难甚至不可能在可以容忍的时间
内覆盖所有的可能情况,但是形式化验证技术则在理论上有给出一个完全覆盖所有
可能的测试结果的可能。这就是形式化验证相比传统动态验证技术的最大优势。

%这一段加引用
在诸多形式化验证技术中,模型检测由于数学要求相对不高,自动化程度较高,工
具化较为容易,在学术界较为流行。定理证明作为另一个完全不同的分支,由于自
动化程度和工具化程度太低,大量工作需要人的手工参与,因此使用者相对较少,
应用数量较少。但是,这模型检测和定理证明在大规模项目上的应用都有限制。前
者受理论和计算机硬件的限制,状态空间爆炸这个问题一直无法根本解决。这也是
诸多学者一直在努力的方向之一。后者由于其理论依据中对高阶逻辑的支持,看似
可以在理论上解决空间爆炸的问题,但是由于目前定理证明主要还是靠人工参与推
理,面对大型项目,人工的推理几乎不可完成。理论上可以进行验证,但是成本过
于高昂,这也是工业界对形式化验证技术一直难以应用的主要原因。相比之下,传
统的测试技术,对项目敏感度没有形式化技术这么高,反而容易取得不错的结果。

然而,在一些领域,程序的正确性涉及到人身财产安全甚至国家安全,不完备的测
试几乎不可能满足正确性的验证要求。形式化验证看似很美好,但随着验证规模的
增长,成本也成指数级增长,这时一个折中的做法对验证对象进行一定的抽象,然
后再对抽象出的模型进行验证。通常,抽象之后的模型相比原程序在验证规模上会
缩小很多。如果能保证抽象过程对于待验证性质的保持,即针对待验证性质,程序
和抽象模型之间存在一个完好的精化关系,就可以对模型进行形式化验证。

\section{研究课题}
\label{sec:subject}
我的课题是基于Uppaal的中断时间性质分析和验证。课题来源于一个针对某嵌入
式平台的实时软件系统的验证项目。我希望利用时间自动机的理论分析中断的时间
性质,进而对某些性质给出肯定或否定的验证结论。

\subsection{Uppaal简介}
\label{subsec:Uppaal_intro}
%这一段加引用
UPPAAL是一个进行建模,仿真和验证实时系统的集成工具环境,由丹麦奥尔堡大
学的计算机科学基础研究中心(Basic Research in Computer Science, BRICS)和瑞典乌普萨拉大学信息技术系联合开发。它适合那些可以被建模为具有
有限的控制结构和实数值时钟,通过信道或共享变量通信的非确定性过程集合的系
统。典型的应用领域包括实时控制器和特定的实时性至关重要的通信协议。

\subsection{预期成果}
\label{subsec:expectation}
本课题预期有以下成果:
\begin{itemize}
	\item 在Uppaal中针对各类中断的时间自动机模型
	\item 针对某嵌入式平台的实时软件系统的中断的时间性质的验证
\end{itemize}

\section{论文结构}
\label{sec:structure}
本文第一章简要介绍我研究的课题以及相关背景。在第二章,本文将介绍一些有关
中断分析验证以及时间自动机的相关工作。在第三章,本文将描述对中断行为的分
析,中断模型的建立和利用时间自动机的分析过程。在第四章,本文将描述将本文
的理论模型应用某嵌入式平台的实施软件系统进行验证的实验过程以及实验结果。
在第五章,本文将对该课题工作进行总结。


