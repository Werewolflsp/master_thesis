%%% Local Variables:
%%% mode: latex
%%% TeX-master: t
%%% End:

\chapter{中断分析的现状}
本章介绍中断的概念和现在学术界对中断的研究。就是文献综述。

\section{什么是中断}
中断(Interrupt)是指处理器接收到特殊信号,提示某个事件发生,应当采取对应措施的情况。发出这样的信号的行为被称作中断请求(Interrupt Request,IRQ)。
这个信号可以是来自外围硬件的异步信号,也可以是由运行在处理器上的软件发出同步信号。前者被称为硬件中断(Hardware Interrupt),后者则被称为软件中断(Software Interrupt)。

处理器在接收到中断信号以后,通常会保存当前的执行状态,该执行状态被称为上下文(Context)。上下文的内容视各个硬件平台的不同而有所区别,但是大多包含程序计数器
和程序状态字以及部分通用计算寄存器。之后,处理器会根据中断向量表的指示跳转到指定代码片段,即当前中断的处理程序。在执行完中断处理程序之后,处理器会重新加
载之前保存的上下文,继续执行之前被打断的程序。进入中断和退出中断都包含了将保存当前上下文,载入新的上下文的操作。此类操作被称作为上下文切换(Context Switch)。\footnote{严格意义上的上下文切换并不要求保存一个完好的上下文,载入另一个完好的上下文,实际情况中经常发生载入的上下文只有部分内容有意义的情况。
或者对原有上下文并未做出妥善保存即另行载入,相当于舍弃了当前上下文。这在处理器响应中断时尤其多见。}

人们引入中断是为了提高计算机系统的性能。如果没有中断,处理器在接受外部硬件通信时只能采取轮询方式。例如,处理器向某硬件发出某指令并需求其回复时,只能采取
繁忙等待(Busy-waiting)模式,这样就会浪费许多处理器周期。即使在软件层面运用多线程技术进行优化,此类轮询操作依然是性能的损失。在引入中断之后,处理器就可以
专注于当前任务,并且可以在需要时候与外部硬件通信,从而大幅提高运行效率。\footnote{处理器检查中断的原理其实也类似于一个轮询操作。每个周期,处理器}

\subsection{中断与线程}
\subsection{嵌入式系统中的中断}

\section{中断的研究现状}