%%% Local Variables:
%%% mode: latex
%%% TeX-master: t
%%% End:

\chapter{引言}
\label{cha:intro}
%引出问题。

\section{选题背景}
\label{sec:background}

\subsection{中断}
\label{subsec:intr}
中断(Interrupt)是指处理器接收到特殊信号,提示某个事件发生,应当采取
对应措施的情况。发出这样的信号的行为被称作中断请求(Interrupt Request,
IRQ)。这个信号可以是来自外围硬件的异步信号,也可以是由运行在处理器上的
软件发出同步信号。前者被称为硬件中断(Hardware Interrupt),后者则被
称为软件中断(Software Interrupt,常简称为软中断)。

处理器在接收到中断信号以后,通常会保存当前的执行状态,该执行状态被称为上
下文(Context)。上下文的内容视各个硬件平台的不同而有所区别,但是大多包
含程序计数器(Program Counter,PC)和程序状态字(Program Status Word, 
PSW)以及部分通用计算寄存器。之后,处理器会根据中断向量表的指示跳转到指定
代码片段,即当前中断的处理程序。在执行完中断处理程序之后,处理器会重新加
载之前保存的上下文,继续执行之前被打断的程序。进入中断和退出中断都包含了
将保存当前上下文,载入新的上下文的操作。此类操作被称作为上下文切换(Context 
Switch)。\footnote{严格意义上的上下文切换并不要求保存一个完好的上下文,
载入另一个完好的上下文,实际情况中经常发生载入的上下文只有部分内容有意义
的情况。或者对原有上下文并未做出妥善保存即另行载入,相当于舍弃了当前上下
文。这在处理器响应中断时尤其多见。}

人们引入中断是为了提高计算机系统的性能。如果没有中断,处理器在接受外部硬
件通信时只能采取轮询方式。例如,处理器向某硬件发出某指令并需求其回复时,
只能采取繁忙等待(Busy-waiting)模式,这样就会浪费许多处理器周期。即使
在软件层面运用多线程技术进行优化,此类轮询操作依然是性能的损失。在引入中
断之后,处理器就可以专注于当前任务,并且可以在需要时候与外部硬件通信,从
而大幅提高运行效率。\footnote{处理器检查中断的原理其实也类似于一个轮询
操作。每个周期,处理器会检查特定的寄存器的某些位来判断是否有中断需要处理。
只不过这个处理相比于外部硬件通信快速得多。}后来被用于CPU外部与内部紧急事
件的处理、机器故障的处理、时间控制等多个方面,并产生通过软件方式进入中断
处理(软中断)的概念。

\subsection{程序的正确性}
\label{subsec:correctness}
人们通常关心程序的许多性质。我们常说一个程序是否正确,其实涵盖了很多方面。

程序正确性,从狭义上来说,是指一段代码实现的功能是否如预期。这个要求并没
有看上去那么简单。程序不仅在接受各种合法输入之后需要给出预期的输出,在接
受非法输入之后也应该能判断出输入非法,并作出相应的处理措施。

从广义上来说,程序正确性还包含了其在指定环境下运行的正确性。现在的程序很
少是完全孤立运行。大部分程序运行在操作系统中,需要与其他程序共享CPU,内
存,硬盘,网络等资源。那么,程序的正确性就包含了该程序在共享资源的条件下
依然能保持上述狭义的正确性质。对多线程程序的研究就是着眼于多个线程在共享
CPU的条件下能否保证其功能的正确,尤其是当线程间共享的资源不仅仅是CPU,
还有内存中的变量,共同的文件或网络链接等资源时,情况将变得更加复杂。即使
在单线程的运行环境中,程序还会受到中断的干扰\footnote{上述的多线程的环
境下,通常也是有中断存在的。多线程的时间片轮转模式就完全依赖时钟中断。只
不过由于多线程的运行环境已经十分复杂,许多研究就忽略了中断的参与以简化问
题。}除了时钟中断以外的其他中断相比多线程环境,行为的随机性更高,一旦出现
问题,想要完全重现问题场景更为困难,研究起来更加困难。

由于外部环境的参与,一些原本不属于正确性的性质也会对程序的正确运行产生影
响。举个例子,程序的运行时间在大多数情况,本与程序是否正确没有关系,从软
件工程的需求分析角度来说,运行时间,或者说效率只能算作程序的非功能需求。
一段程序运行时间长短似乎不会影响到程序运行的结果。然而,事实并非总是如此。

当程序的运行时间影响到共享资源的占用,而共享资源又对程序的功能正确性造成
影响的时候,程序的运行时间就会成为程序正确性的内涵之一。这在一些十分接近
硬件底层的程序中,体现得尤为明显。举个简单的例子,一个简单的Clinet-Server
(CS)架构,一个客户端和一个服务器。客户端给服务器发送一个请求,经过一段
时间服务器返回结果,客户端继续一段逻辑。在网络编程中,考虑到网络环境的不
稳定性,程序员通常会设置等待超时,即当一定时间后得不到网络另一端的响应,
就认为此次请求失效,接下来就进行例如重发等措施。在这个场景下,服务器接受
请求返回应答的程序的运行时间就会成为影响客户端程序正确性的因素。在极端情
况下,服务器端程序的运行时间过长,客户端一直无法在超时限制之前得到响应,
那么客户端的正常逻辑就永远无法进行下去。

\subsection{嵌入式程序的中断实时性验证}
\label{subsec:embedded_intr}
人们现在已经越来越重视程序的正确性。尤其是在与重大科技或军事项目中,与硬
件结合越紧密,程序正确性就越重要。工程人员都会极力避免类似Ariane 5火箭
爆炸事件的再次发生。如今的嵌入式程序运行在世界上所有的飞机、汽车、火车甚
至我们身边的手机、冰箱、电视上。其安全性质至关重要。然而,在与硬件紧密结
合的嵌入式程序中,单纯的软件验证技术无法检测到与硬件相关的可能错误。人们
看待程序正确性这个问题的角度不应该局限在软件或者硬件。

另一方面,并非所有的程序缺陷都存在代码中。在嵌入式领域,实时性是一项重要
的性质。人们在组建嵌入式系统的时候很多时候十分依赖嵌入式平台的实时性。但
是,不恰当的设置或者程序代码会破坏设计者心中假设的实时性。一旦这样的失误
发生在关键的地方,也许就会酿成一场灾难。

我这项工作来源于一个某航空动力控制系统中的中断实时性的验证项目。中断作为
软硬件结合一个典型案例,对嵌入式领域的程序正确性有很重要的影响。许多编程
人员在编写中断时通常会假设中断的实时性是符合预期的,很大程度上是因为他们
并不能左右中断的基本机制,然而这并不能保证他们的假设就是正确的。因此我们
希望能够在理解中断机制的基础上,分析和验证中断的实时性。此项工作可以作为
传统嵌入式程序验证的重要补充。

\section{研究现状}
\label{sec:study}

目前,学术界对中断的实时性分析和验证的研究并不充分。但是已经比一二十年前
进步了许多。目前众多研究依然是立足于代码,在做中断的分析的时候更多的是从
数据的角度来看待中断的性质,诸如数据竞争的判定等。从时间性质上来说,从代
码出发会导致执行路径的爆炸式增长,从而会使分析规模急剧扩大以致自动分析几
乎是不可能的任务。

另一方面,大部分研究对中断的定义过于狭窄。事实上有许多的中断不能以简单的
抢占\pozhehao 执行\pozhehao 退出来看待。这也是本文试图做出一点贡献的地
方。

更多相关研究的工作请参阅第~\ref{cha:related_work} 章。

\section{研究内容}
\label{sec:subject}
本文的工作是基于时间自动机的中断实时性的分析和验证。首先,我需要深入理解
现有的中断极其实现机制。然后,我需要构造出中断的时间自动机模型。然后将这
些时间自动机模型应用到一个具体的案例中,并利用自动模型检测工具验证其实时
性约束是否被满足。

\subsection{预期成果}
\label{subsec:expectation}
本课题预期有以下成果:
\begin{itemize}
	\item 针对各类中断的时间自动机模型
	\item 针对某嵌入式平台的实时软件系统的中断模型和对其实时性的验证
\end{itemize}

\section{论文结构}
\label{sec:structure}
本文第~\ref{cha:intro} 章简要介绍我研究的内容以及相关背景。在
第~\ref{cha:related_work} 章,本文将介绍一些有关中断分析验证以及时间自
动机的相关工作。在第~\ref{cha:intr} 章,本文将描述对中断行为的分析,中
断模型的建立和利用时间自动机的分析过程。在第~\ref{cha:case} 章,本文将
描述将本文的理论模型应用某嵌入式平台的实施软件系统进行验证的实验过程以及
实验结果。在第~\ref{cha:sum} 章,本文将对该课题工作进行总结和未来工作的
展望。


