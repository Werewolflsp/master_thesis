%%% Local Variables:
%%% mode: latex
%%% TeX-master: t
%%% End:

\chapter{基于Uppaal的中断模型}
\label{cha:intr}

\section{Uppaal中的模型组成}
\label{sec:model_combine}	
一套Uppaal模型由以下三部分组成。
\begin{itemize}
	\item \emph{声明}:整个模型系统中共有的声明,可以是变量或函数。在整个
	系统中都可以访问。
	\item \emph{自动机模板}:各类自动机的通用模板,一个模型系统可以有多个
	模板,一个模板在系统中可以对应多个实例。
		\begin{enumerate}[(1)]
			\item \emph{声明}:模板内部的变量或函数,只有本模板的实例可
			以访问。
			\item \emph{位置}:时间自动机的位置,每个位置可以有初始(
			initial),紧急(urgent),关键(committed)。关键位置与紧
			急位置上,模型中的时钟都停止。不同的是,当有自动机在关键位置时,
			在下一个状态迁移必须从某一个关键位置发出。
			\item \emph{变迁}:位置到位置的迁移。变迁包含选择(select)
			、条件(guard)、同步(sync)、更新(update)四个属性。其中,
			同步和更新是同时发生的。
		\end{enumerate}	
	\item \emph{模型声明}:定义组成系统的模板实例。
\end{itemize}

\section{中断基本模型}
\label{sec:basic}
通常,在我们接触到的非实时性电脑环境中,中断的行为就是符合一个中断的基本模型。
其行为模式就是简单的抢占当前线程,在中断程序执行结束之后恢复上下文继续执行被抢
占的线程。

\begin{figure}[H]
	\centering
	%LaTeX with PSTricks extensions
%%Creator: inkscape 0.91
%%Please note this file requires PSTricks extensions
\psset{xunit=.5pt,yunit=.5pt,runit=.5pt}
\begin{pspicture}(438.14290264,242.42857685)
{
\newrgbcolor{curcolor}{0 0 0}
\pscustom[linewidth=0.57914072,linecolor=curcolor]
{
\newpath
\moveto(300.71042076,199.7857195)
\curveto(300.71042076,223.17679387)(263.2000076,242.13900624)(216.92856804,242.13900624)
\curveto(170.65712849,242.13900624)(133.14671532,223.17679387)(133.14671532,199.7857195)
\curveto(133.14671532,176.39464514)(170.65712849,157.43243276)(216.92856804,157.43243276)
\curveto(263.2000076,157.43243276)(300.71042076,176.39464514)(300.71042076,199.7857195)
\closepath
}
}
{
\newrgbcolor{curcolor}{0 0 0}
\pscustom[linestyle=none,fillstyle=solid,fillcolor=curcolor]
{
\newpath
\moveto(217.19642083,184.4732195)
\curveto(216.46725416,183.2232195)(215.26933749,182.5982195)(213.60267083,182.5982195)
\lineto(210.32142083,182.5982195)
\curveto(208.23808749,182.5982195)(207.19642083,183.5357195)(207.19642083,185.4107195)
\lineto(207.19642083,204.4732195)
\lineto(205.16517083,204.4732195)
\curveto(205.26933749,198.74405283)(204.59225416,194.10863617)(203.13392083,190.5669695)
\curveto(201.77975416,187.1294695)(199.07142083,183.90030283)(195.00892083,180.8794695)
\lineto(194.54017083,181.3482195)
\curveto(197.76933749,184.1607195)(200.00892083,187.2857195)(201.25892083,190.7232195)
\curveto(202.61308749,194.26488617)(203.23808749,198.8482195)(203.13392083,204.4732195)
\lineto(201.41517083,204.4732195)
\lineto(200.00892083,204.1607195)
\lineto(198.75892083,205.4107195)
\lineto(203.13392083,205.4107195)
\curveto(203.13392083,212.18155283)(203.08183749,216.08780283)(202.97767083,217.1294695)
\lineto(206.25892083,215.5669695)
\lineto(205.16517083,214.4732195)
\lineto(205.16517083,205.4107195)
\lineto(212.50892083,205.4107195)
\lineto(214.38392083,207.2857195)
\lineto(217.04017083,204.4732195)
\lineto(209.22767083,204.4732195)
\lineto(209.22767083,186.1919695)
\curveto(209.22767083,184.9419695)(209.85267083,184.3169695)(211.10267083,184.3169695)
\lineto(212.35267083,184.3169695)
\curveto(213.49850416,184.3169695)(214.12350416,184.83780283)(214.22767083,185.8794695)
\curveto(214.33183749,187.02530283)(214.38392083,188.6919695)(214.38392083,190.8794695)
\lineto(215.16517083,190.8794695)
\lineto(215.47767083,187.1294695)
\curveto(215.68600416,185.8794695)(216.25892083,184.99405283)(217.19642083,184.4732195)
\closepath
\moveto(208.29017083,214.7857195)
\curveto(210.16517083,213.8482195)(211.46725416,213.01488617)(212.19642083,212.2857195)
\curveto(212.92558749,211.6607195)(213.18600416,210.93155283)(212.97767083,210.0982195)
\curveto(212.76933749,209.36905283)(212.40475416,208.79613617)(211.88392083,208.3794695)
\curveto(211.46725416,208.0669695)(211.05058749,208.58780283)(210.63392083,209.9419695)
\curveto(210.21725416,211.40030283)(209.33183749,212.85863617)(207.97767083,214.3169695)
\lineto(208.29017083,214.7857195)
\closepath
\moveto(186.88392083,204.4732195)
\lineto(186.88392083,198.8482195)
\lineto(195.47767083,198.8482195)
\lineto(195.47767083,204.4732195)
\lineto(186.88392083,204.4732195)
\closepath
\moveto(189.07142083,217.1294695)
\curveto(190.42558749,216.40030283)(191.46725416,215.67113617)(192.19642083,214.9419695)
\curveto(192.92558749,214.21280283)(193.18600416,213.5357195)(192.97767083,212.9107195)
\curveto(192.87350416,212.2857195)(192.56100416,211.71280283)(192.04017083,211.1919695)
\curveto(191.51933749,210.77530283)(191.05058749,211.24405283)(190.63392083,212.5982195)
\curveto(190.32142083,214.05655283)(189.64433749,215.46280283)(188.60267083,216.8169695)
\lineto(189.07142083,217.1294695)
\closepath
\moveto(181.72767083,210.4107195)
\lineto(196.72767083,210.4107195)
\lineto(198.44642083,212.1294695)
\lineto(201.10267083,209.4732195)
\lineto(186.88392083,209.4732195)
\curveto(185.32142083,209.4732195)(184.07142083,209.3169695)(183.13392083,209.0044695)
\lineto(181.72767083,210.4107195)
\closepath
\moveto(184.69642083,196.1919695)
\curveto(184.80058749,197.85863617)(184.85267083,199.57738617)(184.85267083,201.3482195)
\curveto(184.85267083,203.11905283)(184.80058749,204.88988617)(184.69642083,206.6607195)
\lineto(186.88392083,205.4107195)
\lineto(195.32142083,205.4107195)
\lineto(196.57142083,206.8169695)
\lineto(198.75892083,204.7857195)
\lineto(197.50892083,203.8482195)
\curveto(197.50892083,201.0357195)(197.56100416,198.79613617)(197.66517083,197.1294695)
\lineto(195.47767083,196.1919695)
\lineto(195.47767083,197.9107195)
\lineto(192.50892083,197.9107195)
\lineto(192.50892083,185.5669695)
\curveto(192.50892083,184.52530283)(192.24850416,183.74405283)(191.72767083,183.2232195)
\curveto(191.31100416,182.70238617)(190.58183749,182.23363617)(189.54017083,181.8169695)
\curveto(189.43600416,183.0669695)(188.08183749,184.10863617)(185.47767083,184.9419695)
\lineto(185.47767083,185.5669695)
\curveto(187.87350416,185.2544695)(189.33183749,185.0982195)(189.85267083,185.0982195)
\curveto(190.37350416,185.20238617)(190.63392083,185.5669695)(190.63392083,186.1919695)
\lineto(190.63392083,197.9107195)
\lineto(186.88392083,197.9107195)
\lineto(186.88392083,196.9732195)
\lineto(184.69642083,196.1919695)
\closepath
\moveto(188.75892083,193.0669695)
\lineto(187.35267083,192.2857195)
\curveto(185.26933749,189.05655283)(183.29017083,186.45238617)(181.41517083,184.4732195)
\lineto(180.94642083,184.7857195)
\curveto(182.09225416,186.55655283)(183.08183749,188.32738617)(183.91517083,190.0982195)
\curveto(184.85267083,191.86905283)(185.52975416,193.48363617)(185.94642083,194.9419695)
\lineto(188.75892083,193.0669695)
\closepath
\moveto(194.22767083,194.0044695)
\lineto(194.54017083,194.4732195)
\curveto(196.10267083,193.5357195)(197.19642083,192.7544695)(197.82142083,192.1294695)
\curveto(198.44642083,191.5044695)(198.75892083,190.82738617)(198.75892083,190.0982195)
\curveto(198.75892083,189.4732195)(198.44642083,188.79613617)(197.82142083,188.0669695)
\curveto(197.30058749,187.4419695)(196.88392083,187.85863617)(196.57142083,189.3169695)
\curveto(196.25892083,190.77530283)(195.47767083,192.33780283)(194.22767083,194.0044695)
\closepath
}
}
{
\newrgbcolor{curcolor}{0 0 0}
\pscustom[linestyle=none,fillstyle=solid,fillcolor=curcolor]
{
\newpath
\moveto(243.44642083,209.6294695)
\lineto(243.44642083,203.3794695)
\lineto(244.54017083,203.3794695)
\curveto(245.99850416,205.04613617)(247.56100416,207.1294695)(249.22767083,209.6294695)
\lineto(243.44642083,209.6294695)
\closepath
\moveto(240.16517083,196.0357195)
\lineto(240.16517083,191.3482195)
\lineto(250.47767083,191.3482195)
\lineto(250.47767083,196.0357195)
\lineto(240.16517083,196.0357195)
\closepath
\moveto(240.16517083,190.4107195)
\lineto(240.16517083,184.9419695)
\lineto(250.47767083,184.9419695)
\lineto(250.47767083,190.4107195)
\lineto(240.16517083,190.4107195)
\closepath
\moveto(237.97767083,180.8794695)
\curveto(238.08183749,183.48363617)(238.13392083,188.0669695)(238.13392083,194.6294695)
\curveto(235.21725416,192.54613617)(232.97767083,191.13988617)(231.41517083,190.4107195)
\lineto(231.10267083,190.8794695)
\curveto(234.01933749,192.96280283)(236.36308749,194.88988617)(238.13392083,196.6607195)
\curveto(238.13392083,197.18155283)(238.08183749,197.85863617)(237.97767083,198.6919695)
\lineto(239.22767083,197.7544695)
\curveto(240.89433749,199.3169695)(242.45683749,200.8794695)(243.91517083,202.4419695)
\lineto(237.97767083,202.4419695)
\curveto(236.62350416,202.4419695)(235.37350416,202.2857195)(234.22767083,201.9732195)
\lineto(232.82142083,203.3794695)
\lineto(241.41517083,203.3794695)
\lineto(241.41517083,209.6294695)
\lineto(239.85267083,209.6294695)
\curveto(238.49850416,209.6294695)(237.24850416,209.4732195)(236.10267083,209.1607195)
\lineto(234.69642083,210.5669695)
\lineto(241.41517083,210.5669695)
\curveto(241.41517083,214.0044695)(241.36308749,216.3482195)(241.25892083,217.5982195)
\lineto(244.69642083,216.0357195)
\lineto(243.44642083,214.9419695)
\lineto(243.44642083,210.5669695)
\lineto(245.16517083,210.5669695)
\lineto(247.19642083,212.2857195)
\lineto(249.38392083,210.0982195)
\curveto(250.21725416,211.3482195)(251.05058749,212.9107195)(251.88392083,214.7857195)
\lineto(254.85267083,212.5982195)
\curveto(254.01933749,212.38988617)(252.92558749,211.3482195)(251.57142083,209.4732195)
\curveto(250.32142083,207.5982195)(248.81100416,205.5669695)(247.04017083,203.3794695)
\lineto(252.19642083,203.3794695)
\lineto(254.38392083,205.4107195)
\lineto(257.04017083,202.4419695)
\lineto(246.41517083,202.4419695)
\curveto(244.64433749,200.5669695)(242.82142083,198.74405283)(240.94642083,196.9732195)
\lineto(250.16517083,196.9732195)
\lineto(251.25892083,198.5357195)
\lineto(253.75892083,196.5044695)
\lineto(252.50892083,195.5669695)
\lineto(252.50892083,185.5669695)
\curveto(252.50892083,184.21280283)(252.56100416,183.01488617)(252.66517083,181.9732195)
\lineto(250.47767083,181.0357195)
\lineto(250.47767083,184.0044695)
\lineto(240.16517083,184.0044695)
\lineto(240.16517083,181.8169695)
\lineto(237.97767083,180.8794695)
\closepath
\moveto(221.10267083,186.9732195)
\curveto(222.04017083,187.07738617)(223.65475416,187.33780283)(225.94642083,187.7544695)
\curveto(228.34225416,188.17113617)(231.36308749,188.74405283)(235.00892083,189.4732195)
\lineto(235.16517083,188.8482195)
\curveto(232.24850416,188.01488617)(229.69642083,187.23363617)(227.50892083,186.5044695)
\curveto(225.32142083,185.77530283)(223.70683749,184.9419695)(222.66517083,184.0044695)
\lineto(221.10267083,186.9732195)
\closepath
\moveto(228.44642083,217.1294695)
\lineto(231.57142083,215.2544695)
\curveto(230.73808749,214.83780283)(229.64433749,213.58780283)(228.29017083,211.5044695)
\curveto(226.93600416,209.52530283)(225.37350416,207.2857195)(223.60267083,204.7857195)
\lineto(230.79017083,204.9419695)
\curveto(231.62350416,206.29613617)(232.35267083,207.7544695)(232.97767083,209.3169695)
\lineto(235.47767083,207.2857195)
\curveto(234.33183749,206.45238617)(232.97767083,204.99405283)(231.41517083,202.9107195)
\curveto(229.95683749,200.82738617)(227.76933749,198.2232195)(224.85267083,195.0982195)
\curveto(227.76933749,195.51488617)(231.05058749,196.08780283)(234.69642083,196.8169695)
\lineto(234.85267083,196.1919695)
\curveto(232.14433749,195.35863617)(230.00892083,194.68155283)(228.44642083,194.1607195)
\curveto(226.88392083,193.63988617)(225.52975416,192.96280283)(224.38392083,192.1294695)
\lineto(222.35267083,195.4107195)
\curveto(223.18600416,195.7232195)(224.17558749,196.45238617)(225.32142083,197.5982195)
\curveto(226.46725416,198.74405283)(228.13392083,200.93155283)(230.32142083,204.1607195)
\curveto(228.44642083,203.95238617)(226.98808749,203.6919695)(225.94642083,203.3794695)
\curveto(224.90475416,203.17113617)(223.86308749,202.7544695)(222.82142083,202.1294695)
\lineto(221.10267083,204.9419695)
\curveto(222.04017083,205.04613617)(223.23808749,206.29613617)(224.69642083,208.6919695)
\curveto(226.15475416,211.08780283)(227.40475416,213.90030283)(228.44642083,217.1294695)
\closepath
}
}
{
\newrgbcolor{curcolor}{0 0 0}
\pscustom[linestyle=none,fillstyle=solid,fillcolor=curcolor]
{
\newpath
\moveto(61.52679148,62.75447382)
\lineto(75.12054148,62.75447382)
\lineto(77.30804148,64.78572382)
\lineto(80.12054148,61.81697382)
\lineto(65.74554148,61.81697382)
\curveto(65.01637481,61.81697382)(64.07887481,61.66072382)(62.93304148,61.34822382)
\lineto(61.52679148,62.75447382)
\closepath
\moveto(60.58929148,42.44197382)
\curveto(61.63095814,42.65030715)(62.62054148,43.32739048)(63.55804148,44.47322382)
\curveto(64.49554148,45.61905715)(65.48512481,47.07739048)(66.52679148,48.84822382)
\curveto(67.56845814,50.61905715)(68.29762481,52.23364048)(68.71429148,53.69197382)
\lineto(63.08929148,53.69197382)
\curveto(62.25595814,53.69197382)(61.26637481,53.53572382)(60.12054148,53.22322382)
\lineto(58.71429148,54.62947382)
\lineto(78.08929148,54.62947382)
\lineto(80.43304148,56.97322382)
\lineto(83.40179148,53.69197382)
\lineto(69.18304148,53.69197382)
\lineto(71.99554148,51.66072382)
\curveto(71.16220814,51.66072382)(69.91220814,50.51489048)(68.24554148,48.22322382)
\curveto(66.68304148,46.03572382)(64.96429148,44.00447382)(63.08929148,42.12947382)
\lineto(76.99554148,42.75447382)
\curveto(76.26637481,44.31697382)(75.12054148,45.98364048)(73.55804148,47.75447382)
\lineto(74.02679148,48.06697382)
\curveto(76.21429148,46.71280715)(77.88095814,45.41072382)(79.02679148,44.16072382)
\curveto(80.27679148,43.01489048)(80.79762481,41.92114048)(80.58929148,40.87947382)
\curveto(80.48512481,39.94197382)(80.17262481,39.26489048)(79.65179148,38.84822382)
\curveto(79.13095814,38.43155715)(78.71429148,38.63989048)(78.40179148,39.47322382)
\curveto(78.08929148,40.41072382)(77.77679148,41.24405715)(77.46429148,41.97322382)
\curveto(72.04762481,41.45239048)(68.40179148,40.98364048)(66.52679148,40.56697382)
\curveto(64.75595814,40.15030715)(63.40179148,39.62947382)(62.46429148,39.00447382)
\lineto(60.58929148,42.44197382)
\closepath
\moveto(51.37054148,65.56697382)
\lineto(51.83929148,65.87947382)
\curveto(53.92262481,64.83780715)(55.32887481,63.95239048)(56.05804148,63.22322382)
\curveto(56.78720814,62.49405715)(57.09970814,61.71280715)(56.99554148,60.87947382)
\curveto(56.89137481,60.15030715)(56.47470814,59.47322382)(55.74554148,58.84822382)
\curveto(55.12054148,58.32739048)(54.65179148,58.79614048)(54.33929148,60.25447382)
\curveto(54.02679148,61.71280715)(53.03720814,63.48364048)(51.37054148,65.56697382)
\closepath
\moveto(56.21429148,39.94197382)
\curveto(57.15179148,38.90030715)(58.45387481,37.91072382)(60.12054148,36.97322382)
\curveto(61.78720814,36.03572382)(64.18304148,35.46280715)(67.30804148,35.25447382)
\curveto(70.53720814,35.15030715)(73.55804148,35.15030715)(76.37054148,35.25447382)
\curveto(79.28720814,35.35864048)(82.04762481,35.61905715)(84.65179148,36.03572382)
\lineto(84.65179148,35.41072382)
\curveto(82.46429148,34.78572382)(81.37054148,33.90030715)(81.37054148,32.75447382)
\curveto(76.68304148,32.75447382)(72.77679148,32.80655715)(69.65179148,32.91072382)
\curveto(66.63095814,33.01489048)(64.13095814,33.48364048)(62.15179148,34.31697382)
\curveto(60.27679148,35.04614048)(58.76637481,35.93155715)(57.62054148,36.97322382)
\curveto(56.47470814,38.11905715)(55.64137481,38.74405715)(55.12054148,38.84822382)
\curveto(54.59970814,38.95239048)(53.71429148,38.37947382)(52.46429148,37.12947382)
\curveto(51.31845814,35.87947382)(50.48512481,34.88989048)(49.96429148,34.16072382)
\lineto(47.77679148,36.34822382)
\curveto(49.44345814,37.70239048)(51.57887481,38.90030715)(54.18304148,39.94197382)
\lineto(54.18304148,52.59822382)
\lineto(53.08929148,52.59822382)
\curveto(51.73512481,52.59822382)(50.48512481,52.44197382)(49.33929148,52.12947382)
\lineto(47.93304148,53.53572382)
\lineto(53.87054148,53.53572382)
\lineto(54.96429148,55.25447382)
\lineto(57.62054148,53.22322382)
\lineto(56.21429148,52.12947382)
\lineto(56.21429148,39.94197382)
\closepath
}
}
{
\newrgbcolor{curcolor}{0 0 0}
\pscustom[linestyle=none,fillstyle=solid,fillcolor=curcolor]
{
\newpath
\moveto(103.87054148,63.22322382)
\lineto(117.15179148,63.22322382)
\lineto(119.33929148,65.25447382)
\lineto(121.99554148,62.28572382)
\lineto(109.02679148,62.28572382)
\curveto(107.67262481,62.28572382)(106.42262481,62.12947382)(105.27679148,61.81697382)
\lineto(103.87054148,63.22322382)
\closepath
\moveto(105.27679148,35.41072382)
\curveto(108.08929148,35.20239048)(109.96429148,35.09822382)(110.90179148,35.09822382)
\curveto(111.83929148,35.09822382)(112.30804148,35.72322382)(112.30804148,36.97322382)
\lineto(112.30804148,52.91072382)
\lineto(106.37054148,52.91072382)
\curveto(105.01637481,52.91072382)(103.76637481,52.75447382)(102.62054148,52.44197382)
\lineto(101.21429148,53.84822382)
\lineto(119.18304148,53.84822382)
\lineto(121.52679148,56.03572382)
\lineto(124.18304148,52.91072382)
\lineto(114.49554148,52.91072382)
\lineto(114.49554148,36.19197382)
\curveto(114.59970814,33.37947382)(113.29762481,31.71280715)(110.58929148,31.19197382)
\curveto(110.58929148,32.65030715)(108.81845814,33.79614048)(105.27679148,34.62947382)
\lineto(105.27679148,35.41072382)
\closepath
\moveto(97.62054148,67.12947382)
\lineto(100.43304148,64.94197382)
\curveto(99.70387481,64.73364048)(98.45387481,63.58780715)(96.68304148,61.50447382)
\curveto(94.91220814,59.52530715)(92.67262481,57.54614048)(89.96429148,55.56697382)
\lineto(89.65179148,56.03572382)
\curveto(91.42262481,57.70239048)(93.03720814,59.57739048)(94.49554148,61.66072382)
\curveto(96.05804148,63.84822382)(97.09970814,65.67114048)(97.62054148,67.12947382)
\closepath
\moveto(97.93304148,49.31697382)
\lineto(97.93304148,32.28572382)
\lineto(95.58929148,30.87947382)
\curveto(95.69345814,33.17114048)(95.74554148,39.52530715)(95.74554148,49.94197382)
\curveto(93.34970814,47.44197382)(91.00595814,45.41072382)(88.71429148,43.84822382)
\lineto(88.40179148,44.31697382)
\curveto(90.58929148,46.50447382)(92.67262481,48.95239048)(94.65179148,51.66072382)
\curveto(96.73512481,54.36905715)(98.29762481,56.97322382)(99.33929148,59.47322382)
\lineto(101.99554148,57.12947382)
\lineto(100.58929148,56.34822382)
\curveto(99.23512481,54.36905715)(98.08929148,52.80655715)(97.15179148,51.66072382)
\lineto(98.87054148,50.25447382)
\lineto(97.93304148,49.31697382)
\closepath
}
}
{
\newrgbcolor{curcolor}{0 0 0}
\pscustom[linewidth=0.57914072,linecolor=curcolor]
{
\newpath
\moveto(167.85329141,48.35713495)
\curveto(167.85329141,71.74820931)(130.34287825,90.71042169)(84.07143869,90.71042169)
\curveto(37.79999914,90.71042169)(0.28958597,71.74820931)(0.28958597,48.35713495)
\curveto(0.28958597,24.96606059)(37.79999914,6.00384821)(84.07143869,6.00384821)
\curveto(130.34287825,6.00384821)(167.85329141,24.96606059)(167.85329141,48.35713495)
\closepath
}
}
{
\newrgbcolor{curcolor}{0 0 0}
\pscustom[linestyle=none,fillstyle=solid,fillcolor=curcolor]
{
\newpath
\moveto(325.09818896,48.13394253)
\curveto(326.86902229,46.46727587)(328.17110562,44.90477587)(329.00443896,43.44644253)
\curveto(329.83777229,42.09227587)(330.20235562,40.5818592)(330.09818896,38.91519253)
\curveto(330.09818896,37.35269253)(329.68152229,36.10269253)(328.84818896,35.16519253)
\curveto(328.01485562,34.22769253)(326.76485562,33.60269253)(325.09818896,33.29019253)
\curveto(325.09818896,34.6443592)(324.05652229,35.63394253)(321.97318896,36.25894253)
\lineto(321.97318896,36.88394253)
\curveto(323.84818896,36.6756092)(325.20235562,36.57144253)(326.03568896,36.57144253)
\curveto(326.97318896,36.6756092)(327.54610562,37.40477587)(327.75443896,38.75894253)
\curveto(328.06693896,40.21727587)(327.96277229,41.57144253)(327.44193896,42.82144253)
\curveto(327.02527229,44.1756092)(325.98360562,45.94644253)(324.31693896,48.13394253)
\lineto(326.97318896,57.66519253)
\lineto(321.19193896,57.66519253)
\lineto(321.19193896,26.57144253)
\lineto(319.00443896,25.32144253)
\curveto(319.10860562,29.69644253)(319.16068896,36.2068592)(319.16068896,44.85269253)
\curveto(319.16068896,53.49852587)(319.10860562,58.5506092)(319.00443896,60.00894253)
\lineto(321.19193896,58.60269253)
\lineto(326.97318896,58.60269253)
\lineto(328.06693896,60.16519253)
\lineto(330.72318896,58.13394253)
\curveto(329.68152229,57.50894253)(328.79610562,56.41519253)(328.06693896,54.85269253)
\curveto(327.33777229,53.3943592)(326.34818896,51.15477587)(325.09818896,48.13394253)
\closepath
\moveto(335.41068896,57.35269253)
\lineto(335.41068896,48.75894253)
\lineto(345.41068896,48.75894253)
\lineto(345.41068896,57.35269253)
\lineto(335.41068896,57.35269253)
\closepath
\moveto(335.41068896,47.82144253)
\lineto(335.41068896,39.22769253)
\lineto(345.41068896,39.22769253)
\lineto(345.41068896,47.82144253)
\lineto(335.41068896,47.82144253)
\closepath
\moveto(335.41068896,38.29019253)
\lineto(335.41068896,28.75894253)
\lineto(345.41068896,28.75894253)
\lineto(345.41068896,38.29019253)
\lineto(335.41068896,38.29019253)
\closepath
\moveto(326.66068896,28.75894253)
\lineto(333.37943896,28.75894253)
\curveto(333.37943896,46.57144253)(333.32735562,56.88394253)(333.22318896,59.69644253)
\lineto(335.41068896,58.29019253)
\lineto(345.25443896,58.29019253)
\lineto(346.50443896,60.00894253)
\lineto(349.00443896,57.97769253)
\lineto(347.59818896,56.72769253)
\lineto(347.59818896,28.75894253)
\lineto(348.37943896,28.75894253)
\lineto(350.56693896,30.94644253)
\lineto(353.37943896,27.82144253)
\lineto(331.81693896,27.82144253)
\curveto(330.46277229,27.82144253)(329.21277229,27.66519253)(328.06693896,27.35269253)
\lineto(326.66068896,28.75894253)
\closepath
}
}
{
\newrgbcolor{curcolor}{0 0 0}
\pscustom[linestyle=none,fillstyle=solid,fillcolor=curcolor]
{
\newpath
\moveto(370.87943896,45.00894253)
\lineto(370.87943896,41.41519253)
\lineto(378.22318896,41.41519253)
\lineto(378.22318896,45.00894253)
\lineto(370.87943896,45.00894253)
\closepath
\moveto(385.87943896,52.66519253)
\lineto(387.59818896,55.47769253)
\lineto(362.59818896,55.47769253)
\curveto(362.80652229,54.3318592)(362.70235562,53.3943592)(362.28568896,52.66519253)
\curveto(361.86902229,51.93602587)(361.08777229,51.57144253)(359.94193896,51.57144253)
\curveto(358.90027229,51.6756092)(358.84818896,52.24852587)(359.78568896,53.29019253)
\curveto(360.72318896,54.43602587)(361.34818896,56.0506092)(361.66068896,58.13394253)
\lineto(362.28568896,58.13394253)
\lineto(362.44193896,56.41519253)
\lineto(387.44193896,56.41519253)
\lineto(388.69193896,57.82144253)
\lineto(391.03568896,55.16519253)
\curveto(389.57735562,55.06102587)(388.01485562,54.12352587)(386.34818896,52.35269253)
\lineto(385.87943896,52.66519253)
\closepath
\moveto(371.66068896,61.41519253)
\lineto(371.97318896,61.88394253)
\curveto(374.36902229,61.0506092)(375.77527229,60.2693592)(376.19193896,59.54019253)
\curveto(376.71277229,58.91519253)(376.60860562,58.13394253)(375.87943896,57.19644253)
\curveto(375.25443896,56.25894253)(374.73360562,56.3631092)(374.31693896,57.50894253)
\curveto(374.00443896,58.75894253)(373.11902229,60.06102587)(371.66068896,61.41519253)
\closepath
\moveto(361.19193896,50.79019253)
\lineto(368.84818896,50.79019253)
\curveto(368.84818896,52.1443592)(368.79610562,53.5506092)(368.69193896,55.00894253)
\lineto(372.12943896,53.60269253)
\lineto(370.87943896,52.66519253)
\lineto(370.87943896,50.79019253)
\lineto(378.22318896,50.79019253)
\curveto(378.22318896,52.1443592)(378.17110562,53.5506092)(378.06693896,55.00894253)
\lineto(381.66068896,53.60269253)
\lineto(380.25443896,52.50894253)
\lineto(380.25443896,50.79019253)
\lineto(383.06693896,50.79019253)
\lineto(384.94193896,52.66519253)
\lineto(387.75443896,49.85269253)
\lineto(380.25443896,49.85269253)
\lineto(380.25443896,45.94644253)
\lineto(383.22318896,45.94644253)
\lineto(385.09818896,47.82144253)
\lineto(387.91068896,45.00894253)
\lineto(380.25443896,45.00894253)
\lineto(380.25443896,41.41519253)
\lineto(386.81693896,41.41519253)
\lineto(389.00443896,43.60269253)
\lineto(392.12943896,40.47769253)
\lineto(379.78568896,40.47769253)
\curveto(381.55652229,38.29019253)(383.63985562,36.62352587)(386.03568896,35.47769253)
\curveto(388.53568896,34.43602587)(390.77527229,33.8631092)(392.75443896,33.75894253)
\lineto(392.75443896,33.13394253)
\curveto(391.29610562,33.02977587)(390.30652229,32.40477587)(389.78568896,31.25894253)
\curveto(387.59818896,32.19644253)(385.51485562,33.3943592)(383.53568896,34.85269253)
\curveto(381.66068896,36.41519253)(380.09818896,38.29019253)(378.84818896,40.47769253)
\lineto(370.56693896,40.47769253)
\curveto(369.21277229,38.0818592)(367.28568896,35.94644253)(364.78568896,34.07144253)
\curveto(362.38985562,32.19644253)(359.73360562,30.68602587)(356.81693896,29.54019253)
\lineto(356.50443896,30.00894253)
\curveto(359.52527229,31.6756092)(361.97318896,33.3943592)(363.84818896,35.16519253)
\curveto(365.82735562,37.04019253)(367.23360562,38.81102587)(368.06693896,40.47769253)
\lineto(362.59818896,40.47769253)
\curveto(361.24402229,40.47769253)(359.99402229,40.32144253)(358.84818896,40.00894253)
\lineto(357.44193896,41.41519253)
\lineto(368.84818896,41.41519253)
\lineto(368.84818896,45.00894253)
\lineto(367.28568896,45.00894253)
\curveto(365.93152229,45.00894253)(364.68152229,44.85269253)(363.53568896,44.54019253)
\lineto(362.12943896,45.94644253)
\lineto(368.84818896,45.94644253)
\lineto(368.84818896,49.85269253)
\lineto(366.34818896,49.85269253)
\curveto(364.99402229,49.85269253)(363.74402229,49.69644253)(362.59818896,49.38394253)
\lineto(361.19193896,50.79019253)
\closepath
\moveto(370.87943896,49.85269253)
\lineto(370.87943896,45.94644253)
\lineto(378.22318896,45.94644253)
\lineto(378.22318896,49.85269253)
\lineto(370.87943896,49.85269253)
\closepath
\moveto(366.03568896,34.85269253)
\lineto(373.69193896,34.85269253)
\curveto(373.69193896,36.31102587)(373.63985562,37.82144253)(373.53568896,39.38394253)
\lineto(377.12943896,37.82144253)
\lineto(375.87943896,36.72769253)
\lineto(375.87943896,34.85269253)
\lineto(379.47318896,34.85269253)
\lineto(381.19193896,36.57144253)
\lineto(383.84818896,33.91519253)
\lineto(375.87943896,33.91519253)
\lineto(375.87943896,27.97769253)
\lineto(383.37943896,27.97769253)
\lineto(385.56693896,30.16519253)
\lineto(388.69193896,27.04019253)
\lineto(365.56693896,27.04019253)
\curveto(364.21277229,27.04019253)(362.96277229,26.88394253)(361.81693896,26.57144253)
\lineto(360.41068896,27.97769253)
\lineto(373.69193896,27.97769253)
\lineto(373.69193896,33.91519253)
\lineto(371.19193896,33.91519253)
\curveto(369.83777229,33.91519253)(368.58777229,33.75894253)(367.44193896,33.44644253)
\lineto(366.03568896,34.85269253)
\closepath
}
}
{
\newrgbcolor{curcolor}{0 0 0}
\pscustom[linewidth=0.57914072,linecolor=curcolor]
{
\newpath
\moveto(437.85330828,42.64285367)
\curveto(437.85330828,66.03392803)(400.34289511,84.99614041)(354.07145556,84.99614041)
\curveto(307.80001601,84.99614041)(270.28960284,66.03392803)(270.28960284,42.64285367)
\curveto(270.28960284,19.2517793)(307.80001601,0.28956692)(354.07145556,0.28956692)
\curveto(400.34289511,0.28956692)(437.85330828,19.2517793)(437.85330828,42.64285367)
\closepath
}
}
{
\newrgbcolor{curcolor}{0 0 0}
\pscustom[linewidth=0.94413179,linecolor=curcolor]
{
\newpath
\moveto(160.57185526,167.71163985)
\lineto(93.34127526,90.43281985)
\lineto(93.34127526,93.19276985)
}
}
{
\newrgbcolor{curcolor}{0 0 0}
\pscustom[linestyle=none,fillstyle=solid,fillcolor=curcolor]
{
\newpath
\moveto(150.78658032,162.84641486)
\lineto(161.38226153,168.6685066)
\lineto(157.08261526,157.36902604)
\curveto(156.4138523,160.35360553)(153.86412532,162.5579574)(150.78658032,162.84641486)
\closepath
}
}
{
\newrgbcolor{curcolor}{0 0 0}
\pscustom[linewidth=0.64909061,linecolor=curcolor]
{
\newpath
\moveto(150.78658032,162.84641486)
\lineto(161.38226153,168.6685066)
\lineto(157.08261526,157.36902604)
\curveto(156.4138523,160.35360553)(153.86412532,162.5579574)(150.78658032,162.84641486)
\closepath
}
}
{
\newrgbcolor{curcolor}{0 0 0}
\pscustom[linewidth=0.90990525,linecolor=curcolor]
{
\newpath
\moveto(124.27171526,86.00716985)
\lineto(189.51007526,159.97643985)
\lineto(189.51007526,157.33469985)
}
}
{
\newrgbcolor{curcolor}{0 0 0}
\pscustom[linestyle=none,fillstyle=solid,fillcolor=curcolor]
{
\newpath
\moveto(133.73385302,90.63192623)
\lineto(123.48444863,85.09031189)
\lineto(127.70201812,95.95179859)
\curveto(128.32700579,93.07110875)(130.7698293,90.93004464)(133.73385302,90.63192623)
\closepath
}
}
{
\newrgbcolor{curcolor}{0 0 0}
\pscustom[linewidth=0.62555986,linecolor=curcolor]
{
\newpath
\moveto(133.73385302,90.63192623)
\lineto(123.48444863,85.09031189)
\lineto(127.70201812,95.95179859)
\curveto(128.32700579,93.07110875)(130.7698293,90.93004464)(133.73385302,90.63192623)
\closepath
}
}
{
\newrgbcolor{curcolor}{0 0 0}
\pscustom[linewidth=0.94413179,linecolor=curcolor]
{
\newpath
\moveto(160.57185526,167.71163985)
\lineto(93.34127526,90.43281985)
\lineto(93.34127526,93.19276985)
}
}
{
\newrgbcolor{curcolor}{0 0 0}
\pscustom[linestyle=none,fillstyle=solid,fillcolor=curcolor]
{
\newpath
\moveto(150.78658032,162.84641486)
\lineto(161.38226153,168.6685066)
\lineto(157.08261526,157.36902604)
\curveto(156.4138523,160.35360553)(153.86412532,162.5579574)(150.78658032,162.84641486)
\closepath
}
}
{
\newrgbcolor{curcolor}{0 0 0}
\pscustom[linewidth=0.64909061,linecolor=curcolor]
{
\newpath
\moveto(150.78658032,162.84641486)
\lineto(161.38226153,168.6685066)
\lineto(157.08261526,157.36902604)
\curveto(156.4138523,160.35360553)(153.86412532,162.5579574)(150.78658032,162.84641486)
\closepath
}
}
{
\newrgbcolor{curcolor}{0 0 0}
\pscustom[linewidth=1,linecolor=curcolor]
{
\newpath
\moveto(163.35714526,61.92858985)
\lineto(274.78571526,60.50001985)
}
}
{
\newrgbcolor{curcolor}{0 0 0}
\pscustom[linestyle=none,fillstyle=solid,fillcolor=curcolor]
{
\newpath
\moveto(264.15302842,65.07380957)
\lineto(276.11385039,60.50060853)
\lineto(264.03971683,56.2355486)
\curveto(265.99305666,58.81998723)(266.02775873,62.3897658)(264.15302842,65.07380957)
\closepath
}
}
{
\newrgbcolor{curcolor}{0 0 0}
\pscustom[linewidth=0.6875,linecolor=curcolor]
{
\newpath
\moveto(264.15302842,65.07380957)
\lineto(276.11385039,60.50060853)
\lineto(264.03971683,56.2355486)
\curveto(265.99305666,58.81998723)(266.02775873,62.3897658)(264.15302842,65.07380957)
\closepath
}
}
{
\newrgbcolor{curcolor}{0 0 0}
\pscustom[linewidth=0.98835522,linecolor=curcolor]
{
\newpath
\moveto(270.45302526,34.79508985)
\lineto(162.39879526,33.35601985)
}
}
{
\newrgbcolor{curcolor}{0 0 0}
\pscustom[linestyle=none,fillstyle=solid,fillcolor=curcolor]
{
\newpath
\moveto(173.02221727,29.11167449)
\lineto(161.08658952,33.32113284)
\lineto(172.9058811,37.84695908)
\curveto(171.04270228,35.24303385)(171.10062248,31.71513329)(173.02221727,29.11167449)
\closepath
}
}
{
\newrgbcolor{curcolor}{0 0 0}
\pscustom[linewidth=0.67949421,linecolor=curcolor]
{
\newpath
\moveto(173.02221727,29.11167449)
\lineto(161.08658952,33.32113284)
\lineto(172.9058811,37.84695908)
\curveto(171.04270228,35.24303385)(171.10062248,31.71513329)(173.02221727,29.11167449)
\closepath
}
}
\end{pspicture}

	\caption{单个线程的状态}
	\label{fig:thread_state}
\end{figure}

单独的中断程序的行为,与多线程程序研究中的线程行为十分相似。我们对中断模型的构
建就参考了多线程程序中的线程模型。如图~\ref{fig:thread_state}所示,通常,一个
线程会被刻画为以下三个状态。
\begin{itemize}
	\item \emph{就绪}:线程可以运行但是当前并不占有CPU。
	\item \emph{运行}:线程正在运行。
	\item \emph{阻塞}:线程在等待处理器以外的资源,暂时无法运行。
\end{itemize}
由于绝大部分多线程程序研究的场景中,并不关心线程产生和终止。换言之,线程在这类
应用场景里直接存在,且永不终止。中断研究中,一个中断程序通常具有一个从产生到终
止的完整周期。而且,一个中断并非只触发一次,因此一个中断程序可能重复多次上述周
期。这与传统的多线程程序研究是不同的。所以针对一个中断程序,我们类比线程再加以
修改可以得到如图~\ref{fig:interrupt_state}所示的状态机。每个状态的含义如下。

\begin{figure}[H]
	\centering
	%LaTeX with PSTricks extensions
%%Creator: 0.91_64bit
%%Please note this file requires PSTricks extensions
\psset{xunit=.5pt,yunit=.5pt,runit=.5pt}
\begin{pspicture}(438.14290264,259.57143889)
{
\newrgbcolor{curcolor}{0 0 0}
\pscustom[linewidth=0.57914072,linecolor=curcolor]
{
\newpath
\moveto(172.13899308,215.50001854)
\curveto(172.13899308,238.8910929)(134.62857991,257.85330528)(88.35714036,257.85330528)
\curveto(42.08570081,257.85330528)(4.57528764,238.8910929)(4.57528764,215.50001854)
\curveto(4.57528764,192.10894417)(42.08570081,173.14673179)(88.35714036,173.14673179)
\curveto(134.62857991,173.14673179)(172.13899308,192.10894417)(172.13899308,215.50001854)
\closepath
}
}
{
\newrgbcolor{curcolor}{0 0 0}
\pscustom[linestyle=none,fillstyle=solid,fillcolor=curcolor]
{
\newpath
\moveto(88.62499314,200.18751854)
\curveto(87.89582648,198.93751854)(86.69790981,198.31251854)(85.03124314,198.31251854)
\lineto(81.74999314,198.31251854)
\curveto(79.66665981,198.31251854)(78.62499314,199.25001854)(78.62499314,201.12501854)
\lineto(78.62499314,220.18751854)
\lineto(76.59374314,220.18751854)
\curveto(76.69790981,214.45835187)(76.02082648,209.8229352)(74.56249314,206.28126854)
\curveto(73.20832648,202.84376854)(70.49999314,199.61460187)(66.43749314,196.59376854)
\lineto(65.96874314,197.06251854)
\curveto(69.19790981,199.87501854)(71.43749314,203.00001854)(72.68749314,206.43751854)
\curveto(74.04165981,209.9791852)(74.66665981,214.56251854)(74.56249314,220.18751854)
\lineto(72.84374314,220.18751854)
\lineto(71.43749314,219.87501854)
\lineto(70.18749314,221.12501854)
\lineto(74.56249314,221.12501854)
\curveto(74.56249314,227.89585187)(74.51040981,231.80210187)(74.40624314,232.84376854)
\lineto(77.68749314,231.28126854)
\lineto(76.59374314,230.18751854)
\lineto(76.59374314,221.12501854)
\lineto(83.93749314,221.12501854)
\lineto(85.81249314,223.00001854)
\lineto(88.46874314,220.18751854)
\lineto(80.65624314,220.18751854)
\lineto(80.65624314,201.90626854)
\curveto(80.65624314,200.65626854)(81.28124314,200.03126854)(82.53124314,200.03126854)
\lineto(83.78124314,200.03126854)
\curveto(84.92707648,200.03126854)(85.55207648,200.55210187)(85.65624314,201.59376854)
\curveto(85.76040981,202.73960187)(85.81249314,204.40626854)(85.81249314,206.59376854)
\lineto(86.59374314,206.59376854)
\lineto(86.90624314,202.84376854)
\curveto(87.11457648,201.59376854)(87.68749314,200.70835187)(88.62499314,200.18751854)
\closepath
\moveto(79.71874314,230.50001854)
\curveto(81.59374314,229.56251854)(82.89582648,228.7291852)(83.62499314,228.00001854)
\curveto(84.35415981,227.37501854)(84.61457648,226.64585187)(84.40624314,225.81251854)
\curveto(84.19790981,225.08335187)(83.83332648,224.5104352)(83.31249314,224.09376854)
\curveto(82.89582648,223.78126854)(82.47915981,224.30210187)(82.06249314,225.65626854)
\curveto(81.64582648,227.11460187)(80.76040981,228.5729352)(79.40624314,230.03126854)
\lineto(79.71874314,230.50001854)
\closepath
\moveto(58.31249314,220.18751854)
\lineto(58.31249314,214.56251854)
\lineto(66.90624314,214.56251854)
\lineto(66.90624314,220.18751854)
\lineto(58.31249314,220.18751854)
\closepath
\moveto(60.49999314,232.84376854)
\curveto(61.85415981,232.11460187)(62.89582648,231.3854352)(63.62499314,230.65626854)
\curveto(64.35415981,229.92710187)(64.61457648,229.25001854)(64.40624314,228.62501854)
\curveto(64.30207648,228.00001854)(63.98957648,227.42710187)(63.46874314,226.90626854)
\curveto(62.94790981,226.48960187)(62.47915981,226.95835187)(62.06249314,228.31251854)
\curveto(61.74999314,229.77085187)(61.07290981,231.17710187)(60.03124314,232.53126854)
\lineto(60.49999314,232.84376854)
\closepath
\moveto(53.15624314,226.12501854)
\lineto(68.15624314,226.12501854)
\lineto(69.87499314,227.84376854)
\lineto(72.53124314,225.18751854)
\lineto(58.31249314,225.18751854)
\curveto(56.74999314,225.18751854)(55.49999314,225.03126854)(54.56249314,224.71876854)
\lineto(53.15624314,226.12501854)
\closepath
\moveto(56.12499314,211.90626854)
\curveto(56.22915981,213.5729352)(56.28124314,215.2916852)(56.28124314,217.06251854)
\curveto(56.28124314,218.83335187)(56.22915981,220.6041852)(56.12499314,222.37501854)
\lineto(58.31249314,221.12501854)
\lineto(66.74999314,221.12501854)
\lineto(67.99999314,222.53126854)
\lineto(70.18749314,220.50001854)
\lineto(68.93749314,219.56251854)
\curveto(68.93749314,216.75001854)(68.98957648,214.5104352)(69.09374314,212.84376854)
\lineto(66.90624314,211.90626854)
\lineto(66.90624314,213.62501854)
\lineto(63.93749314,213.62501854)
\lineto(63.93749314,201.28126854)
\curveto(63.93749314,200.23960187)(63.67707648,199.45835187)(63.15624314,198.93751854)
\curveto(62.73957648,198.4166852)(62.01040981,197.9479352)(60.96874314,197.53126854)
\curveto(60.86457648,198.78126854)(59.51040981,199.8229352)(56.90624314,200.65626854)
\lineto(56.90624314,201.28126854)
\curveto(59.30207648,200.96876854)(60.76040981,200.81251854)(61.28124314,200.81251854)
\curveto(61.80207648,200.9166852)(62.06249314,201.28126854)(62.06249314,201.90626854)
\lineto(62.06249314,213.62501854)
\lineto(58.31249314,213.62501854)
\lineto(58.31249314,212.68751854)
\lineto(56.12499314,211.90626854)
\closepath
\moveto(60.18749314,208.78126854)
\lineto(58.78124314,208.00001854)
\curveto(56.69790981,204.77085187)(54.71874314,202.1666852)(52.84374314,200.18751854)
\lineto(52.37499314,200.50001854)
\curveto(53.52082648,202.27085187)(54.51040981,204.0416852)(55.34374314,205.81251854)
\curveto(56.28124314,207.58335187)(56.95832648,209.1979352)(57.37499314,210.65626854)
\lineto(60.18749314,208.78126854)
\closepath
\moveto(65.65624314,209.71876854)
\lineto(65.96874314,210.18751854)
\curveto(67.53124314,209.25001854)(68.62499314,208.46876854)(69.24999314,207.84376854)
\curveto(69.87499314,207.21876854)(70.18749314,206.5416852)(70.18749314,205.81251854)
\curveto(70.18749314,205.18751854)(69.87499314,204.5104352)(69.24999314,203.78126854)
\curveto(68.72915981,203.15626854)(68.31249314,203.5729352)(67.99999314,205.03126854)
\curveto(67.68749314,206.48960187)(66.90624314,208.05210187)(65.65624314,209.71876854)
\closepath
}
}
{
\newrgbcolor{curcolor}{0 0 0}
\pscustom[linestyle=none,fillstyle=solid,fillcolor=curcolor]
{
\newpath
\moveto(114.87499314,225.34376854)
\lineto(114.87499314,219.09376854)
\lineto(115.96874314,219.09376854)
\curveto(117.42707648,220.7604352)(118.98957648,222.84376854)(120.65624314,225.34376854)
\lineto(114.87499314,225.34376854)
\closepath
\moveto(111.59374314,211.75001854)
\lineto(111.59374314,207.06251854)
\lineto(121.90624314,207.06251854)
\lineto(121.90624314,211.75001854)
\lineto(111.59374314,211.75001854)
\closepath
\moveto(111.59374314,206.12501854)
\lineto(111.59374314,200.65626854)
\lineto(121.90624314,200.65626854)
\lineto(121.90624314,206.12501854)
\lineto(111.59374314,206.12501854)
\closepath
\moveto(109.40624314,196.59376854)
\curveto(109.51040981,199.1979352)(109.56249314,203.78126854)(109.56249314,210.34376854)
\curveto(106.64582648,208.2604352)(104.40624314,206.8541852)(102.84374314,206.12501854)
\lineto(102.53124314,206.59376854)
\curveto(105.44790981,208.67710187)(107.79165981,210.6041852)(109.56249314,212.37501854)
\curveto(109.56249314,212.89585187)(109.51040981,213.5729352)(109.40624314,214.40626854)
\lineto(110.65624314,213.46876854)
\curveto(112.32290981,215.03126854)(113.88540981,216.59376854)(115.34374314,218.15626854)
\lineto(109.40624314,218.15626854)
\curveto(108.05207648,218.15626854)(106.80207648,218.00001854)(105.65624314,217.68751854)
\lineto(104.24999314,219.09376854)
\lineto(112.84374314,219.09376854)
\lineto(112.84374314,225.34376854)
\lineto(111.28124314,225.34376854)
\curveto(109.92707648,225.34376854)(108.67707648,225.18751854)(107.53124314,224.87501854)
\lineto(106.12499314,226.28126854)
\lineto(112.84374314,226.28126854)
\curveto(112.84374314,229.71876854)(112.79165981,232.06251854)(112.68749314,233.31251854)
\lineto(116.12499314,231.75001854)
\lineto(114.87499314,230.65626854)
\lineto(114.87499314,226.28126854)
\lineto(116.59374314,226.28126854)
\lineto(118.62499314,228.00001854)
\lineto(120.81249314,225.81251854)
\curveto(121.64582648,227.06251854)(122.47915981,228.62501854)(123.31249314,230.50001854)
\lineto(126.28124314,228.31251854)
\curveto(125.44790981,228.1041852)(124.35415981,227.06251854)(122.99999314,225.18751854)
\curveto(121.74999314,223.31251854)(120.23957648,221.28126854)(118.46874314,219.09376854)
\lineto(123.62499314,219.09376854)
\lineto(125.81249314,221.12501854)
\lineto(128.46874314,218.15626854)
\lineto(117.84374314,218.15626854)
\curveto(116.07290981,216.28126854)(114.24999314,214.45835187)(112.37499314,212.68751854)
\lineto(121.59374314,212.68751854)
\lineto(122.68749314,214.25001854)
\lineto(125.18749314,212.21876854)
\lineto(123.93749314,211.28126854)
\lineto(123.93749314,201.28126854)
\curveto(123.93749314,199.92710187)(123.98957648,198.7291852)(124.09374314,197.68751854)
\lineto(121.90624314,196.75001854)
\lineto(121.90624314,199.71876854)
\lineto(111.59374314,199.71876854)
\lineto(111.59374314,197.53126854)
\lineto(109.40624314,196.59376854)
\closepath
\moveto(92.53124314,202.68751854)
\curveto(93.46874314,202.7916852)(95.08332648,203.05210187)(97.37499314,203.46876854)
\curveto(99.77082648,203.8854352)(102.79165981,204.45835187)(106.43749314,205.18751854)
\lineto(106.59374314,204.56251854)
\curveto(103.67707648,203.7291852)(101.12499314,202.9479352)(98.93749314,202.21876854)
\curveto(96.74999314,201.48960187)(95.13540981,200.65626854)(94.09374314,199.71876854)
\lineto(92.53124314,202.68751854)
\closepath
\moveto(99.87499314,232.84376854)
\lineto(102.99999314,230.96876854)
\curveto(102.16665981,230.55210187)(101.07290981,229.30210187)(99.71874314,227.21876854)
\curveto(98.36457648,225.23960187)(96.80207648,223.00001854)(95.03124314,220.50001854)
\lineto(102.21874314,220.65626854)
\curveto(103.05207648,222.0104352)(103.78124314,223.46876854)(104.40624314,225.03126854)
\lineto(106.90624314,223.00001854)
\curveto(105.76040981,222.1666852)(104.40624314,220.70835187)(102.84374314,218.62501854)
\curveto(101.38540981,216.5416852)(99.19790981,213.93751854)(96.28124314,210.81251854)
\curveto(99.19790981,211.2291852)(102.47915981,211.80210187)(106.12499314,212.53126854)
\lineto(106.28124314,211.90626854)
\curveto(103.57290981,211.0729352)(101.43749314,210.39585187)(99.87499314,209.87501854)
\curveto(98.31249314,209.3541852)(96.95832648,208.67710187)(95.81249314,207.84376854)
\lineto(93.78124314,211.12501854)
\curveto(94.61457648,211.43751854)(95.60415981,212.1666852)(96.74999314,213.31251854)
\curveto(97.89582648,214.45835187)(99.56249314,216.64585187)(101.74999314,219.87501854)
\curveto(99.87499314,219.6666852)(98.41665981,219.40626854)(97.37499314,219.09376854)
\curveto(96.33332648,218.8854352)(95.29165981,218.46876854)(94.24999314,217.84376854)
\lineto(92.53124314,220.65626854)
\curveto(93.46874314,220.7604352)(94.66665981,222.0104352)(96.12499314,224.40626854)
\curveto(97.58332648,226.80210187)(98.83332648,229.61460187)(99.87499314,232.84376854)
\closepath
}
}
{
\newrgbcolor{curcolor}{0 0 0}
\pscustom[linestyle=none,fillstyle=solid,fillcolor=curcolor]
{
\newpath
\moveto(61.52678999,62.75448575)
\lineto(75.12053999,62.75448575)
\lineto(77.30803999,64.78573575)
\lineto(80.12053999,61.81698575)
\lineto(65.74553999,61.81698575)
\curveto(65.01637333,61.81698575)(64.07887333,61.66073575)(62.93303999,61.34823575)
\lineto(61.52678999,62.75448575)
\closepath
\moveto(60.58928999,42.44198575)
\curveto(61.63095666,42.65031909)(62.62053999,43.32740242)(63.55803999,44.47323575)
\curveto(64.49553999,45.61906909)(65.48512333,47.07740242)(66.52678999,48.84823575)
\curveto(67.56845666,50.61906909)(68.29762333,52.23365242)(68.71428999,53.69198575)
\lineto(63.08928999,53.69198575)
\curveto(62.25595666,53.69198575)(61.26637333,53.53573575)(60.12053999,53.22323575)
\lineto(58.71428999,54.62948575)
\lineto(78.08928999,54.62948575)
\lineto(80.43303999,56.97323575)
\lineto(83.40178999,53.69198575)
\lineto(69.18303999,53.69198575)
\lineto(71.99553999,51.66073575)
\curveto(71.16220666,51.66073575)(69.91220666,50.51490242)(68.24553999,48.22323575)
\curveto(66.68303999,46.03573575)(64.96428999,44.00448575)(63.08928999,42.12948575)
\lineto(76.99553999,42.75448575)
\curveto(76.26637333,44.31698575)(75.12053999,45.98365242)(73.55803999,47.75448575)
\lineto(74.02678999,48.06698575)
\curveto(76.21428999,46.71281909)(77.88095666,45.41073575)(79.02678999,44.16073575)
\curveto(80.27678999,43.01490242)(80.79762333,41.92115242)(80.58928999,40.87948575)
\curveto(80.48512333,39.94198575)(80.17262333,39.26490242)(79.65178999,38.84823575)
\curveto(79.13095666,38.43156909)(78.71428999,38.63990242)(78.40178999,39.47323575)
\curveto(78.08928999,40.41073575)(77.77678999,41.24406909)(77.46428999,41.97323575)
\curveto(72.04762333,41.45240242)(68.40178999,40.98365242)(66.52678999,40.56698575)
\curveto(64.75595666,40.15031909)(63.40178999,39.62948575)(62.46428999,39.00448575)
\lineto(60.58928999,42.44198575)
\closepath
\moveto(51.37053999,65.56698575)
\lineto(51.83928999,65.87948575)
\curveto(53.92262333,64.83781909)(55.32887333,63.95240242)(56.05803999,63.22323575)
\curveto(56.78720666,62.49406909)(57.09970666,61.71281909)(56.99553999,60.87948575)
\curveto(56.89137333,60.15031909)(56.47470666,59.47323575)(55.74553999,58.84823575)
\curveto(55.12053999,58.32740242)(54.65178999,58.79615242)(54.33928999,60.25448575)
\curveto(54.02678999,61.71281909)(53.03720666,63.48365242)(51.37053999,65.56698575)
\closepath
\moveto(56.21428999,39.94198575)
\curveto(57.15178999,38.90031909)(58.45387333,37.91073575)(60.12053999,36.97323575)
\curveto(61.78720666,36.03573575)(64.18303999,35.46281909)(67.30803999,35.25448575)
\curveto(70.53720666,35.15031909)(73.55803999,35.15031909)(76.37053999,35.25448575)
\curveto(79.28720666,35.35865242)(82.04762333,35.61906909)(84.65178999,36.03573575)
\lineto(84.65178999,35.41073575)
\curveto(82.46428999,34.78573575)(81.37053999,33.90031909)(81.37053999,32.75448575)
\curveto(76.68303999,32.75448575)(72.77678999,32.80656909)(69.65178999,32.91073575)
\curveto(66.63095666,33.01490242)(64.13095666,33.48365242)(62.15178999,34.31698575)
\curveto(60.27678999,35.04615242)(58.76637333,35.93156909)(57.62053999,36.97323575)
\curveto(56.47470666,38.11906909)(55.64137333,38.74406909)(55.12053999,38.84823575)
\curveto(54.59970666,38.95240242)(53.71428999,38.37948575)(52.46428999,37.12948575)
\curveto(51.31845666,35.87948575)(50.48512333,34.88990242)(49.96428999,34.16073575)
\lineto(47.77678999,36.34823575)
\curveto(49.44345666,37.70240242)(51.57887333,38.90031909)(54.18303999,39.94198575)
\lineto(54.18303999,52.59823575)
\lineto(53.08928999,52.59823575)
\curveto(51.73512333,52.59823575)(50.48512333,52.44198575)(49.33928999,52.12948575)
\lineto(47.93303999,53.53573575)
\lineto(53.87053999,53.53573575)
\lineto(54.96428999,55.25448575)
\lineto(57.62053999,53.22323575)
\lineto(56.21428999,52.12948575)
\lineto(56.21428999,39.94198575)
\closepath
}
}
{
\newrgbcolor{curcolor}{0 0 0}
\pscustom[linestyle=none,fillstyle=solid,fillcolor=curcolor]
{
\newpath
\moveto(103.87053999,63.22323575)
\lineto(117.15178999,63.22323575)
\lineto(119.33928999,65.25448575)
\lineto(121.99553999,62.28573575)
\lineto(109.02678999,62.28573575)
\curveto(107.67262333,62.28573575)(106.42262333,62.12948575)(105.27678999,61.81698575)
\lineto(103.87053999,63.22323575)
\closepath
\moveto(105.27678999,35.41073575)
\curveto(108.08928999,35.20240242)(109.96428999,35.09823575)(110.90178999,35.09823575)
\curveto(111.83928999,35.09823575)(112.30803999,35.72323575)(112.30803999,36.97323575)
\lineto(112.30803999,52.91073575)
\lineto(106.37053999,52.91073575)
\curveto(105.01637333,52.91073575)(103.76637333,52.75448575)(102.62053999,52.44198575)
\lineto(101.21428999,53.84823575)
\lineto(119.18303999,53.84823575)
\lineto(121.52678999,56.03573575)
\lineto(124.18303999,52.91073575)
\lineto(114.49553999,52.91073575)
\lineto(114.49553999,36.19198575)
\curveto(114.59970666,33.37948575)(113.29762333,31.71281909)(110.58928999,31.19198575)
\curveto(110.58928999,32.65031909)(108.81845666,33.79615242)(105.27678999,34.62948575)
\lineto(105.27678999,35.41073575)
\closepath
\moveto(97.62053999,67.12948575)
\lineto(100.43303999,64.94198575)
\curveto(99.70387333,64.73365242)(98.45387333,63.58781909)(96.68303999,61.50448575)
\curveto(94.91220666,59.52531909)(92.67262333,57.54615242)(89.96428999,55.56698575)
\lineto(89.65178999,56.03573575)
\curveto(91.42262333,57.70240242)(93.03720666,59.57740242)(94.49553999,61.66073575)
\curveto(96.05803999,63.84823575)(97.09970666,65.67115242)(97.62053999,67.12948575)
\closepath
\moveto(97.93303999,49.31698575)
\lineto(97.93303999,32.28573575)
\lineto(95.58928999,30.87948575)
\curveto(95.69345666,33.17115242)(95.74553999,39.52531909)(95.74553999,49.94198575)
\curveto(93.34970666,47.44198575)(91.00595666,45.41073575)(88.71428999,43.84823575)
\lineto(88.40178999,44.31698575)
\curveto(90.58928999,46.50448575)(92.67262333,48.95240242)(94.65178999,51.66073575)
\curveto(96.73512333,54.36906909)(98.29762333,56.97323575)(99.33928999,59.47323575)
\lineto(101.99553999,57.12948575)
\lineto(100.58928999,56.34823575)
\curveto(99.23512333,54.36906909)(98.08928999,52.80656909)(97.15178999,51.66073575)
\lineto(98.87053999,50.25448575)
\lineto(97.93303999,49.31698575)
\closepath
}
}
{
\newrgbcolor{curcolor}{0 0 0}
\pscustom[linewidth=0.57914072,linecolor=curcolor]
{
\newpath
\moveto(167.85328993,48.35714689)
\curveto(167.85328993,71.74822125)(130.34287676,90.71043363)(84.07143721,90.71043363)
\curveto(37.79999766,90.71043363)(0.28958449,71.74822125)(0.28958449,48.35714689)
\curveto(0.28958449,24.96607252)(37.79999766,6.00386014)(84.07143721,6.00386014)
\curveto(130.34287676,6.00386014)(167.85328993,24.96607252)(167.85328993,48.35714689)
\closepath
}
}
{
\newrgbcolor{curcolor}{0 0 0}
\pscustom[linestyle=none,fillstyle=solid,fillcolor=curcolor]
{
\newpath
\moveto(325.09818748,48.13395297)
\curveto(326.86902081,46.4672863)(328.17110414,44.9047863)(329.00443748,43.44645297)
\curveto(329.83777081,42.0922863)(330.20235414,40.58186964)(330.09818748,38.91520297)
\curveto(330.09818748,37.35270297)(329.68152081,36.10270297)(328.84818748,35.16520297)
\curveto(328.01485414,34.22770297)(326.76485414,33.60270297)(325.09818748,33.29020297)
\curveto(325.09818748,34.64436964)(324.05652081,35.63395297)(321.97318748,36.25895297)
\lineto(321.97318748,36.88395297)
\curveto(323.84818748,36.67561964)(325.20235414,36.57145297)(326.03568748,36.57145297)
\curveto(326.97318748,36.67561964)(327.54610414,37.4047863)(327.75443748,38.75895297)
\curveto(328.06693748,40.2172863)(327.96277081,41.57145297)(327.44193748,42.82145297)
\curveto(327.02527081,44.17561964)(325.98360414,45.94645297)(324.31693748,48.13395297)
\lineto(326.97318748,57.66520297)
\lineto(321.19193748,57.66520297)
\lineto(321.19193748,26.57145297)
\lineto(319.00443748,25.32145297)
\curveto(319.10860414,29.69645297)(319.16068748,36.20686964)(319.16068748,44.85270297)
\curveto(319.16068748,53.4985363)(319.10860414,58.55061964)(319.00443748,60.00895297)
\lineto(321.19193748,58.60270297)
\lineto(326.97318748,58.60270297)
\lineto(328.06693748,60.16520297)
\lineto(330.72318748,58.13395297)
\curveto(329.68152081,57.50895297)(328.79610414,56.41520297)(328.06693748,54.85270297)
\curveto(327.33777081,53.39436964)(326.34818748,51.1547863)(325.09818748,48.13395297)
\closepath
\moveto(335.41068748,57.35270297)
\lineto(335.41068748,48.75895297)
\lineto(345.41068748,48.75895297)
\lineto(345.41068748,57.35270297)
\lineto(335.41068748,57.35270297)
\closepath
\moveto(335.41068748,47.82145297)
\lineto(335.41068748,39.22770297)
\lineto(345.41068748,39.22770297)
\lineto(345.41068748,47.82145297)
\lineto(335.41068748,47.82145297)
\closepath
\moveto(335.41068748,38.29020297)
\lineto(335.41068748,28.75895297)
\lineto(345.41068748,28.75895297)
\lineto(345.41068748,38.29020297)
\lineto(335.41068748,38.29020297)
\closepath
\moveto(326.66068748,28.75895297)
\lineto(333.37943748,28.75895297)
\curveto(333.37943748,46.57145297)(333.32735414,56.88395297)(333.22318748,59.69645297)
\lineto(335.41068748,58.29020297)
\lineto(345.25443748,58.29020297)
\lineto(346.50443748,60.00895297)
\lineto(349.00443748,57.97770297)
\lineto(347.59818748,56.72770297)
\lineto(347.59818748,28.75895297)
\lineto(348.37943748,28.75895297)
\lineto(350.56693748,30.94645297)
\lineto(353.37943748,27.82145297)
\lineto(331.81693748,27.82145297)
\curveto(330.46277081,27.82145297)(329.21277081,27.66520297)(328.06693748,27.35270297)
\lineto(326.66068748,28.75895297)
\closepath
}
}
{
\newrgbcolor{curcolor}{0 0 0}
\pscustom[linestyle=none,fillstyle=solid,fillcolor=curcolor]
{
\newpath
\moveto(370.87943748,45.00895297)
\lineto(370.87943748,41.41520297)
\lineto(378.22318748,41.41520297)
\lineto(378.22318748,45.00895297)
\lineto(370.87943748,45.00895297)
\closepath
\moveto(385.87943748,52.66520297)
\lineto(387.59818748,55.47770297)
\lineto(362.59818748,55.47770297)
\curveto(362.80652081,54.33186964)(362.70235414,53.39436964)(362.28568748,52.66520297)
\curveto(361.86902081,51.9360363)(361.08777081,51.57145297)(359.94193748,51.57145297)
\curveto(358.90027081,51.67561964)(358.84818748,52.2485363)(359.78568748,53.29020297)
\curveto(360.72318748,54.4360363)(361.34818748,56.05061964)(361.66068748,58.13395297)
\lineto(362.28568748,58.13395297)
\lineto(362.44193748,56.41520297)
\lineto(387.44193748,56.41520297)
\lineto(388.69193748,57.82145297)
\lineto(391.03568748,55.16520297)
\curveto(389.57735414,55.0610363)(388.01485414,54.1235363)(386.34818748,52.35270297)
\lineto(385.87943748,52.66520297)
\closepath
\moveto(371.66068748,61.41520297)
\lineto(371.97318748,61.88395297)
\curveto(374.36902081,61.05061964)(375.77527081,60.26936964)(376.19193748,59.54020297)
\curveto(376.71277081,58.91520297)(376.60860414,58.13395297)(375.87943748,57.19645297)
\curveto(375.25443748,56.25895297)(374.73360414,56.36311964)(374.31693748,57.50895297)
\curveto(374.00443748,58.75895297)(373.11902081,60.0610363)(371.66068748,61.41520297)
\closepath
\moveto(361.19193748,50.79020297)
\lineto(368.84818748,50.79020297)
\curveto(368.84818748,52.14436964)(368.79610414,53.55061964)(368.69193748,55.00895297)
\lineto(372.12943748,53.60270297)
\lineto(370.87943748,52.66520297)
\lineto(370.87943748,50.79020297)
\lineto(378.22318748,50.79020297)
\curveto(378.22318748,52.14436964)(378.17110414,53.55061964)(378.06693748,55.00895297)
\lineto(381.66068748,53.60270297)
\lineto(380.25443748,52.50895297)
\lineto(380.25443748,50.79020297)
\lineto(383.06693748,50.79020297)
\lineto(384.94193748,52.66520297)
\lineto(387.75443748,49.85270297)
\lineto(380.25443748,49.85270297)
\lineto(380.25443748,45.94645297)
\lineto(383.22318748,45.94645297)
\lineto(385.09818748,47.82145297)
\lineto(387.91068748,45.00895297)
\lineto(380.25443748,45.00895297)
\lineto(380.25443748,41.41520297)
\lineto(386.81693748,41.41520297)
\lineto(389.00443748,43.60270297)
\lineto(392.12943748,40.47770297)
\lineto(379.78568748,40.47770297)
\curveto(381.55652081,38.29020297)(383.63985414,36.6235363)(386.03568748,35.47770297)
\curveto(388.53568748,34.4360363)(390.77527081,33.86311964)(392.75443748,33.75895297)
\lineto(392.75443748,33.13395297)
\curveto(391.29610414,33.0297863)(390.30652081,32.4047863)(389.78568748,31.25895297)
\curveto(387.59818748,32.19645297)(385.51485414,33.39436964)(383.53568748,34.85270297)
\curveto(381.66068748,36.41520297)(380.09818748,38.29020297)(378.84818748,40.47770297)
\lineto(370.56693748,40.47770297)
\curveto(369.21277081,38.08186964)(367.28568748,35.94645297)(364.78568748,34.07145297)
\curveto(362.38985414,32.19645297)(359.73360414,30.6860363)(356.81693748,29.54020297)
\lineto(356.50443748,30.00895297)
\curveto(359.52527081,31.67561964)(361.97318748,33.39436964)(363.84818748,35.16520297)
\curveto(365.82735414,37.04020297)(367.23360414,38.8110363)(368.06693748,40.47770297)
\lineto(362.59818748,40.47770297)
\curveto(361.24402081,40.47770297)(359.99402081,40.32145297)(358.84818748,40.00895297)
\lineto(357.44193748,41.41520297)
\lineto(368.84818748,41.41520297)
\lineto(368.84818748,45.00895297)
\lineto(367.28568748,45.00895297)
\curveto(365.93152081,45.00895297)(364.68152081,44.85270297)(363.53568748,44.54020297)
\lineto(362.12943748,45.94645297)
\lineto(368.84818748,45.94645297)
\lineto(368.84818748,49.85270297)
\lineto(366.34818748,49.85270297)
\curveto(364.99402081,49.85270297)(363.74402081,49.69645297)(362.59818748,49.38395297)
\lineto(361.19193748,50.79020297)
\closepath
\moveto(370.87943748,49.85270297)
\lineto(370.87943748,45.94645297)
\lineto(378.22318748,45.94645297)
\lineto(378.22318748,49.85270297)
\lineto(370.87943748,49.85270297)
\closepath
\moveto(366.03568748,34.85270297)
\lineto(373.69193748,34.85270297)
\curveto(373.69193748,36.3110363)(373.63985414,37.82145297)(373.53568748,39.38395297)
\lineto(377.12943748,37.82145297)
\lineto(375.87943748,36.72770297)
\lineto(375.87943748,34.85270297)
\lineto(379.47318748,34.85270297)
\lineto(381.19193748,36.57145297)
\lineto(383.84818748,33.91520297)
\lineto(375.87943748,33.91520297)
\lineto(375.87943748,27.97770297)
\lineto(383.37943748,27.97770297)
\lineto(385.56693748,30.16520297)
\lineto(388.69193748,27.04020297)
\lineto(365.56693748,27.04020297)
\curveto(364.21277081,27.04020297)(362.96277081,26.88395297)(361.81693748,26.57145297)
\lineto(360.41068748,27.97770297)
\lineto(373.69193748,27.97770297)
\lineto(373.69193748,33.91520297)
\lineto(371.19193748,33.91520297)
\curveto(369.83777081,33.91520297)(368.58777081,33.75895297)(367.44193748,33.44645297)
\lineto(366.03568748,34.85270297)
\closepath
}
}
{
\newrgbcolor{curcolor}{0 0 0}
\pscustom[linewidth=0.57914072,linecolor=curcolor]
{
\newpath
\moveto(437.8533068,42.6428641)
\curveto(437.8533068,66.03393847)(400.34289363,84.99615084)(354.07145408,84.99615084)
\curveto(307.80001452,84.99615084)(270.28960135,66.03393847)(270.28960135,42.6428641)
\curveto(270.28960135,19.25178974)(307.80001452,0.28957736)(354.07145408,0.28957736)
\curveto(400.34289363,0.28957736)(437.8533068,19.25178974)(437.8533068,42.6428641)
\closepath
}
}
{
\newrgbcolor{curcolor}{0 0 0}
\pscustom[linewidth=1.25399995,linecolor=curcolor]
{
\newpath
\moveto(89.12134578,89.37438889)
\lineto(88.85882578,175.65171889)
\lineto(88.85882578,172.57039889)
}
}
{
\newrgbcolor{curcolor}{0 0 0}
\pscustom[linestyle=none,fillstyle=solid,fillcolor=curcolor]
{
\newpath
\moveto(94.64466381,102.79707646)
\lineto(89.14850153,87.70912874)
\lineto(83.5606255,102.76335137)
\curveto(86.83995713,100.36557801)(91.31658636,100.39307277)(94.64466381,102.79707646)
\closepath
}
}
{
\newrgbcolor{curcolor}{0 0 0}
\pscustom[linewidth=0.86212496,linecolor=curcolor]
{
\newpath
\moveto(94.64466381,102.79707646)
\lineto(89.14850153,87.70912874)
\lineto(83.5606255,102.76335137)
\curveto(86.83995713,100.36557801)(91.31658636,100.39307277)(94.64466381,102.79707646)
\closepath
}
}
{
\newrgbcolor{curcolor}{0 0 0}
\pscustom[linewidth=1.34200001,linecolor=curcolor]
{
\newpath
\moveto(125.59134578,87.00878889)
\lineto(351.19624578,174.70453889)
\lineto(351.19624578,171.57257889)
}
}
{
\newrgbcolor{curcolor}{0 0 0}
\pscustom[linestyle=none,fillstyle=solid,fillcolor=curcolor]
{
\newpath
\moveto(141.1205939,86.65657675)
\lineto(123.93879257,86.3410567)
\lineto(136.82296682,97.71259647)
\curveto(135.68998274,93.51531028)(137.43956642,89.05533439)(141.1205939,86.65657675)
\closepath
}
}
{
\newrgbcolor{curcolor}{0 0 0}
\pscustom[linewidth=0.92262501,linecolor=curcolor]
{
\newpath
\moveto(141.1205939,86.65657675)
\lineto(123.93879257,86.3410567)
\lineto(136.82296682,97.71259647)
\curveto(135.68998274,93.51531028)(137.43956642,89.05533439)(141.1205939,86.65657675)
\closepath
}
}
{
\newrgbcolor{curcolor}{0 0 0}
\pscustom[linewidth=1.3884443,linecolor=curcolor]
{
\newpath
\moveto(310.81111578,179.71843889)
\lineto(94.51310578,92.63512889)
\lineto(94.51310578,95.74522889)
}
}
{
\newrgbcolor{curcolor}{0 0 0}
\pscustom[linestyle=none,fillstyle=solid,fillcolor=curcolor]
{
\newpath
\moveto(294.74120904,179.88980336)
\lineto(312.51243846,180.42976963)
\lineto(299.32466239,168.50539395)
\curveto(300.44460504,172.86170929)(298.57916951,177.45395892)(294.74120904,179.88980336)
\closepath
}
}
{
\newrgbcolor{curcolor}{0 0 0}
\pscustom[linewidth=0.95455546,linecolor=curcolor]
{
\newpath
\moveto(294.74120904,179.88980336)
\lineto(312.51243846,180.42976963)
\lineto(299.32466239,168.50539395)
\curveto(300.44460504,172.86170929)(298.57916951,177.45395892)(294.74120904,179.88980336)
\closepath
}
}
{
\newrgbcolor{curcolor}{0 0 0}
\pscustom[linewidth=1.29999995,linecolor=curcolor]
{
\newpath
\moveto(163.35714578,61.92859889)
\lineto(274.78571578,60.50002889)
}
}
{
\newrgbcolor{curcolor}{0 0 0}
\pscustom[linestyle=none,fillstyle=solid,fillcolor=curcolor]
{
\newpath
\moveto(260.96322339,66.44595531)
\lineto(276.51229139,60.50079417)
\lineto(260.81591833,54.95621647)
\curveto(263.35526001,58.31598656)(263.40037271,62.95669853)(260.96322339,66.44595531)
\closepath
}
}
{
\newrgbcolor{curcolor}{0 0 0}
\pscustom[linewidth=0.89374997,linecolor=curcolor]
{
\newpath
\moveto(260.96322339,66.44595531)
\lineto(276.51229139,60.50079417)
\lineto(260.81591833,54.95621647)
\curveto(263.35526001,58.31598656)(263.40037271,62.95669853)(260.96322339,66.44595531)
\closepath
}
}
{
\newrgbcolor{curcolor}{0 0 0}
\pscustom[linewidth=1.19559932,linecolor=curcolor]
{
\newpath
\moveto(271.98802578,38.45679889)
\lineto(168.04163578,38.48199889)
}
}
{
\newrgbcolor{curcolor}{0 0 0}
\pscustom[linestyle=none,fillstyle=solid,fillcolor=curcolor]
{
\newpath
\moveto(180.82183993,33.17389623)
\lineto(166.45385276,38.46132383)
\lineto(180.82440272,43.74178311)
\curveto(178.52803812,40.6227019)(178.54023071,36.35449282)(180.82183993,33.17389623)
\closepath
}
}
{
\newrgbcolor{curcolor}{0 0 0}
\pscustom[linewidth=0.82197453,linecolor=curcolor]
{
\newpath
\moveto(180.82183993,33.17389623)
\lineto(166.45385276,38.46132383)
\lineto(180.82440272,43.74178311)
\curveto(178.52803812,40.6227019)(178.54023071,36.35449282)(180.82183993,33.17389623)
\closepath
}
}
{
\newrgbcolor{curcolor}{0 0 0}
\pscustom[linewidth=0.57914072,linecolor=curcolor]
{
\newpath
\moveto(437.85329737,216.92858575)
\curveto(437.85329737,240.31966012)(400.3428842,259.2818725)(354.07144464,259.2818725)
\curveto(307.80000509,259.2818725)(270.28959192,240.31966012)(270.28959192,216.92858575)
\curveto(270.28959192,193.53751139)(307.80000509,174.57529901)(354.07144464,174.57529901)
\curveto(400.3428842,174.57529901)(437.85329737,193.53751139)(437.85329737,216.92858575)
\closepath
}
}
{
\newrgbcolor{curcolor}{0 0 0}
\pscustom[linestyle=none,fillstyle=solid,fillcolor=curcolor]
{
\newpath
\moveto(313.87051691,221.4821441)
\curveto(314.18301691,225.33631077)(314.39135024,228.8258941)(314.49551691,231.9508941)
\lineto(306.05801691,231.9508941)
\curveto(304.70385024,231.9508941)(303.45385024,231.7946441)(302.30801691,231.4821441)
\lineto(300.90176691,232.8883941)
\lineto(326.05801691,232.8883941)
\lineto(328.40176691,235.2321441)
\lineto(331.68301691,231.9508941)
\lineto(316.83926691,231.9508941)
\curveto(316.73510024,229.65922744)(316.52676691,226.1696441)(316.21426691,221.4821441)
\lineto(327.46426691,221.4821441)
\lineto(330.12051691,224.1383941)
\lineto(333.71426691,220.5446441)
\lineto(319.65176691,220.5446441)
\lineto(319.65176691,204.9196441)
\curveto(319.75593358,203.56547744)(320.64135024,202.94047744)(322.30801691,203.0446441)
\lineto(328.71426691,203.0446441)
\curveto(329.96426691,203.14881077)(330.69343358,203.72172744)(330.90176691,204.7633941)
\curveto(331.11010024,205.90922744)(331.26635024,207.94047744)(331.37051691,210.8571441)
\lineto(332.15176691,210.8571441)
\curveto(332.04760024,206.1696441)(332.82885024,203.8258941)(334.49551691,203.8258941)
\curveto(333.66218358,201.84672744)(331.83926691,200.96131077)(329.02676691,201.1696441)
\lineto(320.74551691,201.1696441)
\curveto(318.55801691,201.1696441)(317.46426691,202.05506077)(317.46426691,203.8258941)
\lineto(317.46426691,220.5446441)
\lineto(316.05801691,220.5446441)
\curveto(315.32885024,215.33631077)(313.71426691,211.06547744)(311.21426691,207.7321441)
\curveto(308.71426691,204.50297744)(304.39135024,201.74256077)(298.24551691,199.4508941)
\lineto(298.08926691,200.0758941)
\curveto(303.29760024,202.5758941)(306.99551691,205.3883941)(309.18301691,208.5133941)
\curveto(311.47468358,211.74256077)(312.98510024,215.75297744)(313.71426691,220.5446441)
\lineto(303.40176691,220.5446441)
\curveto(302.04760024,220.5446441)(300.79760024,220.3883941)(299.65176691,220.0758941)
\lineto(298.24551691,221.4821441)
\lineto(313.87051691,221.4821441)
\closepath
}
}
{
\newrgbcolor{curcolor}{0 0 0}
\pscustom[linestyle=none,fillstyle=solid,fillcolor=curcolor]
{
\newpath
\moveto(356.99551691,226.0133941)
\lineto(356.99551691,215.7008941)
\lineto(368.24551691,215.7008941)
\lineto(368.24551691,226.0133941)
\lineto(356.99551691,226.0133941)
\closepath
\moveto(354.65176691,226.9508941)
\curveto(354.65176691,231.0133941)(354.59968358,234.08631077)(354.49551691,236.1696441)
\lineto(358.24551691,234.4508941)
\lineto(356.99551691,233.2008941)
\lineto(356.99551691,226.9508941)
\lineto(367.93301691,226.9508941)
\lineto(369.18301691,228.8258941)
\lineto(371.83926691,226.7946441)
\lineto(370.43301691,225.5446441)
\lineto(370.43301691,216.4821441)
\curveto(370.43301691,215.33631077)(370.48510024,214.1383941)(370.58926691,212.8883941)
\lineto(368.24551691,211.9508941)
\lineto(368.24551691,214.7633941)
\lineto(356.99551691,214.7633941)
\curveto(356.99551691,206.84672744)(357.04760024,202.21131077)(357.15176691,200.8571441)
\lineto(354.49551691,199.4508941)
\curveto(354.59968358,201.84672744)(354.65176691,206.9508941)(354.65176691,214.7633941)
\lineto(343.87051691,214.7633941)
\lineto(343.87051691,212.8883941)
\lineto(341.52676691,211.6383941)
\curveto(341.63093358,213.5133941)(341.68301691,216.22172744)(341.68301691,219.7633941)
\curveto(341.68301691,223.30506077)(341.63093358,226.1696441)(341.52676691,228.3571441)
\lineto(343.87051691,226.9508941)
\lineto(354.65176691,226.9508941)
\closepath
\moveto(343.87051691,226.0133941)
\lineto(343.87051691,215.7008941)
\lineto(354.65176691,215.7008941)
\lineto(354.65176691,226.0133941)
\lineto(343.87051691,226.0133941)
\closepath
}
}
{
\newrgbcolor{curcolor}{0 0 0}
\pscustom[linestyle=none,fillstyle=solid,fillcolor=curcolor]
{
\newpath
\moveto(384.33926691,230.8571441)
\curveto(385.90176691,229.71131077)(386.99551691,228.6696441)(387.62051691,227.7321441)
\curveto(388.34968358,226.89881077)(388.34968358,225.96131077)(387.62051691,224.9196441)
\curveto(386.99551691,223.87797744)(386.52676691,224.03422744)(386.21426691,225.3883941)
\curveto(385.90176691,226.84672744)(385.17260024,228.56547744)(384.02676691,230.5446441)
\lineto(384.33926691,230.8571441)
\closepath
\moveto(395.12051691,231.6383941)
\lineto(397.93301691,229.6071441)
\curveto(397.30801691,229.39881077)(396.73510024,228.93006077)(396.21426691,228.2008941)
\curveto(395.69343358,227.47172744)(394.59968358,225.90922744)(392.93301691,223.5133941)
\lineto(392.46426691,223.8258941)
\curveto(393.92260024,227.15922744)(394.80801691,229.7633941)(395.12051691,231.6383941)
\closepath
\moveto(391.52676691,218.8258941)
\curveto(391.52676691,213.5133941)(391.57885024,209.55506077)(391.68301691,206.9508941)
\lineto(389.33926691,206.0133941)
\curveto(389.44343358,208.93006077)(389.49551691,212.78422744)(389.49551691,217.5758941)
\curveto(388.34968358,214.65922744)(386.37051691,211.69047744)(383.55801691,208.6696441)
\lineto(383.08926691,208.9821441)
\curveto(385.69343358,212.62797744)(387.51635024,216.74256077)(388.55801691,221.3258941)
\lineto(385.58926691,221.3258941)
\lineto(384.18301691,221.0133941)
\lineto(382.93301691,222.2633941)
\lineto(389.49551691,222.2633941)
\curveto(389.49551691,227.68006077)(389.44343358,232.15922744)(389.33926691,235.7008941)
\lineto(392.77676691,233.9821441)
\lineto(391.52676691,232.8883941)
\lineto(391.52676691,222.2633941)
\lineto(394.65176691,222.2633941)
\lineto(396.21426691,223.8258941)
\lineto(398.71426691,221.3258941)
\lineto(391.52676691,221.3258941)
\lineto(391.52676691,219.6071441)
\curveto(394.23510024,217.94047744)(396.00593358,216.6383941)(396.83926691,215.7008941)
\curveto(397.77676691,214.7633941)(397.93301691,213.72172744)(397.30801691,212.5758941)
\curveto(396.68301691,211.43006077)(396.00593358,211.69047744)(395.27676691,213.3571441)
\curveto(394.65176691,215.12797744)(393.40176691,216.9508941)(391.52676691,218.8258941)
\closepath
\moveto(380.58926691,206.0133941)
\lineto(380.58926691,222.8883941)
\curveto(380.58926691,225.28422744)(380.53718358,229.19047744)(380.43301691,234.6071441)
\lineto(383.87051691,232.8883941)
\lineto(382.62051691,231.7946441)
\lineto(382.62051691,205.2321441)
\lineto(391.83926691,205.2321441)
\lineto(393.55801691,206.9508941)
\lineto(396.21426691,204.2946441)
\lineto(382.77676691,204.2946441)
\lineto(381.52676691,202.8883941)
\lineto(379.33926691,204.9196441)
\lineto(380.58926691,206.0133941)
\closepath
\moveto(407.30801691,200.0758941)
\curveto(407.41218358,206.0133941)(407.46426691,212.99256077)(407.46426691,221.0133941)
\lineto(401.68301691,221.0133941)
\curveto(401.89135024,214.4508941)(401.26635024,209.7633941)(399.80801691,206.9508941)
\curveto(398.45385024,204.1383941)(396.16218358,201.58631077)(392.93301691,199.2946441)
\lineto(392.46426691,199.7633941)
\curveto(394.65176691,201.74256077)(396.26635024,203.6696441)(397.30801691,205.5446441)
\curveto(398.34968358,207.4196441)(398.97468358,209.50297744)(399.18301691,211.7946441)
\curveto(399.49551691,214.08631077)(399.65176691,217.31547744)(399.65176691,221.4821441)
\curveto(399.65176691,225.75297744)(399.59968358,229.6071441)(399.49551691,233.0446441)
\lineto(401.52676691,231.7946441)
\curveto(403.71426691,232.21131077)(405.53718358,232.68006077)(406.99551691,233.2008941)
\curveto(408.55801691,233.72172744)(409.75593358,234.34672744)(410.58926691,235.0758941)
\lineto(413.08926691,232.4196441)
\curveto(412.15176691,232.4196441)(411.21426691,232.31547744)(410.27676691,232.1071441)
\curveto(409.33926691,232.00297744)(406.47468358,231.6383941)(401.68301691,231.0133941)
\lineto(401.68301691,221.9508941)
\lineto(409.80801691,221.9508941)
\lineto(411.68301691,223.8258941)
\lineto(414.49551691,221.0133941)
\lineto(409.49551691,221.0133941)
\lineto(409.49551691,207.5758941)
\curveto(409.49551691,206.43006077)(409.54760024,204.2946441)(409.65176691,201.1696441)
\lineto(407.30801691,200.0758941)
\closepath
}
}
{
\newrgbcolor{curcolor}{0 0 0}
\pscustom[linewidth=1.39199996,linecolor=curcolor]
{
\newpath
\moveto(173.05083578,216.20244889)
\lineto(267.91342578,214.67605889)
\lineto(267.91342578,214.73055889)
}
}
{
\newrgbcolor{curcolor}{0 0 0}
\pscustom[linestyle=none,fillstyle=solid,fillcolor=curcolor]
{
\newpath
\moveto(187.83064135,209.78737921)
\lineto(171.20207893,216.2076737)
\lineto(188.02859251,222.08965671)
\curveto(185.2977965,218.50102698)(185.23324566,213.53207983)(187.83064135,209.78737921)
\closepath
}
}
{
\newrgbcolor{curcolor}{0 0 0}
\pscustom[linewidth=0.95699997,linecolor=curcolor]
{
\newpath
\moveto(187.83064135,209.78737921)
\lineto(171.20207893,216.2076737)
\lineto(188.02859251,222.08965671)
\curveto(185.2977965,218.50102698)(185.23324566,213.53207983)(187.83064135,209.78737921)
\closepath
}
}
\end{pspicture}

	\caption{单个中断程序的状态}
	\label{fig:interrupt_state}
\end{figure}

在多线程的场景中,线程之间的切换往往由线程调度器掌管。线程调度器不是硬件,而是
一段代码,负责实现线程的调度算法。时间片轮转,优先级不同的线程之间的抢占,甚至
是线程优先级的动态变化,都由线程调度器实现。一般而言,线程调度器是系统内核的一
重要组成部分。线程调度器工作的基础就是如图~所示的线程表结构

\begin{itemize}
	\item \emph{无中断}:当前中断没有触发。
	\item \emph{就绪}:中断触发但是有优先级更高的中断在执行。
	\item \emph{运行}:中断程序正在运行。
	\item \emph{阻塞}:中断程序被优先级更高的中断打断。
\end{itemize}

\subsection{通用的硬件实现}
\label{subsec:basic_hardware}

\subsection{Uppaal中的基本中断模型}
\label{subsec:basic_uppaal}

\section{带重入的中断模型}
\label{sec:reentrant}

\subsection{重入的硬件实现}
\label{subsec:reentrant_hardware}

\subsection{Uppaal中的重入中断模型}
\label{subsec:reentrant_uppaal}

\section{分段中断模型}
\label{sec:segment}

\subsection{软件的二次实现}
\label{subsec:segment_software}

\subsection{Uppaal中的分段中断模型}
\label{subsec:segment_uppaal}