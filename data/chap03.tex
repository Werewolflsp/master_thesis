%%% Local Variables:
%%% mode: latex
%%% TeX-master: t
%%% End:

\chapter{基于Uppaal的中断模型}
\label{cha:intr}

\section{Uppaal中的模型组成}
\label{sec:model_combine}	
一套Uppaal模型由以下三部分组成。
\begin{itemize}
	\item \emph{声明}:整个模型系统中共有的声明,可以是变量或函数。在整个
	系统中都可以访问。
	\item \emph{自动机模板}:各类自动机的通用模板,一个模型系统可以有多个
	模板,一个模板在系统中可以对应多个实例。
		\begin{enumerate}[(1)]
			\item \emph{声明}:模板内部的变量或函数,只有本模板的实例可
			以访问。
			\item \emph{位置}:时间自动机的位置,每个位置可以有初始(
			initial),紧急(urgent),关键(committed)。关键位置与紧
			急位置上,模型中的时钟都停止。不同的是,当有自动机在关键位置时,
			在下一个状态迁移必须从某一个关键位置发出。
			\item \emph{变迁}:位置到位置的迁移。变迁包含选择(select)
			、条件(guard)、同步(sync)、更新(update)四个属性。其中,
			同步和更新是同时发生的。
		\end{enumerate}	
	\item \emph{模型声明}:定义组成系统的模板实例。
\end{itemize}

\section{中断基本模型}
\label{sec:basic}
通常,在我们接触到的非实时性电脑环境中,中断的行为就是符合一个中断的基本模型。
其行为模式就是简单的抢占当前线程,在中断程序执行结束之后恢复上下文继续执行被抢
占的线程。

\begin{figure}
	\centering
	%LaTeX with PSTricks extensions
%%Creator: inkscape 0.91
%%Please note this file requires PSTricks extensions
\psset{xunit=.5pt,yunit=.5pt,runit=.5pt}
\begin{pspicture}(438.14290264,242.42857685)
{
\newrgbcolor{curcolor}{0 0 0}
\pscustom[linewidth=0.57914072,linecolor=curcolor]
{
\newpath
\moveto(300.71042076,199.7857195)
\curveto(300.71042076,223.17679387)(263.2000076,242.13900624)(216.92856804,242.13900624)
\curveto(170.65712849,242.13900624)(133.14671532,223.17679387)(133.14671532,199.7857195)
\curveto(133.14671532,176.39464514)(170.65712849,157.43243276)(216.92856804,157.43243276)
\curveto(263.2000076,157.43243276)(300.71042076,176.39464514)(300.71042076,199.7857195)
\closepath
}
}
{
\newrgbcolor{curcolor}{0 0 0}
\pscustom[linestyle=none,fillstyle=solid,fillcolor=curcolor]
{
\newpath
\moveto(217.19642083,184.4732195)
\curveto(216.46725416,183.2232195)(215.26933749,182.5982195)(213.60267083,182.5982195)
\lineto(210.32142083,182.5982195)
\curveto(208.23808749,182.5982195)(207.19642083,183.5357195)(207.19642083,185.4107195)
\lineto(207.19642083,204.4732195)
\lineto(205.16517083,204.4732195)
\curveto(205.26933749,198.74405283)(204.59225416,194.10863617)(203.13392083,190.5669695)
\curveto(201.77975416,187.1294695)(199.07142083,183.90030283)(195.00892083,180.8794695)
\lineto(194.54017083,181.3482195)
\curveto(197.76933749,184.1607195)(200.00892083,187.2857195)(201.25892083,190.7232195)
\curveto(202.61308749,194.26488617)(203.23808749,198.8482195)(203.13392083,204.4732195)
\lineto(201.41517083,204.4732195)
\lineto(200.00892083,204.1607195)
\lineto(198.75892083,205.4107195)
\lineto(203.13392083,205.4107195)
\curveto(203.13392083,212.18155283)(203.08183749,216.08780283)(202.97767083,217.1294695)
\lineto(206.25892083,215.5669695)
\lineto(205.16517083,214.4732195)
\lineto(205.16517083,205.4107195)
\lineto(212.50892083,205.4107195)
\lineto(214.38392083,207.2857195)
\lineto(217.04017083,204.4732195)
\lineto(209.22767083,204.4732195)
\lineto(209.22767083,186.1919695)
\curveto(209.22767083,184.9419695)(209.85267083,184.3169695)(211.10267083,184.3169695)
\lineto(212.35267083,184.3169695)
\curveto(213.49850416,184.3169695)(214.12350416,184.83780283)(214.22767083,185.8794695)
\curveto(214.33183749,187.02530283)(214.38392083,188.6919695)(214.38392083,190.8794695)
\lineto(215.16517083,190.8794695)
\lineto(215.47767083,187.1294695)
\curveto(215.68600416,185.8794695)(216.25892083,184.99405283)(217.19642083,184.4732195)
\closepath
\moveto(208.29017083,214.7857195)
\curveto(210.16517083,213.8482195)(211.46725416,213.01488617)(212.19642083,212.2857195)
\curveto(212.92558749,211.6607195)(213.18600416,210.93155283)(212.97767083,210.0982195)
\curveto(212.76933749,209.36905283)(212.40475416,208.79613617)(211.88392083,208.3794695)
\curveto(211.46725416,208.0669695)(211.05058749,208.58780283)(210.63392083,209.9419695)
\curveto(210.21725416,211.40030283)(209.33183749,212.85863617)(207.97767083,214.3169695)
\lineto(208.29017083,214.7857195)
\closepath
\moveto(186.88392083,204.4732195)
\lineto(186.88392083,198.8482195)
\lineto(195.47767083,198.8482195)
\lineto(195.47767083,204.4732195)
\lineto(186.88392083,204.4732195)
\closepath
\moveto(189.07142083,217.1294695)
\curveto(190.42558749,216.40030283)(191.46725416,215.67113617)(192.19642083,214.9419695)
\curveto(192.92558749,214.21280283)(193.18600416,213.5357195)(192.97767083,212.9107195)
\curveto(192.87350416,212.2857195)(192.56100416,211.71280283)(192.04017083,211.1919695)
\curveto(191.51933749,210.77530283)(191.05058749,211.24405283)(190.63392083,212.5982195)
\curveto(190.32142083,214.05655283)(189.64433749,215.46280283)(188.60267083,216.8169695)
\lineto(189.07142083,217.1294695)
\closepath
\moveto(181.72767083,210.4107195)
\lineto(196.72767083,210.4107195)
\lineto(198.44642083,212.1294695)
\lineto(201.10267083,209.4732195)
\lineto(186.88392083,209.4732195)
\curveto(185.32142083,209.4732195)(184.07142083,209.3169695)(183.13392083,209.0044695)
\lineto(181.72767083,210.4107195)
\closepath
\moveto(184.69642083,196.1919695)
\curveto(184.80058749,197.85863617)(184.85267083,199.57738617)(184.85267083,201.3482195)
\curveto(184.85267083,203.11905283)(184.80058749,204.88988617)(184.69642083,206.6607195)
\lineto(186.88392083,205.4107195)
\lineto(195.32142083,205.4107195)
\lineto(196.57142083,206.8169695)
\lineto(198.75892083,204.7857195)
\lineto(197.50892083,203.8482195)
\curveto(197.50892083,201.0357195)(197.56100416,198.79613617)(197.66517083,197.1294695)
\lineto(195.47767083,196.1919695)
\lineto(195.47767083,197.9107195)
\lineto(192.50892083,197.9107195)
\lineto(192.50892083,185.5669695)
\curveto(192.50892083,184.52530283)(192.24850416,183.74405283)(191.72767083,183.2232195)
\curveto(191.31100416,182.70238617)(190.58183749,182.23363617)(189.54017083,181.8169695)
\curveto(189.43600416,183.0669695)(188.08183749,184.10863617)(185.47767083,184.9419695)
\lineto(185.47767083,185.5669695)
\curveto(187.87350416,185.2544695)(189.33183749,185.0982195)(189.85267083,185.0982195)
\curveto(190.37350416,185.20238617)(190.63392083,185.5669695)(190.63392083,186.1919695)
\lineto(190.63392083,197.9107195)
\lineto(186.88392083,197.9107195)
\lineto(186.88392083,196.9732195)
\lineto(184.69642083,196.1919695)
\closepath
\moveto(188.75892083,193.0669695)
\lineto(187.35267083,192.2857195)
\curveto(185.26933749,189.05655283)(183.29017083,186.45238617)(181.41517083,184.4732195)
\lineto(180.94642083,184.7857195)
\curveto(182.09225416,186.55655283)(183.08183749,188.32738617)(183.91517083,190.0982195)
\curveto(184.85267083,191.86905283)(185.52975416,193.48363617)(185.94642083,194.9419695)
\lineto(188.75892083,193.0669695)
\closepath
\moveto(194.22767083,194.0044695)
\lineto(194.54017083,194.4732195)
\curveto(196.10267083,193.5357195)(197.19642083,192.7544695)(197.82142083,192.1294695)
\curveto(198.44642083,191.5044695)(198.75892083,190.82738617)(198.75892083,190.0982195)
\curveto(198.75892083,189.4732195)(198.44642083,188.79613617)(197.82142083,188.0669695)
\curveto(197.30058749,187.4419695)(196.88392083,187.85863617)(196.57142083,189.3169695)
\curveto(196.25892083,190.77530283)(195.47767083,192.33780283)(194.22767083,194.0044695)
\closepath
}
}
{
\newrgbcolor{curcolor}{0 0 0}
\pscustom[linestyle=none,fillstyle=solid,fillcolor=curcolor]
{
\newpath
\moveto(243.44642083,209.6294695)
\lineto(243.44642083,203.3794695)
\lineto(244.54017083,203.3794695)
\curveto(245.99850416,205.04613617)(247.56100416,207.1294695)(249.22767083,209.6294695)
\lineto(243.44642083,209.6294695)
\closepath
\moveto(240.16517083,196.0357195)
\lineto(240.16517083,191.3482195)
\lineto(250.47767083,191.3482195)
\lineto(250.47767083,196.0357195)
\lineto(240.16517083,196.0357195)
\closepath
\moveto(240.16517083,190.4107195)
\lineto(240.16517083,184.9419695)
\lineto(250.47767083,184.9419695)
\lineto(250.47767083,190.4107195)
\lineto(240.16517083,190.4107195)
\closepath
\moveto(237.97767083,180.8794695)
\curveto(238.08183749,183.48363617)(238.13392083,188.0669695)(238.13392083,194.6294695)
\curveto(235.21725416,192.54613617)(232.97767083,191.13988617)(231.41517083,190.4107195)
\lineto(231.10267083,190.8794695)
\curveto(234.01933749,192.96280283)(236.36308749,194.88988617)(238.13392083,196.6607195)
\curveto(238.13392083,197.18155283)(238.08183749,197.85863617)(237.97767083,198.6919695)
\lineto(239.22767083,197.7544695)
\curveto(240.89433749,199.3169695)(242.45683749,200.8794695)(243.91517083,202.4419695)
\lineto(237.97767083,202.4419695)
\curveto(236.62350416,202.4419695)(235.37350416,202.2857195)(234.22767083,201.9732195)
\lineto(232.82142083,203.3794695)
\lineto(241.41517083,203.3794695)
\lineto(241.41517083,209.6294695)
\lineto(239.85267083,209.6294695)
\curveto(238.49850416,209.6294695)(237.24850416,209.4732195)(236.10267083,209.1607195)
\lineto(234.69642083,210.5669695)
\lineto(241.41517083,210.5669695)
\curveto(241.41517083,214.0044695)(241.36308749,216.3482195)(241.25892083,217.5982195)
\lineto(244.69642083,216.0357195)
\lineto(243.44642083,214.9419695)
\lineto(243.44642083,210.5669695)
\lineto(245.16517083,210.5669695)
\lineto(247.19642083,212.2857195)
\lineto(249.38392083,210.0982195)
\curveto(250.21725416,211.3482195)(251.05058749,212.9107195)(251.88392083,214.7857195)
\lineto(254.85267083,212.5982195)
\curveto(254.01933749,212.38988617)(252.92558749,211.3482195)(251.57142083,209.4732195)
\curveto(250.32142083,207.5982195)(248.81100416,205.5669695)(247.04017083,203.3794695)
\lineto(252.19642083,203.3794695)
\lineto(254.38392083,205.4107195)
\lineto(257.04017083,202.4419695)
\lineto(246.41517083,202.4419695)
\curveto(244.64433749,200.5669695)(242.82142083,198.74405283)(240.94642083,196.9732195)
\lineto(250.16517083,196.9732195)
\lineto(251.25892083,198.5357195)
\lineto(253.75892083,196.5044695)
\lineto(252.50892083,195.5669695)
\lineto(252.50892083,185.5669695)
\curveto(252.50892083,184.21280283)(252.56100416,183.01488617)(252.66517083,181.9732195)
\lineto(250.47767083,181.0357195)
\lineto(250.47767083,184.0044695)
\lineto(240.16517083,184.0044695)
\lineto(240.16517083,181.8169695)
\lineto(237.97767083,180.8794695)
\closepath
\moveto(221.10267083,186.9732195)
\curveto(222.04017083,187.07738617)(223.65475416,187.33780283)(225.94642083,187.7544695)
\curveto(228.34225416,188.17113617)(231.36308749,188.74405283)(235.00892083,189.4732195)
\lineto(235.16517083,188.8482195)
\curveto(232.24850416,188.01488617)(229.69642083,187.23363617)(227.50892083,186.5044695)
\curveto(225.32142083,185.77530283)(223.70683749,184.9419695)(222.66517083,184.0044695)
\lineto(221.10267083,186.9732195)
\closepath
\moveto(228.44642083,217.1294695)
\lineto(231.57142083,215.2544695)
\curveto(230.73808749,214.83780283)(229.64433749,213.58780283)(228.29017083,211.5044695)
\curveto(226.93600416,209.52530283)(225.37350416,207.2857195)(223.60267083,204.7857195)
\lineto(230.79017083,204.9419695)
\curveto(231.62350416,206.29613617)(232.35267083,207.7544695)(232.97767083,209.3169695)
\lineto(235.47767083,207.2857195)
\curveto(234.33183749,206.45238617)(232.97767083,204.99405283)(231.41517083,202.9107195)
\curveto(229.95683749,200.82738617)(227.76933749,198.2232195)(224.85267083,195.0982195)
\curveto(227.76933749,195.51488617)(231.05058749,196.08780283)(234.69642083,196.8169695)
\lineto(234.85267083,196.1919695)
\curveto(232.14433749,195.35863617)(230.00892083,194.68155283)(228.44642083,194.1607195)
\curveto(226.88392083,193.63988617)(225.52975416,192.96280283)(224.38392083,192.1294695)
\lineto(222.35267083,195.4107195)
\curveto(223.18600416,195.7232195)(224.17558749,196.45238617)(225.32142083,197.5982195)
\curveto(226.46725416,198.74405283)(228.13392083,200.93155283)(230.32142083,204.1607195)
\curveto(228.44642083,203.95238617)(226.98808749,203.6919695)(225.94642083,203.3794695)
\curveto(224.90475416,203.17113617)(223.86308749,202.7544695)(222.82142083,202.1294695)
\lineto(221.10267083,204.9419695)
\curveto(222.04017083,205.04613617)(223.23808749,206.29613617)(224.69642083,208.6919695)
\curveto(226.15475416,211.08780283)(227.40475416,213.90030283)(228.44642083,217.1294695)
\closepath
}
}
{
\newrgbcolor{curcolor}{0 0 0}
\pscustom[linestyle=none,fillstyle=solid,fillcolor=curcolor]
{
\newpath
\moveto(61.52679148,62.75447382)
\lineto(75.12054148,62.75447382)
\lineto(77.30804148,64.78572382)
\lineto(80.12054148,61.81697382)
\lineto(65.74554148,61.81697382)
\curveto(65.01637481,61.81697382)(64.07887481,61.66072382)(62.93304148,61.34822382)
\lineto(61.52679148,62.75447382)
\closepath
\moveto(60.58929148,42.44197382)
\curveto(61.63095814,42.65030715)(62.62054148,43.32739048)(63.55804148,44.47322382)
\curveto(64.49554148,45.61905715)(65.48512481,47.07739048)(66.52679148,48.84822382)
\curveto(67.56845814,50.61905715)(68.29762481,52.23364048)(68.71429148,53.69197382)
\lineto(63.08929148,53.69197382)
\curveto(62.25595814,53.69197382)(61.26637481,53.53572382)(60.12054148,53.22322382)
\lineto(58.71429148,54.62947382)
\lineto(78.08929148,54.62947382)
\lineto(80.43304148,56.97322382)
\lineto(83.40179148,53.69197382)
\lineto(69.18304148,53.69197382)
\lineto(71.99554148,51.66072382)
\curveto(71.16220814,51.66072382)(69.91220814,50.51489048)(68.24554148,48.22322382)
\curveto(66.68304148,46.03572382)(64.96429148,44.00447382)(63.08929148,42.12947382)
\lineto(76.99554148,42.75447382)
\curveto(76.26637481,44.31697382)(75.12054148,45.98364048)(73.55804148,47.75447382)
\lineto(74.02679148,48.06697382)
\curveto(76.21429148,46.71280715)(77.88095814,45.41072382)(79.02679148,44.16072382)
\curveto(80.27679148,43.01489048)(80.79762481,41.92114048)(80.58929148,40.87947382)
\curveto(80.48512481,39.94197382)(80.17262481,39.26489048)(79.65179148,38.84822382)
\curveto(79.13095814,38.43155715)(78.71429148,38.63989048)(78.40179148,39.47322382)
\curveto(78.08929148,40.41072382)(77.77679148,41.24405715)(77.46429148,41.97322382)
\curveto(72.04762481,41.45239048)(68.40179148,40.98364048)(66.52679148,40.56697382)
\curveto(64.75595814,40.15030715)(63.40179148,39.62947382)(62.46429148,39.00447382)
\lineto(60.58929148,42.44197382)
\closepath
\moveto(51.37054148,65.56697382)
\lineto(51.83929148,65.87947382)
\curveto(53.92262481,64.83780715)(55.32887481,63.95239048)(56.05804148,63.22322382)
\curveto(56.78720814,62.49405715)(57.09970814,61.71280715)(56.99554148,60.87947382)
\curveto(56.89137481,60.15030715)(56.47470814,59.47322382)(55.74554148,58.84822382)
\curveto(55.12054148,58.32739048)(54.65179148,58.79614048)(54.33929148,60.25447382)
\curveto(54.02679148,61.71280715)(53.03720814,63.48364048)(51.37054148,65.56697382)
\closepath
\moveto(56.21429148,39.94197382)
\curveto(57.15179148,38.90030715)(58.45387481,37.91072382)(60.12054148,36.97322382)
\curveto(61.78720814,36.03572382)(64.18304148,35.46280715)(67.30804148,35.25447382)
\curveto(70.53720814,35.15030715)(73.55804148,35.15030715)(76.37054148,35.25447382)
\curveto(79.28720814,35.35864048)(82.04762481,35.61905715)(84.65179148,36.03572382)
\lineto(84.65179148,35.41072382)
\curveto(82.46429148,34.78572382)(81.37054148,33.90030715)(81.37054148,32.75447382)
\curveto(76.68304148,32.75447382)(72.77679148,32.80655715)(69.65179148,32.91072382)
\curveto(66.63095814,33.01489048)(64.13095814,33.48364048)(62.15179148,34.31697382)
\curveto(60.27679148,35.04614048)(58.76637481,35.93155715)(57.62054148,36.97322382)
\curveto(56.47470814,38.11905715)(55.64137481,38.74405715)(55.12054148,38.84822382)
\curveto(54.59970814,38.95239048)(53.71429148,38.37947382)(52.46429148,37.12947382)
\curveto(51.31845814,35.87947382)(50.48512481,34.88989048)(49.96429148,34.16072382)
\lineto(47.77679148,36.34822382)
\curveto(49.44345814,37.70239048)(51.57887481,38.90030715)(54.18304148,39.94197382)
\lineto(54.18304148,52.59822382)
\lineto(53.08929148,52.59822382)
\curveto(51.73512481,52.59822382)(50.48512481,52.44197382)(49.33929148,52.12947382)
\lineto(47.93304148,53.53572382)
\lineto(53.87054148,53.53572382)
\lineto(54.96429148,55.25447382)
\lineto(57.62054148,53.22322382)
\lineto(56.21429148,52.12947382)
\lineto(56.21429148,39.94197382)
\closepath
}
}
{
\newrgbcolor{curcolor}{0 0 0}
\pscustom[linestyle=none,fillstyle=solid,fillcolor=curcolor]
{
\newpath
\moveto(103.87054148,63.22322382)
\lineto(117.15179148,63.22322382)
\lineto(119.33929148,65.25447382)
\lineto(121.99554148,62.28572382)
\lineto(109.02679148,62.28572382)
\curveto(107.67262481,62.28572382)(106.42262481,62.12947382)(105.27679148,61.81697382)
\lineto(103.87054148,63.22322382)
\closepath
\moveto(105.27679148,35.41072382)
\curveto(108.08929148,35.20239048)(109.96429148,35.09822382)(110.90179148,35.09822382)
\curveto(111.83929148,35.09822382)(112.30804148,35.72322382)(112.30804148,36.97322382)
\lineto(112.30804148,52.91072382)
\lineto(106.37054148,52.91072382)
\curveto(105.01637481,52.91072382)(103.76637481,52.75447382)(102.62054148,52.44197382)
\lineto(101.21429148,53.84822382)
\lineto(119.18304148,53.84822382)
\lineto(121.52679148,56.03572382)
\lineto(124.18304148,52.91072382)
\lineto(114.49554148,52.91072382)
\lineto(114.49554148,36.19197382)
\curveto(114.59970814,33.37947382)(113.29762481,31.71280715)(110.58929148,31.19197382)
\curveto(110.58929148,32.65030715)(108.81845814,33.79614048)(105.27679148,34.62947382)
\lineto(105.27679148,35.41072382)
\closepath
\moveto(97.62054148,67.12947382)
\lineto(100.43304148,64.94197382)
\curveto(99.70387481,64.73364048)(98.45387481,63.58780715)(96.68304148,61.50447382)
\curveto(94.91220814,59.52530715)(92.67262481,57.54614048)(89.96429148,55.56697382)
\lineto(89.65179148,56.03572382)
\curveto(91.42262481,57.70239048)(93.03720814,59.57739048)(94.49554148,61.66072382)
\curveto(96.05804148,63.84822382)(97.09970814,65.67114048)(97.62054148,67.12947382)
\closepath
\moveto(97.93304148,49.31697382)
\lineto(97.93304148,32.28572382)
\lineto(95.58929148,30.87947382)
\curveto(95.69345814,33.17114048)(95.74554148,39.52530715)(95.74554148,49.94197382)
\curveto(93.34970814,47.44197382)(91.00595814,45.41072382)(88.71429148,43.84822382)
\lineto(88.40179148,44.31697382)
\curveto(90.58929148,46.50447382)(92.67262481,48.95239048)(94.65179148,51.66072382)
\curveto(96.73512481,54.36905715)(98.29762481,56.97322382)(99.33929148,59.47322382)
\lineto(101.99554148,57.12947382)
\lineto(100.58929148,56.34822382)
\curveto(99.23512481,54.36905715)(98.08929148,52.80655715)(97.15179148,51.66072382)
\lineto(98.87054148,50.25447382)
\lineto(97.93304148,49.31697382)
\closepath
}
}
{
\newrgbcolor{curcolor}{0 0 0}
\pscustom[linewidth=0.57914072,linecolor=curcolor]
{
\newpath
\moveto(167.85329141,48.35713495)
\curveto(167.85329141,71.74820931)(130.34287825,90.71042169)(84.07143869,90.71042169)
\curveto(37.79999914,90.71042169)(0.28958597,71.74820931)(0.28958597,48.35713495)
\curveto(0.28958597,24.96606059)(37.79999914,6.00384821)(84.07143869,6.00384821)
\curveto(130.34287825,6.00384821)(167.85329141,24.96606059)(167.85329141,48.35713495)
\closepath
}
}
{
\newrgbcolor{curcolor}{0 0 0}
\pscustom[linestyle=none,fillstyle=solid,fillcolor=curcolor]
{
\newpath
\moveto(325.09818896,48.13394253)
\curveto(326.86902229,46.46727587)(328.17110562,44.90477587)(329.00443896,43.44644253)
\curveto(329.83777229,42.09227587)(330.20235562,40.5818592)(330.09818896,38.91519253)
\curveto(330.09818896,37.35269253)(329.68152229,36.10269253)(328.84818896,35.16519253)
\curveto(328.01485562,34.22769253)(326.76485562,33.60269253)(325.09818896,33.29019253)
\curveto(325.09818896,34.6443592)(324.05652229,35.63394253)(321.97318896,36.25894253)
\lineto(321.97318896,36.88394253)
\curveto(323.84818896,36.6756092)(325.20235562,36.57144253)(326.03568896,36.57144253)
\curveto(326.97318896,36.6756092)(327.54610562,37.40477587)(327.75443896,38.75894253)
\curveto(328.06693896,40.21727587)(327.96277229,41.57144253)(327.44193896,42.82144253)
\curveto(327.02527229,44.1756092)(325.98360562,45.94644253)(324.31693896,48.13394253)
\lineto(326.97318896,57.66519253)
\lineto(321.19193896,57.66519253)
\lineto(321.19193896,26.57144253)
\lineto(319.00443896,25.32144253)
\curveto(319.10860562,29.69644253)(319.16068896,36.2068592)(319.16068896,44.85269253)
\curveto(319.16068896,53.49852587)(319.10860562,58.5506092)(319.00443896,60.00894253)
\lineto(321.19193896,58.60269253)
\lineto(326.97318896,58.60269253)
\lineto(328.06693896,60.16519253)
\lineto(330.72318896,58.13394253)
\curveto(329.68152229,57.50894253)(328.79610562,56.41519253)(328.06693896,54.85269253)
\curveto(327.33777229,53.3943592)(326.34818896,51.15477587)(325.09818896,48.13394253)
\closepath
\moveto(335.41068896,57.35269253)
\lineto(335.41068896,48.75894253)
\lineto(345.41068896,48.75894253)
\lineto(345.41068896,57.35269253)
\lineto(335.41068896,57.35269253)
\closepath
\moveto(335.41068896,47.82144253)
\lineto(335.41068896,39.22769253)
\lineto(345.41068896,39.22769253)
\lineto(345.41068896,47.82144253)
\lineto(335.41068896,47.82144253)
\closepath
\moveto(335.41068896,38.29019253)
\lineto(335.41068896,28.75894253)
\lineto(345.41068896,28.75894253)
\lineto(345.41068896,38.29019253)
\lineto(335.41068896,38.29019253)
\closepath
\moveto(326.66068896,28.75894253)
\lineto(333.37943896,28.75894253)
\curveto(333.37943896,46.57144253)(333.32735562,56.88394253)(333.22318896,59.69644253)
\lineto(335.41068896,58.29019253)
\lineto(345.25443896,58.29019253)
\lineto(346.50443896,60.00894253)
\lineto(349.00443896,57.97769253)
\lineto(347.59818896,56.72769253)
\lineto(347.59818896,28.75894253)
\lineto(348.37943896,28.75894253)
\lineto(350.56693896,30.94644253)
\lineto(353.37943896,27.82144253)
\lineto(331.81693896,27.82144253)
\curveto(330.46277229,27.82144253)(329.21277229,27.66519253)(328.06693896,27.35269253)
\lineto(326.66068896,28.75894253)
\closepath
}
}
{
\newrgbcolor{curcolor}{0 0 0}
\pscustom[linestyle=none,fillstyle=solid,fillcolor=curcolor]
{
\newpath
\moveto(370.87943896,45.00894253)
\lineto(370.87943896,41.41519253)
\lineto(378.22318896,41.41519253)
\lineto(378.22318896,45.00894253)
\lineto(370.87943896,45.00894253)
\closepath
\moveto(385.87943896,52.66519253)
\lineto(387.59818896,55.47769253)
\lineto(362.59818896,55.47769253)
\curveto(362.80652229,54.3318592)(362.70235562,53.3943592)(362.28568896,52.66519253)
\curveto(361.86902229,51.93602587)(361.08777229,51.57144253)(359.94193896,51.57144253)
\curveto(358.90027229,51.6756092)(358.84818896,52.24852587)(359.78568896,53.29019253)
\curveto(360.72318896,54.43602587)(361.34818896,56.0506092)(361.66068896,58.13394253)
\lineto(362.28568896,58.13394253)
\lineto(362.44193896,56.41519253)
\lineto(387.44193896,56.41519253)
\lineto(388.69193896,57.82144253)
\lineto(391.03568896,55.16519253)
\curveto(389.57735562,55.06102587)(388.01485562,54.12352587)(386.34818896,52.35269253)
\lineto(385.87943896,52.66519253)
\closepath
\moveto(371.66068896,61.41519253)
\lineto(371.97318896,61.88394253)
\curveto(374.36902229,61.0506092)(375.77527229,60.2693592)(376.19193896,59.54019253)
\curveto(376.71277229,58.91519253)(376.60860562,58.13394253)(375.87943896,57.19644253)
\curveto(375.25443896,56.25894253)(374.73360562,56.3631092)(374.31693896,57.50894253)
\curveto(374.00443896,58.75894253)(373.11902229,60.06102587)(371.66068896,61.41519253)
\closepath
\moveto(361.19193896,50.79019253)
\lineto(368.84818896,50.79019253)
\curveto(368.84818896,52.1443592)(368.79610562,53.5506092)(368.69193896,55.00894253)
\lineto(372.12943896,53.60269253)
\lineto(370.87943896,52.66519253)
\lineto(370.87943896,50.79019253)
\lineto(378.22318896,50.79019253)
\curveto(378.22318896,52.1443592)(378.17110562,53.5506092)(378.06693896,55.00894253)
\lineto(381.66068896,53.60269253)
\lineto(380.25443896,52.50894253)
\lineto(380.25443896,50.79019253)
\lineto(383.06693896,50.79019253)
\lineto(384.94193896,52.66519253)
\lineto(387.75443896,49.85269253)
\lineto(380.25443896,49.85269253)
\lineto(380.25443896,45.94644253)
\lineto(383.22318896,45.94644253)
\lineto(385.09818896,47.82144253)
\lineto(387.91068896,45.00894253)
\lineto(380.25443896,45.00894253)
\lineto(380.25443896,41.41519253)
\lineto(386.81693896,41.41519253)
\lineto(389.00443896,43.60269253)
\lineto(392.12943896,40.47769253)
\lineto(379.78568896,40.47769253)
\curveto(381.55652229,38.29019253)(383.63985562,36.62352587)(386.03568896,35.47769253)
\curveto(388.53568896,34.43602587)(390.77527229,33.8631092)(392.75443896,33.75894253)
\lineto(392.75443896,33.13394253)
\curveto(391.29610562,33.02977587)(390.30652229,32.40477587)(389.78568896,31.25894253)
\curveto(387.59818896,32.19644253)(385.51485562,33.3943592)(383.53568896,34.85269253)
\curveto(381.66068896,36.41519253)(380.09818896,38.29019253)(378.84818896,40.47769253)
\lineto(370.56693896,40.47769253)
\curveto(369.21277229,38.0818592)(367.28568896,35.94644253)(364.78568896,34.07144253)
\curveto(362.38985562,32.19644253)(359.73360562,30.68602587)(356.81693896,29.54019253)
\lineto(356.50443896,30.00894253)
\curveto(359.52527229,31.6756092)(361.97318896,33.3943592)(363.84818896,35.16519253)
\curveto(365.82735562,37.04019253)(367.23360562,38.81102587)(368.06693896,40.47769253)
\lineto(362.59818896,40.47769253)
\curveto(361.24402229,40.47769253)(359.99402229,40.32144253)(358.84818896,40.00894253)
\lineto(357.44193896,41.41519253)
\lineto(368.84818896,41.41519253)
\lineto(368.84818896,45.00894253)
\lineto(367.28568896,45.00894253)
\curveto(365.93152229,45.00894253)(364.68152229,44.85269253)(363.53568896,44.54019253)
\lineto(362.12943896,45.94644253)
\lineto(368.84818896,45.94644253)
\lineto(368.84818896,49.85269253)
\lineto(366.34818896,49.85269253)
\curveto(364.99402229,49.85269253)(363.74402229,49.69644253)(362.59818896,49.38394253)
\lineto(361.19193896,50.79019253)
\closepath
\moveto(370.87943896,49.85269253)
\lineto(370.87943896,45.94644253)
\lineto(378.22318896,45.94644253)
\lineto(378.22318896,49.85269253)
\lineto(370.87943896,49.85269253)
\closepath
\moveto(366.03568896,34.85269253)
\lineto(373.69193896,34.85269253)
\curveto(373.69193896,36.31102587)(373.63985562,37.82144253)(373.53568896,39.38394253)
\lineto(377.12943896,37.82144253)
\lineto(375.87943896,36.72769253)
\lineto(375.87943896,34.85269253)
\lineto(379.47318896,34.85269253)
\lineto(381.19193896,36.57144253)
\lineto(383.84818896,33.91519253)
\lineto(375.87943896,33.91519253)
\lineto(375.87943896,27.97769253)
\lineto(383.37943896,27.97769253)
\lineto(385.56693896,30.16519253)
\lineto(388.69193896,27.04019253)
\lineto(365.56693896,27.04019253)
\curveto(364.21277229,27.04019253)(362.96277229,26.88394253)(361.81693896,26.57144253)
\lineto(360.41068896,27.97769253)
\lineto(373.69193896,27.97769253)
\lineto(373.69193896,33.91519253)
\lineto(371.19193896,33.91519253)
\curveto(369.83777229,33.91519253)(368.58777229,33.75894253)(367.44193896,33.44644253)
\lineto(366.03568896,34.85269253)
\closepath
}
}
{
\newrgbcolor{curcolor}{0 0 0}
\pscustom[linewidth=0.57914072,linecolor=curcolor]
{
\newpath
\moveto(437.85330828,42.64285367)
\curveto(437.85330828,66.03392803)(400.34289511,84.99614041)(354.07145556,84.99614041)
\curveto(307.80001601,84.99614041)(270.28960284,66.03392803)(270.28960284,42.64285367)
\curveto(270.28960284,19.2517793)(307.80001601,0.28956692)(354.07145556,0.28956692)
\curveto(400.34289511,0.28956692)(437.85330828,19.2517793)(437.85330828,42.64285367)
\closepath
}
}
{
\newrgbcolor{curcolor}{0 0 0}
\pscustom[linewidth=0.94413179,linecolor=curcolor]
{
\newpath
\moveto(160.57185526,167.71163985)
\lineto(93.34127526,90.43281985)
\lineto(93.34127526,93.19276985)
}
}
{
\newrgbcolor{curcolor}{0 0 0}
\pscustom[linestyle=none,fillstyle=solid,fillcolor=curcolor]
{
\newpath
\moveto(150.78658032,162.84641486)
\lineto(161.38226153,168.6685066)
\lineto(157.08261526,157.36902604)
\curveto(156.4138523,160.35360553)(153.86412532,162.5579574)(150.78658032,162.84641486)
\closepath
}
}
{
\newrgbcolor{curcolor}{0 0 0}
\pscustom[linewidth=0.64909061,linecolor=curcolor]
{
\newpath
\moveto(150.78658032,162.84641486)
\lineto(161.38226153,168.6685066)
\lineto(157.08261526,157.36902604)
\curveto(156.4138523,160.35360553)(153.86412532,162.5579574)(150.78658032,162.84641486)
\closepath
}
}
{
\newrgbcolor{curcolor}{0 0 0}
\pscustom[linewidth=0.90990525,linecolor=curcolor]
{
\newpath
\moveto(124.27171526,86.00716985)
\lineto(189.51007526,159.97643985)
\lineto(189.51007526,157.33469985)
}
}
{
\newrgbcolor{curcolor}{0 0 0}
\pscustom[linestyle=none,fillstyle=solid,fillcolor=curcolor]
{
\newpath
\moveto(133.73385302,90.63192623)
\lineto(123.48444863,85.09031189)
\lineto(127.70201812,95.95179859)
\curveto(128.32700579,93.07110875)(130.7698293,90.93004464)(133.73385302,90.63192623)
\closepath
}
}
{
\newrgbcolor{curcolor}{0 0 0}
\pscustom[linewidth=0.62555986,linecolor=curcolor]
{
\newpath
\moveto(133.73385302,90.63192623)
\lineto(123.48444863,85.09031189)
\lineto(127.70201812,95.95179859)
\curveto(128.32700579,93.07110875)(130.7698293,90.93004464)(133.73385302,90.63192623)
\closepath
}
}
{
\newrgbcolor{curcolor}{0 0 0}
\pscustom[linewidth=0.94413179,linecolor=curcolor]
{
\newpath
\moveto(160.57185526,167.71163985)
\lineto(93.34127526,90.43281985)
\lineto(93.34127526,93.19276985)
}
}
{
\newrgbcolor{curcolor}{0 0 0}
\pscustom[linestyle=none,fillstyle=solid,fillcolor=curcolor]
{
\newpath
\moveto(150.78658032,162.84641486)
\lineto(161.38226153,168.6685066)
\lineto(157.08261526,157.36902604)
\curveto(156.4138523,160.35360553)(153.86412532,162.5579574)(150.78658032,162.84641486)
\closepath
}
}
{
\newrgbcolor{curcolor}{0 0 0}
\pscustom[linewidth=0.64909061,linecolor=curcolor]
{
\newpath
\moveto(150.78658032,162.84641486)
\lineto(161.38226153,168.6685066)
\lineto(157.08261526,157.36902604)
\curveto(156.4138523,160.35360553)(153.86412532,162.5579574)(150.78658032,162.84641486)
\closepath
}
}
{
\newrgbcolor{curcolor}{0 0 0}
\pscustom[linewidth=1,linecolor=curcolor]
{
\newpath
\moveto(163.35714526,61.92858985)
\lineto(274.78571526,60.50001985)
}
}
{
\newrgbcolor{curcolor}{0 0 0}
\pscustom[linestyle=none,fillstyle=solid,fillcolor=curcolor]
{
\newpath
\moveto(264.15302842,65.07380957)
\lineto(276.11385039,60.50060853)
\lineto(264.03971683,56.2355486)
\curveto(265.99305666,58.81998723)(266.02775873,62.3897658)(264.15302842,65.07380957)
\closepath
}
}
{
\newrgbcolor{curcolor}{0 0 0}
\pscustom[linewidth=0.6875,linecolor=curcolor]
{
\newpath
\moveto(264.15302842,65.07380957)
\lineto(276.11385039,60.50060853)
\lineto(264.03971683,56.2355486)
\curveto(265.99305666,58.81998723)(266.02775873,62.3897658)(264.15302842,65.07380957)
\closepath
}
}
{
\newrgbcolor{curcolor}{0 0 0}
\pscustom[linewidth=0.98835522,linecolor=curcolor]
{
\newpath
\moveto(270.45302526,34.79508985)
\lineto(162.39879526,33.35601985)
}
}
{
\newrgbcolor{curcolor}{0 0 0}
\pscustom[linestyle=none,fillstyle=solid,fillcolor=curcolor]
{
\newpath
\moveto(173.02221727,29.11167449)
\lineto(161.08658952,33.32113284)
\lineto(172.9058811,37.84695908)
\curveto(171.04270228,35.24303385)(171.10062248,31.71513329)(173.02221727,29.11167449)
\closepath
}
}
{
\newrgbcolor{curcolor}{0 0 0}
\pscustom[linewidth=0.67949421,linecolor=curcolor]
{
\newpath
\moveto(173.02221727,29.11167449)
\lineto(161.08658952,33.32113284)
\lineto(172.9058811,37.84695908)
\curveto(171.04270228,35.24303385)(171.10062248,31.71513329)(173.02221727,29.11167449)
\closepath
}
}
\end{pspicture}

	\caption{单个线程的状态}
	\label{fig:thread_state}
\end{figure}

单独的中断程序的行为,与多线程程序研究中的线程行为十分相似。如图~
\ref{fig:thread_state}所示,通常,一个线程会被刻画为以下三个状态。
\begin{itemize}
	\item \emph{就绪}:线程可以运行但是当前并不占有CPU。
	\item \emph{运行}:线程正在运行。
	\item \emph{阻塞}:线程在等待处理器以外的资源,暂时无法运行。
\end{itemize}
由于绝大部分多线程程序研究的场景中,并不关心线程产生和终止。换言之,线程在这类
应用场景里直接存在,且永不终止。线程的切换由一个软件层面的调度器控制。

中断研究中,一个中断程序通常具有一个从产生到终止的完整周期。而且,一个中断并非
只触发一次,因此一个中断程序可能重复多次上述周期。这与传统的多线程程序研究是不
同的。

\subsection{通用的硬件实现}
\label{subsec:basic_hardware}

\subsection{Uppaal中的基本中断模型}
\label{subsec:basic_uppaal}

\section{带重入的中断模型}
\label{sec:reentrant}

\subsection{重入的硬件实现}
\label{subsec:reentrant_hardware}

\subsection{Uppaal中的重入中断模型}
\label{subsec:reentrant_uppaal}

\section{分段中断模型}
\label{sec:segment}

\subsection{软件的二次实现}
\label{subsec:segment_software}

\subsection{Uppaal中的分段中断模型}
\label{subsec:segment_uppaal}