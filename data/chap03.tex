%%% Local Variables:
%%% mode: latex
%%% TeX-master: t
%%% End:

\chapter{基于Uppaal的中断模型}

\section{Uppaal中的模型组成}	
一套Uppaal模型由以下三部分组成。
\begin{itemize}
	\item \emph{声明}:整个模型系统中共有的声明,可以是变量或函数。在整个
	系统中都可以访问。
	\item \emph{自动机模板}:各类自动机的通用模板,一个模型系统可以有多个
	模板,一个模板在系统中可以对应多个实例。
		\begin{enumerate}[(1)]
			\item \emph{声明}:模板内部的变量或函数,只有本模板的实例可
			以访问。
			\item \emph{位置}:时间自动机的位置,每个位置可以有初始(
			initial),紧急(urgent),关键(committed)。关键位置与紧
			急位置上,模型中的时钟都停止。不同的是,当有自动机在关键位置时,
			在下一个状态迁移必须从某一个关键位置发出。
			\item \emph{变迁}:位置到位置的迁移。变迁包含选择(select)
			、条件(guard)、同步(sync)、更新(update)四个属性。其中,
			同步和更新是同时发生的。
		\end{enumerate}	
	\item \emph{模型声明}:定义组成系统的模板实例。
\end{itemize}

\section{中断基本模型}
\subsection{通用的硬件实现}
\subsection{Uppaal中的基本中断模型}

\section{带重入的中断模型}
\subsection{重入的硬件实现}
\subsection{Uppaal中的重入中断模型}

\section{分段中断模型}
\subsection{软件的二次实现}
\subsection{Uppaal中的分段中断模型}