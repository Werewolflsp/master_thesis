%%% Local Variables:
%%% mode: latex
%%% TeX-master: t
%%% End:

\chapter{相关工作}
\label{cha:related_work}

%  × 时间自动机
%        秒表自动机(stopwatch automata) 
%		 CCS parallele composition
%  × 中断驱动程序验证
%  × 实时性分析

时间自动机(Timed Automata,TA)是由Alur R.和D.L. Dill在上世纪九十年代出
提出来的理论。他们用时间自动机来描述实时系统在时间上的行为。他们从形式化语言的角
度定义了时间自动机的语法和语义\cite{Alur:1994:TTA:180782.180519}。该理论
是对传统自动机理论\cite{Hopcroft:2006:IAT:1196416}的一个重要扩展。得益于
时间自动机的提出,自动机模型以及模型检测技术\cite{Clarke:2000:MC:332656}
才能将时间纳入系统的考虑。

在时间自动机理论中,时钟是一个连续的实数变量。同一个时间自动机系统里的时钟在每个
位置都均匀的增长。时钟的值在一些变迁中可以被重置为0,或者在某些变迁中用时钟的值
的范围作为条件。因此,时间自动机用来描述并发的依赖时间的行为是非常理想的。然而,
在某些情况下,例如抢占式调度,我们需要知道系统在某些位置停留的时间。时钟变量只能
被重置却不能被赋值。这就导致了对时间自动机的扩展。在这项扩展力,时钟在某些位置的
的导数可以是0,即意味着该时钟在该位置下停止计时。这样的自动机被称为聚合图(Integration
Graph)\cite{Kesten:1999:DIG:302392.302397}, 被作为时段演算(Duration 
Calculus)\cite{DBLP:journals/ipl/ChaochenHR91}的模型进行研究。其研究成
果包括此类自动机的可达性问题的不可判定\cite{Alur04decisionproblems},以及
在这类自动机的一个特殊子类别上利用基于缩减问题到线性约束满足性问题\cite{Apt:2003:PCP:1237975}
时有一个可判定的过程。在混合自动机\cite{Henzinger96thetheory}的一项判定性
研究\cite{McManis:1994:SAD:647763.735660}中,提到了类似的自动机。同时,
一项近似验证算法的实现工作\cite{Cassez:2000:IPS:646735.701625}里也提到了
该自动机。

该自动机由于其时钟暂停的特性有一个形象的名字\pozhehao 秒表自动机(Stopwatch
Automata,SWA)。除了上述的理论性研究以外,还有人做了应用方面的探索
\cite{Abdeddaim:2002:PJS:646486.694487}。

% SWA definition here
\begin{definition}
content...
\end{definition}

在秒表自动机的基础上,B{\'e}rard等人针对单执行资源多任务运行的场景提出了中断
时间自动机(Interrupt Timed Automata,ITA)。\cite{Berard:2012:ITA:2158996.2159045}
正如其命名,中断时间自动机在描述中断时非常合适。然而,中断时间自动机要求每个优先
级对应一个时钟,且高优先级的时钟的出现会导致低优先级的时钟停止。这对自动机的语义
表达的限制与本文需要描述的行为相冲突。同时,中断时间自动机对本文将要涉及的部分中
断机制不能很好地描述,因此,本文不采用此类自动机。

用一个单独的状态机来描述一个系统是非常困难的。伴随着时间自动机理论,时间自动机网
络(Network of Timed Automata,NTA)也被提出。

来自大阪大学的Makoto Higashi等人提出了一套在一个CPU模拟器上测试中断的方法以
检测中断带来的潜在数据竞争。该方法包含两个方面:第一是在可能产生数据竞争的指令处
触发中断,第二是人为修改外部中断给中断处理程序的输入。中断会在每次读或写内存的指
令之后触发,以此覆盖所有可能产生数据竞争的情况。外部中断的输入序列则是在程序运行
之前就手动准备好。他们已经在\emph{uClinux}上应用此方法做过实验。\cite{Higashi:2010:EMC:1808266.1808278}

