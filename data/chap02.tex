%%% Local Variables:
%%% mode: latex
%%% TeX-master: t
%%% End:

\chapter{相关工作}
\label{cha:related_work}

%  × 时间自动机
%        秒表自动机(stopwatch automata) 
%		 CCS parallele composition
%  × 中断驱动程序验证
%  × 实时性分析

时间自动机是由Alur R.和D.L. Dill在上世纪九十年代出提出来的理论。他们用时间自动
机来描述实时系统在时间上的行为。他们从形式化语言的角度定义了时间自动机的语法和语
义\cite{Alur:1994:TTA:180782.180519}。该理论是对传统自动机理论\cite{Hopcroft:2006:IAT:1196416}
的一个重要扩展。得益于时间自动机的提出,自动机模型以及模型检测技术\cite{Clarke:2000:MC:332656}
才能将时间纳入系统的考虑。

在时间自动机理论中,时钟是一个连续的实数变量。同一个时间自动机系统里的时钟在每个
位置都均匀的增长。时钟的值在一些变迁中可以被重置为0,或者在某些变迁中用时钟的值
的范围作为条件。因此,时间自动机用来描述并发的依赖时间的行为是非常理想的。然而,
在某些情况下,例如抢占式调度,我们需要知道系统在某些位置停留的时间。时钟变量只能
被重置却不能被赋值。这就导致了对时间自动机的扩展。在这项扩展力,时钟在某些位置的
的导数可以是0,即意味着该时钟在该位置下停止计时。这样的自动机被称为聚合图(Integration
Graph)\cite{Kesten:1999:DIG:302392.302397}, 被作为时段演算(Duration Calculus)
\cite{DBLP:journals/ipl/ChaochenHR91}的模型进行研究。其研究成果包括此类自动机
的可达性问题的不可判定\cite{Alur04decisionproblems},以及在这类自动机的一个特
殊子类别上利用基于缩减问题到线性约束满足性问题\cite{Apt:2003:PCP:1237975}时有
一个可判定的过程。在混合自动机\cite{Henzinger96thetheory}的一项判定性研究
\cite{McManis:1994:SAD:647763.735660}中,提到了类似的自动机。同时,一项近似验
证算法的实现工作\cite{Cassez:2000:IPS:646735.701625}里也提到了该自动机。

该自动机由于其时钟暂停的特性有一个形象的名字\pozhehao 秒表自动机。除了上述的理
论性研究以外,还有人做了应用方面的探索\cite{Abdeddaim:2002:PJS:646486.694487}。

来自大阪大学的Makoto Higashi等人提出了一套在一个CPU模拟器上测试中断的方法以
检测中断带来的潜在数据竞争。该方法包含两个方面:第一是在可能产生数据竞争的指令处
触发中断,第二是人为修改外部中断给中断处理程序的输入。中断会在每次读或写内存的指
令之后触发,以此覆盖所有可能产生数据竞争的情况。外部中断的输入序列则是在程序运行
之前就手动准备好。他们已经在\emph{uClinux}上应用此方法做过实验。\cite{Higashi:2010:EMC:1808266.1808278}

