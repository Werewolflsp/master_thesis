%%% Local Variables:
%%% mode: latex
%%% TeX-master: t
%%% End:

\chapter{相关工作}
\label{cha:related_work}

%  × 时间自动机
%        秒表自动机(stopwatch automata) 
%		 CCS parallele composition
%  × 实时性分析
%  × 中断驱动程序验证

时间自动机(Timed Automata,TA)是由Alur R.和D.L. Dill在上世纪九十年代出提出
来的理论。他们用时间自动机来描述实时系统在时间上的行为。他们从形式化语言的角度定
义了时间自动机的语法和语义\cite{Alur:1994:TTA:180782.180519}。该理论是对传统自
动机理论\cite{Hopcroft:2006:IAT:1196416}的一个重要扩展。得益于时间自动机的提出,
自动机模型以及模型检测技术\cite{Clarke:2000:MC:332656}才能将时间纳入系统的考虑。

我们用$X$表示一个变量的有限集,这些变量被称为时钟。
\begin{definition}
	\label{def:constraint}
	对于一个时钟的集合$X$,时钟约束集合$\varPhi(X)$的元素$\delta$的归纳定义如下
	\begin{equation}
		\delta := x \bowtie c ~|~ x-y \bowtie c~|~ \neg\delta ~|~ \delta_1 
		\wedge \delta_2
	\end{equation}
	其中,$x\in X$,$c\in Z$,$\bowtie\in \{<,\leq,=,\geq,>\}$。
\end{definition}
我们用$\varPhi_{df}(X)$表示$\varPhi(X)$的一个子集,在该子集中,不允许$x-y 
\bowtie c$这类约束。类似的,我们定义$\varPhi_{up}(X)$表示$\varPhi(X)$的另一个
子集,在该子集中,只有规定上界的约束$x\prec c$被允许,其中$\prec \in \{<,\leq\}$。
接着,我们给出时间自动机的定义。

% TA definition here
\begin{definition}
	\label{def:TA}
	一个定义在$\varSigma$上的时间自动机$A$是指一个元组$(L,l_0,X,\varSigma,E,Inv)$。
	\begin{itemize}
		\item $L$表示位置的有限集。
		\item $l_0\in L$,表示初始位置集合。
		\item $X$表示时钟的有限集。
		\item $\varSigma$表示动作的有限字母表。
		\item $E\subseteq L\times \varPhi_{df}(X)\times \varSigma \times 2^X\times L$,
		表示边的有限集。一条边$(l,\delta,a,r,l')\subseteq E$,
		或者$(l\stackrel{\delta,a,r}{\longrightarrow}l')$,
		表示从位置$l$到$l'$的,标签为$a$,约束为$\delta$,重置为$r\in 2^X$的一
		条变迁。
		\item $Inv\subseteq \varPhi_{ub}(X)^L$,与一个位置绑定,
		表示一个时钟上限约束。
	\end{itemize}
\end{definition}

在时间自动机理论中,时钟是一个连续的实数变量。同一个时间自动机系统里的时钟在每个
位置都均匀的增长。时钟的值在一些变迁中可以被重置为0,或者在某些变迁中用时钟的值
的范围作为条件。因此,时间自动机用来描述并发的依赖时间的行为是非常理想的。然而,
在某些情况下,例如抢占式调度,我们需要知道系统在某些位置停留的时间。时钟变量只能
被重置却不能被赋值。这就导致了对时间自动机的扩展。在这项扩展力,时钟在某些位置的
的导数可以是0,即意味着该时钟在该位置下停止计时。这样的自动机被称为聚合图(Integration
Graph)\cite{Kesten:1999:DIG:302392.302397}, 被作为时段演算(Duration Calculus)
\cite{DBLP:journals/ipl/ChaochenHR91}的模型进行研究。其研究成果包括此类自动机
的可达性问题的不可判定\cite{Alur04decisionproblems},以及在这类自动机的一个特殊
子类别上利用基于缩减问题到线性约束满足性问题\cite{Apt:2003:PCP:1237975}时有一个
可判定的过程。在混合自动机\cite{Henzinger96thetheory}的一项判定性研究
\cite{McManis:1994:SAD:647763.735660}中,提到了类似的自动机。同时,一项近似验
证算法的实现工作\cite{Cassez:2000:IPS:646735.701625}里也提到了该自动机。

该自动机由于其时钟暂停的特性有一个形象的名字\pozhehao 秒表自动机(Stopwatch
Automata,SWA)。除了上述的理论性研究以外,还有人做了应用方面的探索
\cite{Abdeddaim:2002:PJS:646486.694487}。

如上文所说,秒表自动机是对时间自动机的拓展。我们在这里给出秒标准自动机的定义。

% SWA definition here
\begin{definition}
	\label{def:SWA}
	一个定义在$\varSigma$上的秒表自动机$B$是指一个元组$(L,l_0,X,\varSigma,E,Inv)$。
	\begin{itemize}
		\item $L$表示位置的有限集。
		\item $l_0\in L$,表示初始位置集合。
		\item $X$表示时钟的有限集。
		\item $\varSigma$表示动作的有限字母表。
		\item $E\subseteq L\times \varPhi_{df}(X)\times \varSigma \times 2^X\times L$,
		表示边的有限集。一条边$(l,\delta,a,r,l')\subseteq E$,
		或者$(l\stackrel{\delta,a,r}{\longrightarrow}l')$,
		表示从位置$l$到$l'$的,标签为$a$,约束为$\delta$,重置为$r\in 2^X$的一
		条变迁。
		\item $Inv\subseteq \varPhi_{ub}(X)^L$,与一个位置绑定,
		表示一个时钟上限约束。
	\end{itemize}
\end{definition}

在秒表自动机的基础上,B{\'e}rard等人针对单执行资源多任务运行的场景提出了中断
时间自动机(Interrupt Timed Automata,ITA)。\cite{Berard:2012:ITA:2158996.2159045}
正如其命名,中断时间自动机在描述中断时非常合适。然而,中断时间自动机要求每个优先
级对应一个时钟,且高优先级的时钟的出现会导致低优先级的时钟停止。这对自动机的语义
表达的限制与本文需要描述的行为相冲突。同时,中断时间自动机对本文将要涉及的部分中
断机制不能很好地描述,因此,本文不采用此类自动机。

用一个单独的状态机来描述一个系统是非常困难的。伴随着时间自动机理论,时间自动机之
间的交互,即时间自动机网络(Network of Timed Automata,NTA)也被提出。
\cite{Alur:1994:TTA:180782.180519}\cite{Bouyer06timedunfoldings}
很多研究都集中在如何将一个时间自动机网络在时间上展开。\cite{Bouyer06timedunfoldings}




来自大阪大学的Makoto Higashi等人提出了一套在一个CPU模拟器上测试中断的方法以
检测中断带来的潜在数据竞争。该方法包含两个方面:第一是在可能产生数据竞争的指令处
触发中断,第二是人为修改外部中断给中断处理程序的输入。中断会在每次读或写内存的指
令之后触发,以此覆盖所有可能产生数据竞争的情况。外部中断的输入序列则是在程序运行
之前就手动准备好。他们已经在\emph{uClinux}上应用此方法做过实验。\cite{Higashi:2010:EMC:1808266.1808278}

