%%% Local Variables:
%%% mode: latex
%%% TeX-master: t
%%% End:

\chapter{相关工作}
\label{cha:related_work}

\section{时间自动机理论}
\label{sec:TA}

时间自动机(Timed Automata,TA)是由Alur R.和D.L. Dill在上世纪九十年代出提出
来的理论。他们用时间自动机来描述实时系统在时间上的行为。他们从形式化语言的角度定
义了时间自动机的语法和语义\cite{Alur:1994:TTA:180782.180519}。该理论是对传统自
动机理论\cite{Hopcroft:2006:IAT:1196416}的一个重要扩展。得益于时间自动机的提出,
自动机模型以及模型检测技术\cite{Clarke:2000:MC:332656}才能将时间纳入系统的考虑。

我们用$X$表示一个变量的有限集,这些变量被称为时钟。
\begin{definition}
	\label{def:constraint}
	对于一个时钟的集合$X$,时钟约束集合$\varPhi(X)$的元素$\delta$的归纳定义如下
	\begin{equation}
		\delta := x \bowtie c ~|~ x-y \bowtie c~|~ \neg\delta ~|~ \delta_1 
		\wedge \delta_2
	\end{equation}
	其中,$x\in X$,$c\in Z$,$\bowtie\in \{<,\leq,=,\geq,>\}$。
\end{definition}
我们用$\varPhi_{df}(X)$表示$\varPhi(X)$的一个子集,在该子集中,不允许$x-y 
\bowtie c$这类约束。类似的,我们定义$\varPhi_{up}(X)$表示$\varPhi(X)$的另一个
子集,在该子集中,只有规定上界的约束$x\prec c$被允许,其中$\prec \in \{<,\leq\}$。

一个时钟取值是指一个函数$v:X\longrightarrow R_+\cup\{0\}$,或者是一个$R_+\cup\{0\}$
上一个$|X|$维向量。我们将所有时钟的取值集合记为$H$,一个自动机的配置是一个二元组
$(l,v)\in L\times H$,包含一个位置和一个时钟取值。每个时钟子集$\rho\subseteq X$
都定义了一个重置函数$r_\rho:H\longrightarrow H$。对每一个时钟取值$v$和每个时钟变
量$x\in X$,有

\begin{equation}
	r_\rho v(x)=
	\begin{cases}
		0 & \mbox{if~$x\in \rho$}\\
		v(x) & \mbox{if~$x\notin \rho$}
	\end{cases}
\end{equation}

即$r_\rho$把集合$\rho$中的所有时钟的值重置为0而保持其他时钟的值不变。接着,我们给
出时间自动机的定义。

\begin{definition}
	\label{def:TA}
	一个定义在$\varSigma$上的时间自动机$A$是指一个元组$(L,l_0,X,\varSigma,E,
	Inv)$。
	\begin{itemize}
		\item $L$表示位置的有限集。
		\item $l_0\in L$,表示初始位置集合。
		\item $X$表示时钟的有限集。
		\item $\varSigma$表示动作的有限字母表。
		\item $E\subseteq L\times \varPhi_{df}(X)\times \varSigma \times 2^X
		\times L$,表示边的有限集。一条边$(l,\delta,a,\rho,l')\subseteq E$,或者
		$(l\stackrel{\delta,a,\rho}{\longrightarrow}l')$,表示从位置$l$到$l'$的,
		标签为$a$,约束为$\delta$,重置为$r_\rho \in 2^X$的一条变迁。
		\item $Inv:L\longrightarrow \varPhi_{ub}(X)$,表示每个位置上的时钟上
		限约束。
	\end{itemize}
\end{definition}

在时间自动机理论中,时钟是一个连续的实数变量。同一个时间自动机系统里的时钟在每个
位置都均匀的增长。时钟的值在一些变迁中可以被重置为0,或者在某些变迁中用时钟的值
的范围作为条件。因此,时间自动机用来描述并行的依赖时间的行为是非常理想的。然而,
在某些情况下,例如抢占式调度,我们需要知道系统在某些位置停留的时间。时钟变量只能
被重置却不能被赋值。这就导致了对时间自动机的扩展。在这项扩展力,时钟在某些位置的
的导数可以是0,即意味着该时钟在该位置下停止计时。这样的自动机被称为聚合图(Integration
Graph)\cite{Kesten:1999:DIG:302392.302397}, 被作为时段演算(Duration Calculus)
\cite{DBLP:journals/ipl/ChaochenHR91}的模型进行研究。其研究成果包括此类自动机
的可达性问题的不可判定\cite{Alur04decisionproblems},以及在这类自动机的一个特殊
子类别上利用基于缩减问题到线性约束满足性问题\cite{Apt:2003:PCP:1237975}时有一个
可判定的过程。在混合自动机\cite{Henzinger96thetheory}的一项判定性研究
\cite{McManis:1994:SAD:647763.735660}中,提到了类似的自动机。同时,一项近似验
证算法的实现工作\cite{Cassez:2000:IPS:646735.701625}里也提到了该自动机。

该自动机由于其时钟暂停的特性有一个形象的名字\pozhehao 秒表自动机(Stopwatch
Automata,SWA)。除了上述的理论性研究以外,还有人做了应用方面的探索
\cite{Abdeddaim:2002:PJS:646486.694487}。

如上文所说,秒表自动机是对时间自动机的拓展。我们在这里给出秒标准自动机的定义。

\begin{definition}
	\label{def:SWA}
	一个定义在$\varSigma$上的秒表自动机$B$是指一个元组$(L,l_0,X,\mu,\varSigma,
	E,Inv)$。
	\begin{itemize}
		\item $L$表示位置的有限集。
		\item $l_0\in L$,表示初始位置集合。
		\item $X$表示时钟的有限集。
		\item $\mu:L\longrightarrow \{0,1\}^{|X|}$,表示每个位置下所有的时钟的导数。
		\item $\varSigma$表示动作的有限字母表。
		\item $E\subseteq L\times \varPhi_{df}(X)\times \varSigma \times 2^X
		\times L$,表示边的有限集。一条边$(l,\delta,a,\rho,l')\subseteq E$,或者
		$(l\stackrel{\delta,a,\rho}{\longrightarrow}l')$,表示从位置$l$到$l'$的,
		标签为$a$,条件为$\delta$,重置为$r_\rho \in 2^X$的一条变迁。
		\item $Inv:L\longrightarrow \varPhi_{ub}(X)$,表示每个位置上的时钟上
		限约束。
	\end{itemize}
\end{definition}

我们用$\mu_l$表示$\mu(l)$,在位置$l$的每个时钟的导数。

\begin{definition}
	\label{def:step}
	一个秒表自动机的\textbf{一步}(step)表示以下其中一种:
	\begin{itemize}
		\item 一个离散步:$(l,v)\stackrel{0}{\longrightarrow}(l',v')$,当存在$e=(l,
		\delta,a,\rho,l')\in E$,使得$v$满足$\delta$且$v'=r_\rho(v)$。
		\item 一个连续步:$(l,v)\stackrel{t}{\longrightarrow}(l,v+t\mu_l), t\in R_+$。
	\end{itemize}
\end{definition}

\begin{definition}
	\label{def:run}
	秒表的自动机的一个\textbf{运行}(run)即一个从$(l_0,v_0)$开始的步有限序列
	\begin{equation}
		\xi: (l_0,v_0)\stackrel{t_1}{\longrightarrow}(l_1,v_1)\stackrel{t_2}
		{\longrightarrow}\dots\stackrel{t_n}{\longrightarrow}(l_n,v_n)
	\end{equation}
\end{definition}

这个运行的逻辑长度就是$n$,其运行时间$|\xi|=t_1+t_2+\dots+t_n$。注意,这里的离
散步(即位置变迁)并没有耗时。

在秒表自动机的基础上,B{\'e}rard等人针对单执行资源多任务运行的场景提出了中断
时间自动机(Interrupt Timed Automata,ITA)。\cite{Berard:2012:ITA:2158996.2159045}
正如其命名,中断时间自动机在描述中断时非常合适。然而,中断时间自动机要求每个优先
级对应一个时钟,且高优先级的时钟的出现会导致低优先级的时钟停止。这对自动机的语义
表达的限制与本文需要描述的行为相冲突。同时,中断时间自动机对本文将要涉及的部分中
断机制不能很好地描述,因此,本文不采用此类自动机。

为了描述一个并发系统,时间自动机的并行组合作为时间自动机的扩展,被加入自动机理论。
在过程运算中,为了建模描述并行过程的不同方面,许多不同的并行组合操作符被提出,例如
CCS\cite{Milner:1989:CC:534666}和CSP\cite{Hoare:1978:CSP:359576.359585}。\uppaal 
建模语言\cite{Larsen97uppaalin}实现了CCS组合操作符。组合允许两个过程$P$和$Q$以
并行的方式组合,记作$P||Q$,这也是过程运算与串行计算模型的关键区别所在。并行组合
允许过程$P$和$Q$同时独立运行。而且,并行组合允许$P$和$Q$之间有交互,即二者可以通
过共享的信道进行同步和信息交互。一个过程可以同时接通多个信道。信道可以是同步或异步
的。同步信道中,信号发送方必须有接收方,发送才能被允许,否则就要一直等待;而异步信
道中,发送方不需要任何接收方即可发送。在某些过程计算中,例如$\pi$运算
\cite{Sangiorgi:2001:PTM:559050},信道本身可以被作为消息通过其他信道发送,这就改
变了过程通信的拓扑结构。

将多个时间自动机通过并行组合整合在一起,我们将这样的系统称之为时间自动机网络(Network 
of Timed Automata,NTA)也被提出。\cite{Alur:1994:TTA:180782.180519,Bouyer06timedunfoldings}
也有部分研究都集中在如何将一个时间自动机网络在时间上展开。\cite{Bouyer06timedunfoldings}
在这个展开的基础上,时间自动机网络可以被转化为一个大型的时间自动机,从而交给工具去
执行各类模型检测算法。\cite{Bengtsson04timedautomata:}

\section{实时性分析}
\label{sec:timing_study}

伴随着时间自动机理论的发展,实时性分析也取得了很大的进步。时至今日,出现了很多优
秀的实时模型检测工具,例如\uppaal \cite{Behrmann04atutorial, Larsen97efficientverification}
和\mycommand{Kronos}\cite{Yovine97kronos:a}。这些工具都以时间自动机的理论为
基础,并将这些理论转化为使用性能很好的算法
\cite{Behrmann:2002:UIS:646847.707113, Behrmann:2006:LUB:1165374.1165376, BehrmannHV00, Behrmann:to}
和数据结构\cite{Behrmann98efficienttimed, Larsen97efficientverification, Larsen:1999:CDD:774455.774459}。

这些工具的成熟性已经被诸多案例所证实。\cite{D'Argenio:1997:BRP:646481.691445,Ernits:2005:MAS:1124427.1124429,Hen05a,Jensen:2000:SUU:646846.706958,Lamport:2005:RMC:2156375.2156396,Larsen:2005:TRE:1086228.1086283,Lindahl:1998:FDA:646482.691449}
从工业界的案例验证到实时控制器和实时通讯协议,这些工具都能很好地处理。尤其是近些
年,通过转化为可达性问题,\uppaal 等模型检测工具被用于解决真实的调度问题。
\cite{Abdeddaim:1,Fehnker:811256,Hune:2001:GSC:774194.774198,Bouyer06timedunfoldings}

\section{中断研究}
\label{sec:intr_study}

在并发程序中,对中断的研究起步要晚于对线程的研究。同为并发场景,中断与线程的差距
很大。\cite{Regehr:2007:IVV:1268075.1268150}中断作为嵌入式/实时系统中的重要组
成部分,给编程带来很多便利,同时使用不当的话也很容易带来很多问题。\cite{Lee:2007:HRE:1534850}

中断的研究中有很大一部分与线程研究的关注点近似,即关注数据竞争之类的错误发生。
\cite{6004502, DBLP:journals/ijdsn/TchamgoueKJ13}来自大阪大学的Makoto Higashi
等人提出了一套在一个CPU模拟器上测试中断的方法以检测中断带来的潜在数据竞争。该方法
包含两个方面:第一是在可能产生数据竞争的指令处触发中断,第二是人为修改外部中断给
中断处理程序的输入。中断会在每次读或写内存的指令之后触发,以此覆盖所有可能产生数
据竞争的情况。外部中断的输入序列则是在程序运行之前就手动准备好。他们已经在
\emph{uClinux}上应用此方法做过实验。\cite{Higashi:2010:EMC:1808266.1808278}

由于中断与硬件的关系十分紧密,也有一些验证方案设计了结合硬件的设置。\cite{Yang:4261310}
还有一些研究集中在中断相关的内存情况。\cite{Chatterjee:2003:SSA:1760267.1760276}

除了对中断的功能性分析验证的研究以外,针对中断性能或者时间性质的研究也越来越多。
有些研究试图从截止时间的角度来验证中断系统的实时性\cite{10.1109/TSE.2004.64},
有些则从上下文切换次数的角度来分析\cite{6148915}。

总得来说,中断这部分的研究并不没有自动机理论本身来得深入,还有很多细节值得探究。


