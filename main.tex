%%% Local Variables:
%%% mode: latex
%%% TeX-master: t
%%% End:

\documentclass[master]{thuthesis}

\usepackage{thutils}
\usepackage{tabu}

%% Settings for add uppaal code 
\usepackage{pstricks}
\usepackage{latexsym,multicol,color}
\usepackage{listings} % comes with texlive−latex−recomended

\lstdefinelanguage{Uppaal}{ % syntax highlight via font
	basicstyle=\small\sffamily, % small sans−serif font (like verdana)
	keywords={after update,assign,before update,break,case,const,continue,
		default,else,enum,for,guard,if,meta,process,progress,return,select,
		state,sync,switch,trans,system,while},
	keywords={[2]broadcast,bool,clock,chan,commit,init,int,scalar,struct,
		typedef,urgent,void}, keywordstyle={[2]\bfseries},
	keywords={[3]false,true}, otherkeywords={[3]−>},
	morekeywords={[3]−>}, keywordstyle={[3]\bfseries},
	comment=[l]{//}, morecomment=[s]{/∗}{∗/}, % single and multi−line
	commentstyle=\itshape, % appear in italic
	tabsize=4, % tab treatment (going to be fixed in Uppaal)
	captionpos=b, % put captions at the bottom
	escapechar=@ % write LaTeX comments escaped by @ symbol
}
\lstdefinelanguage[GUI]{Uppaal}[]{Uppaal}{ % syntax like in GUI
	keywordstyle={[2]\color{black!50!green}}, % slightly darker than in GUI
	otherkeywords={−>}, keywordstyle={[3]\color{magenta}},
	commentstyle={\color{black!50!red}\itshape}, % dark red
	literate={{−−>}{$−−>$}3} % fix arrows
}
\lstdefinelanguage[LIT]{Uppaal}[GUI]{Uppaal}{ % replace some symbols
	literate={{−>}{{$\leadsto$} }2 {−−>}{{$\longrightarrow$} }2
		{=}{{$\gets$ }}2 {==}{{$==$}}2 {:=}{{$\gets$ }}2 {<=}{{$\leq$ }}2
		{>=}{{$\geq$ }}2 {&&}{{$\land$}}2 {||}{{$\lor$}}2 {<>}{{$\Diamond$}}1
		{[]}{{$\Box$}}1 {forall}{{$\forall$}}1 {exists}{{$\exists$}}1}
}
\lstset{language={[GUI]Uppaal}, % use GUI flavor
	columns={[l]flexible},
	frame=single, rulesepcolor=\color{gray},
	numbers=left,xleftmargin=4em,xrightmargin=4em, aboveskip=1em}
%% End of Setting

\usepackage[shortlabels]{enumitem}
\setlist[description]{style=multiline, leftmargin=7em, labelindent=\parindent}

%% to make use of property
\usepackage[amsmath,thmmarks]{ntheorem}
\newtheorem{property}{性质}[chapter]

\def\mycommand{\bgroup\obeyspaces\mycommandaux}
\def\mycommandaux#1{\mycommandauxii #1\relax\relax\egroup}
\def\mycommandauxii#1{%
	\ifx\relax#1\else \ifcat#1\@sptoken{} \expandafter\expandafter\expandafter\mycommandauxii\else
	\ifnum`#1=\uccode`#1 {\normalsize #1}\else {\footnotesize \uppercase{#1}}\fi \expandafter\expandafter\expandafter\mycommandauxii\expandafter\fi\fi}
\def\uppaal {\mycommand{Uppaal}}

% 你可以在这里修改配置文件中的定义,导言区可以使用中文。
\def\myname{刘盛鹏}
\def\intel{Intel\textsuperscript{\textregistered}}
\def\fpri{get\_highest\_pir}

\begin{document}
	
% 定义所有的eps文件在 figures 子目录下
\graphicspath{{figure/}}


%%% 封面部分
\frontmatter

\ctitle{基于Uppaal的\\中断实时性分析与验证}
% 根据自己的情况选,不用这样复杂
\makeatletter
\ifthu@bachelor\relax\else
  \ifthu@doctor
    \cdegree{工学博士}
  \else
    \ifthu@master
      \cdegree{工学硕士}
    \fi
  \fi
\fi
\makeatother


\cdepartment[软件学院]{软件学院}
\cmajor{软件工程}
\cauthor{刘盛鹏} 
\csupervisor{顾明教授}
% 如果没有副指导老师或者联合指导老师,把下面两行相应的删除即可。
\cassosupervisor{贺飞教授}
% 日期自动生成,如果你要自己写就改这个cdate
%\cdate{\CJKdigits{\the\year}年\CJKnumber{\the\month}月}

% 博士后部分
% \cfirstdiscipline{计算机科学与技术}
% \cseconddiscipline{系统结构}
% \postdoctordate{2009年7月——2011年7月}

\etitle{Timing Analysis on Interrupt\\Based on Uppaal} 
% 这块比较复杂,需要分情况讨论:
% 1. 学术型硕士
%    \edegree:必须为Master of Arts或Master of Science(注意大小写)
%              “哲学、文学、历史学、法学、教育学、艺术学门类,公共管理学科
%               填写Master of Arts,其它填写Master of Science”
%    \emajor:“获得一级学科授权的学科填写一级学科名称,其它填写二级学科名称”
% 2. 专业型硕士
%    \edegree:“填写专业学位英文名称全称”
%    \emajor:“工程硕士填写工程领域,其它专业学位不填写此项”
% 3. 学术型博士
%    \edegree:Doctor of Philosophy(注意大小写)
%    \emajor:“获得一级学科授权的学科填写一级学科名称,其它填写二级学科名称”
% 4. 专业型博士
%    \edegree:“填写专业学位英文名称全称”
%    \emajor:不填写此项
\edegree{Master of Engineering} 
\emajor{Software Engineering} 
\eauthor{Liu Shengpeng} 
\esupervisor{Professor Gu Min} 
\eassosupervisor{Professor He Fei} 
% 这个日期也会自动生成,你要改么?
% \edate{December, 2005}

% 定义中英文摘要和关键字
\begin{cabstract}
   待完成
\end{cabstract}

\ckeywords{待完成}

\begin{eabstract} 
   To be filled
\end{eabstract}

\ekeywords{To be filled}

\makecover

% 目录
\tableofcontents

% 符号对照表
\begin{denotation}

\item[IRQ] 中断请求(Interrupt Request)

\end{denotation}



%%% 正文部分
\mainmatter
%%% Local Variables:
%%% mode: latex
%%% TeX-master: t
%%% End:

\chapter{中断分析的现状}
本章介绍中断的概念和现在学术界对中断的研究。就是文献综述。
\section{什么是中断}
\subsection{中断与线程}
\subsection{嵌入式系统中的中断}

\section{中断的研究现状}
%%% Local Variables:
%%% mode: latex
%%% TeX-master: t
%%% End:

\chapter{中断机制}
%%% Local Variables:
%%% mode: latex
%%% TeX-master: t
%%% End:

\chapter{基于Uppaal的中断模型}

\section{Uppaal中的模型组成}	
一套Uppaal模型由以下三部分组成。
\begin{itemize}
	\item \emph{声明}:整个模型系统中共有的声明,可以是变量或函数。在整个
	系统中都可以访问。
	\item \emph{自动机模板}:各类自动机的通用模板,一个模型系统可以有多个
	模板,一个模板在系统中可以对应多个实例。
		\begin{enumerate}[(1)]
			\item \emph{声明}:模板内部的变量或函数,只有本模板的实例可
			以访问。
			\item \emph{位置}:时间自动机的位置,每个位置可以有初始(
			initial),紧急(urgent),关键(committed)。关键位置与紧
			急位置上,模型中的时钟都停止。不同的是,当有自动机在关键位置时,
			在下一个状态迁移必须从某一个关键位置发出。
			\item \emph{变迁}:位置到位置的迁移。变迁包含选择(select)
			、条件(guard)、同步(sync)、更新(update)四个属性。其中,
			同步和更新是同时发生的。
		\end{enumerate}	
	\item \emph{模型声明}:定义组成系统的模板实例。
\end{itemize}

\section{中断基本模型}
\subsection{通用的硬件实现}
\subsection{Uppaal中的基本中断模型}

\section{带重入的中断模型}
\subsection{重入的硬件实现}
\subsection{Uppaal中的重入中断模型}

\section{分段中断模型}
\subsection{软件的二次实现}
\subsection{Uppaal中的分段中断模型}
%%% Local Variables:
%%% mode: latex
%%% TeX-master: t
%%% End:

\chapter{应用——某嵌入式软件的中断时间性质分析和验证}
\label{cha:case}

\section{某嵌入式软件平台}
\label{sec:senario}

\subsection{中断设置}
\label{subsec:intr_setting}

\subsection{抽象}
\label{subsec:abstract}

\section{Intel8259}
\label{sec:8259}

\section{在Uppaal中构建中断模型}
\label{sec:build}

\section{分析与验证}
\label{sec:experiment}
%%% Local Variables:
%%% mode: latex
%%% TeX-master: t
%%% End:

\chapter{总结与展望}
\label{cha:sum}

本文提出并实践了一套完整的利用形式化方法进行嵌入式程序里中断实时性研究的方案。

经过大量的文献阅读和项目实践,本文提出了三种常见中断类型。在详细分析了这三类中断
的实现机制,尤其是对分段中断的软件二次实现进行了深入的理解之后。之后,本文给出了
在扩展的秒表自动机的定义之下,三类中断的时间自动机模型,以及中断系统对应的时间自
动机网络的定义。

本文所提出的三种中断类型覆盖了现在市面上绝大多数软硬件的中断机制,在应用到具体的
项目中时,只需要结合具体的平台和项目需求,稍作修改即可用于中断的时间性质分析。

在第~\ref{cha:case} 章中,本文结合实际项目,应用提出的中断模型以及模型构建的方
法对某航空控制系统的中断进行了建模分析。在\uppaal 中,每个中断都被合适
的时间自动机模型表示。通过\uppaal 集成的模拟器和验证器,本文找出了该程序中断设置
的一个漏洞,针对该漏洞找到一条反例,并得到项目合作方工程师的认可。

本文的主要贡献有:

\begin{itemize}
	\item 提出利用形式化方法进行面向嵌入式程序里中断实时性研究的方案
	\item 归纳现有的各类中断,并给出其时间自动机模型
	\item 针对某嵌入式平台的实时软件系统的中断模型和对其实时性的验证
\end{itemize}

然而,本文的工作有需要改进的地方。本文提出的三种中断模型中,重入和分段的特征在实
际的应用中是可能同时存在一个中断上的。融合这两个特征会使中断自动机更为复杂。也许
可以有其他的建模思路来简洁地表示同时具有重入和分段两个特征的中断。

分段中断在现阶段的实现基本都是靠软件,各家实现了分段中断的实时操作系统的实现机制
在细节上各有不同。即使是同一个操作系统,在不同的硬件平台上的实现细节也因为必须依
赖硬件的行为而有所不同。本文在建模时只深入研究了少数几个实时操作系统在少数硬件平
台上的实现,该分段模型的应用的范围也许不如另外两个模型那么广泛。而且随着实现方案
的不断演进,即使在这么抽象的层次,该模型可能也会需要随之修改。

最后,在应用到某嵌入式平台的中断驱动程序中时,由于部分中断的触发次数太多导致状态
空间过大,以致部分性质无法在可以容忍的时间和经济成本内得到验证。不过这也许不仅仅
通过抽象建模阶段的工作就可以的解决的。也许我们还需要\uppaal 背后依托的时间自动机
理论和模型检测这项技术的原理来改善,直到解决这个问题。



%%% 其它部分
\backmatter

% 参考文献
\bibliographystyle{thubib}
\bibliography{ref/reference}

% 致谢
%%% Local Variables:
%%% mode: latex
%%% TeX-master: "../main"
%%% End:

\begin{ack}

感谢我的导师在我工作和撰写论文期间给与的指导和帮助。感谢郭心睿同学、潘晓梦同学和
王前学长在研究eCos中断机制和HAL代码原理时对我的帮助。感谢在我硕士期间和我一起奋
斗各个战场的战友们。

\end{ack}

%% 附录
%\begin{appendix}
%	\input{data/appendix01}
%\end{appendix}
%
%% 个人简历
\begin{resume}

\resumeitem{个人简历}

1989 年 12 月 30 日出生于 江苏 省 南通 市。

2008 年 8 月考入 清华大学 大学 软件学院 计算机软件 专业,2012 年 7 月本科毕业并获得 工学 学士学位。

2012 年 9 月进入 清华 大学 软件学院 攻读 工学硕士 学位至今。

\resumeitem{发表的学术论文} % 发表的和录用的合在一起

\begin{enumerate}[{[}1{]}]
   	\item Shengpeng Liu, Vania Joloboff, Fei He. Automated Generation of Instruction Set Simulator from Specification. In press. (已被 2015 the 7th International Conference on Communication Software and Networks (ICCSN 2015) 录用. )
\end{enumerate}

\end{resume}

\end{document}